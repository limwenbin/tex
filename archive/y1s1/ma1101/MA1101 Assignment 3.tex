\documentclass[12pt]{article}
\usepackage[margin=1in]{geometry} 
\usepackage{mathtools,amsmath,amsthm,amssymb,amsfonts}

\newcommand*\diff{\mathop{}\!\mathrm{d}}
\newcommand*\Diff[1]{\mathop{}\!\mathrm{d^#1}} 
\newcommand{\N}{\mathbb{N}}
\newcommand{\Z}{\mathbb{Z}}
 
\newenvironment{problem}[2][Problem]{\begin{trivlist}
\item[\hskip \labelsep {\bfseries #1}\hskip \labelsep {\bfseries #2.}]}{\end{trivlist}}
 
\begin{document}

\title{MA1101R Homework Assignment 3}
\author{Lim Wen Bin \\
A0140764H\\
T02}
\maketitle

\begin{problem}{1.i}
\end{problem}
\begin{proof}
\begin{align*}
V &= \text{span}(S)\\
\intertext{$S$ is a basis for $V$ $\Leftrightarrow$ $S$ is linearly independent,}
A &= 
\left( \begin{array}{ccc}
s_1 & s_2 & s_3\\ 
\end{array} \right)\\
=
\left( \begin{array}{ccc}
2 & 5 & 1\\ 
3 & 2 & -1\\ 
1 & 0 & 9\\ 
4 & 1 & 9\\ 
0 & 2 & 2\\ 
\end{array} \right) \xrightarrow{\substack{R_3 \leftrightarrow R_1}} 
&\left( \begin{array}{ccc}
1 & 0 & 9\\ 
3 & 2 & -1\\ 
2 & 5 & 1\\ 
4 & 1 & 9\\ 
0 & 2 & 2\\ 
\end{array} \right) \xrightarrow{\substack{R_2 - 3R_1,\\ R_3 - 2R_1,\\ R_4 - 4R_1}} 
\left( \begin{array}{ccc}
1 & 0 & 9\\ 
0 & 2 & -28\\ 
0 & 5 & -17\\ 
0 & 1 & -27\\ 
0 & 2 & 2\\ 
\end{array} \right) \\
\xrightarrow{\substack{R_2 \times 1/2}}
\left( \begin{array}{ccc}
1 & 0 & 9\\ 
0 & 2 & -14\\ 
0 & 5 & -17\\ 
0 & 1 & -27\\ 
0 & 2 & 2\\ 
\end{array} \right) 
\xrightarrow{\substack{R_3 - 5R_2,\\ R_4 - R_2,\\ R_5 - 2R_2}}
&\left( \begin{array}{ccc}
1 & 0 & 9\\ 
0 & 1 & -14\\ 
0 & 0 & 53\\ 
0 & 0 & -13\\ 
0 & 0 & 30\\ 
\end{array} \right) 
\xrightarrow{\substack{R_3 \times 1/53,\\ R_4 + 13R_3,\\ R_5 - 30R_3}}
\left( \begin{array}{ccc}
1 & 0 & 9\\ 
0 & 1 & -14\\ 
0 & 0 & 1\\ 
0 & 0 & 0\\ 
0 & 0 & 0\\ 
\end{array} \right) \\
\intertext{$ax = 0$ only has trivial solution $\rightarrow s$ is linearly independent and is a basis for $v$.}
\end{align*}
\end{proof}
\filbreak

\begin{problem}{1.ii}
\end{problem}
\begin{proof}
\begin{align*}
\left( \begin{array}{c|c|c|c}
a & t_1 & t_2 & t_3\\ 
\end{array} \right)
&=
\left( \begin{array}{ccc|c|c|c}
2 & 5 & 1 & 0 & 9 & 7\\ 
3 & 2 & -1 & -1 & 8 & 7\\
1 & 0 & 9 & 14 & 2 & -27\\ 
4 & 1 & 9 & 15 & 9 & -25\\ 
0 & 2 & 2 & 2 & 2 & -2\\ 
\end{array} \right) \\
\xrightarrow{\substack{R_3 \leftrightarrow R_1}} 
\left( \begin{array}{ccc|c|c|c}
1 & 0 & 9 & 14 & 2 & -27\\ 
3 & 2 & -1 & -1 & 8 & 7\\
2 & 5 & 1 & 0 & 9 & 7\\ 
4 & 1 & 9 & 15 & 9 & -25\\ 
0 & 2 & 2 & 2 & 2 & -2\\ 
\end{array} \right) 
&\xrightarrow{\substack{R_2 - 3R_1,\\ R_3 - 2R_1,\\ R_4 - 4R_1}} 
\left( \begin{array}{ccc|c|c|c}
1 & 0 & 9 & 14 & 2 & -27\\ 
0 & 2 & -28 & -43 & 2 & 88\\
0 & 5 & -17 & -28 & 5 & 61\\ 
0 & 1 & -27 & -41 & 1 & 83\\ 
0 & 2 & 2 & 2 & 2 & -2\\ 
\end{array} \right) \\
\xrightarrow{\substack{R_2 \times 1/2}} 
\left( \begin{array}{ccc|c|c|c}
1 & 0 & 9 & 14 & 2 & -27\\ 
0 & 1 & -14 & -43/2 & 1 & 44\\
0 & 5 & -17 & 28 & 5 & 61\\ 
0 & 1 & -27 & -41 & 1 & 83\\ 
0 & 2 & 2 & 2 & 2 & -2\\ 
\end{array} \right) 
&\xrightarrow{\substack{R_3 - 5R_2,\\ R_4 - R_2,\\ R_5 - 2R_2}}
\left( \begin{array}{ccc|c|c|c}
1 & 0 & 9 & 14 & 2 & -27\\ 
0 & 1 & -14 & -43/2 & 1 & 44\\
0 & 0 & 53 & 159/2 & 0 & -159\\ 
0 & 0 & -13 & -39/2 & 0 & 39\\ 
0 & 0 & 30 & 45 & 0 & -90\\ 
\end{array} \right)\\ 
\xrightarrow{\substack{R_3 \times 1/53,\\ R_4 + 13R_3,\\ R_5 - 30R_3}}
\left( \begin{array}{ccc|c|c|c}
1 & 0 & 9 & 14 & 2 & -27\\ 
0 & 1 & -14 & -43/2 & 1 & 44\\
0 & 0 & 1 & 3/2 & 0 & -3\\ 
0 & 0 & 0 & 0 & 0 & 0 \\
0 & 0 & 0 & 0 & 0 & 0 \\
\end{array} \right)
&\xrightarrow{\substack{R_2 + 14R_3,\\ R_1 - 9R_3}}
\left( \begin{array}{ccc|c|c|c}
1 & 0 & 0 & 1/2 & 2 & 0\\ 
0 & 1 & 0 & -1/2 & 1 & 2\\
0 & 0 & 1 & 3/2 & 0 & -3\\ 
0 & 0 & 0 & 0 & 0 & 0 \\
0 & 0 & 0 & 0 & 0 & 0 \\
\end{array} \right)\\
\therefore \text{span}(t) &\subseteq v\\
\left( \begin{array}{ccc}
(t_1)_s & (t_2)_s & (t_3)_s\\
\end{array} \right) &=
\left( \begin{array}{ccc}
1/2 & 2 & 0\\ 
-1/2 & 1 & 2\\
3/2 & 0 & -3\\ 
\end{array} \right)\\
\xrightarrow{\substack{R_2 + R_1,\\ R_3 - 3R_1}} 
\left( \begin{array}{ccc}
1/2 & 2 & 0\\ 
0 & 3 & 2\\
0 & -6 & -3\\ 
\end{array} \right)
&\xrightarrow{\substack{R_3 + 2R_2}} 
\left( \begin{array}{ccc}
1/2 & 2 & 0\\ 
0 & 3 & 2\\
0 & 0 & 1\\ 
\end{array} \right)
\intertext{$\therefore (t_1)_s, (t_2)_s, (t_3)_s$ are linearly independent and dim(span$\{(t_1)_s, (t_2)_s, (t_3)_s\}) = $ dim$(v) = 3 \rightarrow t$ is a basis for $v$.}
\end{align*}
\end{proof}
\filbreak

\begin{problem}{1.iii}
\end{problem}
\begin{align*}
P_{S \to T}
&=
P_{T \to S}^{-1}\\
\left( \begin{array}{ccc|ccc}
1/2 & 2 & 0 & 1 & 0 & 0\\ 
-1/2 & 1 & 2 & 0 & 1 & 0\\
3/2 & 0 & -3 & 0 & 0 & 1\\ 
\end{array} \right)
&\xrightarrow{\substack{R_1 \times 2}} 
\left( \begin{array}{ccc|ccc}
1 & 4 & 0 & 2 & 0 & 0\\ 
-1/2 & 1 & 2 & 0 & 1 & 0\\
3/2 & 0 & -3 & 0 & 0 & 1\\ 
\end{array} \right)\\
\xrightarrow{\substack{R_2 + 1/2R_1,\\ R_3 - 3/2R_1}} 
\left( \begin{array}{ccc|ccc}
1 & 4 & 0 & 2 & 0 & 0\\ 
0 & 3 & 2 & 1 & 1 & 0\\
0 & -6 & -3 & -3 & 0 & 1\\ 
\end{array} \right)
&\xrightarrow{\substack{R_2 \times 1/3}} 
\left( \begin{array}{ccc|ccc}
1 & 4 & 0 & 2 & 0 & 0\\ 
0 & 1 & 2/3 & 1/3 & 1/3 & 0\\
0 & -6 & -3 & -3 & 0 & 1\\ 
\end{array} \right)\\
\xrightarrow{\substack{R_3 + 6R_2,\\ R_3 - 3/2R_1}} 
\left( \begin{array}{ccc|ccc}
1 & 4 & 0 & 2 & 0 & 0\\ 
0 & 1 & 2/3 & 1/3 & 1/3 & 0\\
0 & 0 & 1 & -2 & 2 & 1\\ 
\end{array} \right)
&\xrightarrow{\substack{R_2 - 2/3R_3,\\ R_1 - 4R_1}} 
\left( \begin{array}{ccc|ccc}
1 & 0 & 0 & -2 & 4 & 8/3\\ 
0 & 1 & 0 & 1 & -1 & -2/3\\
0 & 0 & 1 & -1 & 2 & 1\\ 
\end{array} \right)\\
P_{T \to S} &= 
\left( \begin{array}{ccc}
-2 & 4 & 8/3\\ 
1 & -1 & -2/3\\
-1 & 2 & 1\\ 
\end{array} \right)
\end{align*}
\filbreak

\begin{problem}{1.iv}
\end{problem}
\begin{align*}
P_{T \to S}
&=
\left( \begin{array}{ccc}
1/2 & 2 & 0\\ 
-1/2 & 1 & 2\\
3/2 & 0 & -3\\ 
\end{array} \right)
\end{align*}
\filbreak

\begin{problem}{1.v}
\end{problem}
\begin{align*}
w &= \left( \begin{array}{ccc}
t_1 & t_2 & t_3\\ 
\end{array} \right)
\left( \begin{array}{c}
2\\
-3\\
1\\ 
\end{array} \right)\\
&= 
\left( \begin{array}{c}
-20\\
-19\\
-5\\
-22\\
-4\\ 
\end{array} \right)\\
(w)_S &= P_{T \to S} \cdot (w)_T\\
&=
\left( \begin{array}{ccc}
1/2 & 2 & 0\\ 
-1/2 & 1 & 2\\
3/2 & 0 & -3\\ 
\end{array} \right)
\left( \begin{array}{c}
2\\
-3\\
1\\ 
\end{array} \right)\\
&=
\left( \begin{array}{c}
5\\
-2\\
0\\ 
\end{array} \right)
\end{align*}
\filbreak

\begin{problem}{2.i}
\end{problem}
\begin{align*}
A
=
\left( \begin{array}{cccccc}
2 & 8 & -16 & -6 & 12 & 6\\ 
-3 & 11 & 24 & 8 & -17 & -9\\
3 & 13 & -24 & -10 & 19 & 9\\ 
-1 & -2 & 6 & 1 & -4 & -3\\ 
\end{array} \right)
&\xrightarrow{\substack{R_1 \times 1/2}} 
\left( \begin{array}{cccccc}
1 & 4 & -8 & -3 & 6 & 3\\ 
-3 & 11 & 24 & 8 & -17 & -9\\
3 & 13 & -24 & -10 & 19 & 9\\ 
-1 & -2 & 6 & 1 & -4 & -3\\ 
\end{array} \right)\\
\xrightarrow{\substack{R_2 + 3R_1,\\R_3 - 3R_1,\\R_4 + 1R_1}} 
\left( \begin{array}{cccccc}
1 & 4 & -8 & -3 & 6 & 3\\ 
0 & 1 & 0 & -1 & 1 & 0\\
0 & 1 & 0 & -1 & 1 & 0\\ 
0 & 2 & -2 & -2 & 2 & 0\\ 
\end{array} \right)
&\xrightarrow{\substack{R_3 - R_1,\\R_4 - 2R_1}} 
\left( \begin{array}{cccccc}
1 & 4 & -8 & -3 & 6 & 3\\ 
0 & 1 & 0 & -1 & 1 & 0\\
0 & 0 & 0 & 0 & 0 & 0\\ 
0 & 0 & -2 & 0 & 0 & 0\\ 
\end{array} \right)\\
\xrightarrow{\substack{R_4 \times -1/2,\\ R_4\leftrightarrow R_3}} 
\left( \begin{array}{cccccc}
1 & 4 & -8 & -3 & 6 & 3\\ 
0 & 1 & 0 & -1 & 1 & 0\\
0 & 0 & 1 & 0 & 0 & 0\\ 
0 & 0 & 0 & 0 & 0 & 0\\ 
\end{array} \right)\\
A^T
=
\left( \begin{array}{cccc}
2 & -3 & 3 & -1\\ 
8 & -11 & 13 & -2\\
-16 & 24 & -24 & 6\\ 
-6 & 8 & -10 & 1\\ 
12 & -17 & 19 & -4\\ 
6 & -9 & 9 & -3\\ 
\end{array} \right)
&\xrightarrow{\substack{R_1 \times 1/2}} 
\left( \begin{array}{cccc}
1 & -3/2 & 3/2 & -1/2\\ 
8 & -11 & 13 & -2\\
-16 & 24 & -24 & 6\\ 
-6 & 8 & -10 & 1\\ 
12 & -17 & 19 & -4\\ 
6 & -9 & 9 & -3\\ 
\end{array} \right)\\
\xrightarrow{\substack{R_2 - 8R_1,\\R_3 + 16R_1,\\R_4 + 6R_1,\\R_5 - 12R_1,\\R_6 - 6R_1}} 
\left( \begin{array}{cccc}
1 & -3/2 & 3/2 & -1/2\\ 
0 & 1 & 1 & 2\\
0 & 0 & 0 & -2\\ 
0 & -1 & -1 & -2\\ 
0 & 1 & 1 & 2\\ 
0 & 0 & 0 & 0 \\
\end{array} \right)
&\xrightarrow{\substack{R_4 + R_2,\\R_5 - R_2}} 
\left( \begin{array}{cccc}
1 & -3/2 & 3/2 & -1/2\\ 
0 & 1 & 1 & 2\\
0 & 0 & 0 & -2\\ 
0 & 0 & 0 & 0\\ 
0 & 0 & 0 & 0\\ 
0 & 0 & 0 & 0 \\
\end{array} \right)\\
\xrightarrow{\substack{R_3 \times -1/2}} 
\left( \begin{array}{cccc}
1 & -3/2 & 3/2 & -1/2\\ 
0 & 1 & 1 & 2\\
0 & 0 & 0 & 1\\ 
0 & 0 & 0 & 0\\ 
0 & 0 & 0 & 0\\ 
0 & 0 & 0 & 0 \\
\end{array} \right)
\end{align*}
\filbreak

\begin{problem}{2.ii}
\end{problem}
\begin{align*}
\intertext{$\because$ ref$(A)$ has pivots in the first three columns,}
\text{Columns of $A$ that form basis for C$(A)$}
&=
\left( \begin{array}{ccc}
2 & 8 & -16\\ 
-3 & 11 & 24\\
3 & 13 & -24\\ 
-1 & -2 & 6\\ 
\end{array} \right)\\
\intertext{From ref$(A^T)$, C$(A)$ = span$\{ (1, -3/2, 3/2, -1/2), (0, 1, 1, 2),(0, 0, 0, 1)\}$. Then,}
\left( \begin{array}{ccc}
2 & 8 & -16\\ 
-3 & 11 & 24\\
3 & 13 & -24\\ 
-1 & -2 & 6\\ 
\end{array} \right)
&\cup
\left( \begin{array}{c}
0\\ 
0\\
1\\ 
0\\ 
\end{array} \right)
\intertext{would be a basis for $\mathbb{R}^4$.}
\end{align*}
\filbreak

\begin{problem}{2.iii}
\end{problem}
\begin{align*}
\intertext{$\because$ ref$(A^T)$ has pivots in the first second and fourth column,}
\text{Rows of $A$ that form basis for C$(A)$}
&=
\left( \begin{array}{cccccc}
2 & 8 & -16 & -6 & 12 & 6\\ 
-3 & 11 & 24 & 8 & -17 & -9\\
-1 & -2 & 6 & 1 & -4 & -3\\ 
\end{array} \right)\\
\intertext{From ref$(A)$, R$(A)$ = span$\{ (1, 4, -8, -3, 6, 3), (0, 1, 0, -1, 1, 0), (0, 0, 1, 0, 0, 0)\}$. Then,}
\left( \begin{array}{cccccc}
2 & 8 & -16 & -6 & 12 & 6\\ 
-3 & 11 & 24 & 8 & -17 & -9\\
-1 & -2 & 6 & 1 & -4 & -3\\ 
\end{array} \right)
&\cup
\left( \begin{array}{cccccc}
0 & 0 & 0 & 1 & 0 & 0\\ 
0 & 0 & 0 & 0 & 1 & 0\\ 
0 & 0 & 0 & 0 & 0 & 1\\ 
\end{array} \right)
\intertext{would be a basis for $\mathbb{R}^6$.}
\end{align*}
\filbreak

\begin{problem}{2.iv}
\end{problem}
\begin{align*}
\text{ref$(A)$}
&=
\left( \begin{array}{cccccc}
1 & 4 & -8 & -3 & 6 & 3\\ 
0 & 1 & 0 & -1 & 1 & 0\\
0 & 0 & 1 & 0 & 0 & 0\\ 
0 & 0 & 0 & 0 & 0 & 0\\ 
\end{array} \right)\\
\text{rref$(A)$}
&=
\left( \begin{array}{cccccc}
1 & 0 & 0 & 1 & 2 & 3\\ 
0 & 1 & 0 & -1 & 1 & 0\\
0 & 0 & 1 & 0 & 0 & 0\\ 
0 & 0 & 0 & 0 & 0 & 0\\ 
\end{array} \right)\\
Ax = 0 \Rightarrow x &= 
a\left( \begin{array}{c}
-1\\ 
1\\
0\\
1\\
0\\ 
0\\ 
\end{array} \right)
+ b\left( \begin{array}{c}
-2\\ 
-1\\
0\\
0\\
1\\ 
0\\ 
\end{array} \right)
+ c\left( \begin{array}{c}
-3\\ 
0\\
0\\
0\\
0\\ 
1\\ 
\end{array} \right),
\quad a, b, c \in \mathbb{R}
\intertext{$\therefore$ basis for nullspace of $A$ = $\{(-1, 1, 0, 1, 0, 0), (-2, -1, 0, 0, 1, 0), (-3, 0, 0, 0, 0, 1)\}$}
\end{align*}
\filbreak

\begin{problem}{2.v}
\end{problem}
\begin{align*}
\text{ref$(A^T)$}
&=
\left( \begin{array}{cccc}
1 & -3/2 & 3/2 & -1/2\\ 
0 & 1 & 1 & 2\\
0 & 0 & 0 & 1\\ 
0 & 0 & 0 & 0\\ 
0 & 0 & 0 & 0\\ 
0 & 0 & 0 & 0 \\
\end{array} \right)\\
\text{rref$(A^T)$}
&=
\left( \begin{array}{cccc}
1 & 0 & 3 & 0 \\
0 & 1 & 1 & 0\\
0 & 0 & 0 & 1\\ 
0 & 0 & 0 & 0\\ 
0 & 0 & 0 & 0\\ 
0 & 0 & 0 & 0 \\
\end{array} \right)\\
Ax = 0 \Rightarrow x &= 
a\left( \begin{array}{c}
-3\\ 
-1\\
1\\
0\\
\end{array} \right),
\quad a \in \mathbb{R}
\intertext{$\therefore$ basis for nullspace of $A^T$ = $\{(-3, -1, 1, 0)\}$}
\end{align*}
\filbreak

\begin{problem}{3}
\end{problem}
\begin{proof}
\begin{align*}
\intertext{For any $v \in \mathbb{R}^n$,}
v &= [v]_E\\
\intertext{Let $Av = u$.}
\text{RHS} &= P^{-1}AP[v]_S\\
&= P^{-1}A[v]_E \quad \text{($\because P$ is the transition matrix from $S$ to $E$)}\\
&= P^{-1}Av\\
&= P^{-1}u\\
&= P^{-1}[u]_E\\
&= [u]_S \quad \text{($\because P^{-1}$ is the transition matrix from $E$ to $S$)}\\
&= [Av]_S\\
&= \text{LHS}\\
\end{align*}
\end{proof}
\filbreak

\begin{problem}{4.i}
\end{problem}
\begin{proof}
\begin{align*}
\text{N$(B)$} &= \{x \in \mathbb{R}^p: Bx = 0\}\\
\text{N$(AB)$} &= \{x \in \mathbb{R}^p: ABx = 0\}\\
\text{rank$(AB)$} = \text{rank$(AB)$} &\Rightarrow \text{nullity$(AB)$} = \text{nullity$(B))$}\\
\intertext{Suppose $Bx_0 = 0$,}
ABx_0 &= A(Bx_0)\\
&= A0\\
&= 0
\intertext{$\therefore$ N$(B) \subseteq$ N$(AB)$. Suppose $ABx_1 = 0$ and $Bx_1 = v, v \ne 0$,}
ABx_1 = Av = 0 &\Rightarrow \text{nullity}(AB) = \text{nullity}(B) + 1 
\intertext{This is a contradiction and $ABx = 0 \Rightarrow Bx = 0$ $\Rightarrow$ N$(AB) \subseteq$ N$(B)$. $\therefore$ N$(AB) =$ N$(B)$.}
\end{align*}
\end{proof}
\filbreak

\begin{problem}{4.ii}
\end{problem}
\begin{proof}
\begin{align*}
ABu &= ABv\\
ABu - ABv = 0\\
A(Bu - Bv) = 0\\
AB(u - v) = 0\\
\intertext{Then by 4(i),}
B(u - v) = 0\\
Bu - Bv = 0\\
Bu = Bv \\
\end{align*}
\end{proof}
\filbreak

\begin{problem}{5.i}
\end{problem}
\begin{proof}
\begin{align*}
\intertext{Let rank$(A)$ be $\alpha$, rank$(B)$ be $\beta$.}
AB &= \left( \begin{array}{cccc}
Ab_1 & Ab_2 & \cdots & Ab_p\\
\end{array} \right)\\
\intertext{$B$ contains $\beta$ linearly independent vectors.}
\therefore \text{nullity}(A) &\ge \beta\\
n - \text{rank}(A) &\ge \text{rank}(B) \quad (\because n = \text{rank}(A) + \text{nullity}(A))\\
\text{rank}(A) + \text{rank}(B) &\le n
\end{align*}
\end{proof}
\filbreak

\begin{problem}{5.ii}
\end{problem}
\begin{proof}
\begin{align*}
A^2 &= A\\
A - A^2 &= 0\\
A(I - A) &= 0\\
\shortintertext{By 5(i),}
\text{rank}(A) + \text{rank}(I - A) &\le n\\
I &= (I - A) + A\\
\intertext{Suppose C$(I - A) =$ span$(S)$ and C$(A) =$ span$(T)$.}
\text{C}(I) &= \text{span}(S \cup T)\\
\text{dim(C$(I)$))} &\le \text{dim}(S) + \text{dim}(T)\\
\therefore \text{rank}(I) &\le \text{rank}(I - A) + \text{rank}(A)\\
\text{rank}(A) + \text{rank}(I - A) &\ge n \quad (\because \text{rank}(I) = n)\\
\therefore \text{rank}(A) + \text{rank}(I - A) &= n\\
\end{align*}
\end{proof}
\filbreak

\end{document}