\documentclass[12pt]{article}
\usepackage[margin=1in]{geometry} 
\usepackage{mathtools,amsmath,amsthm,amssymb,amsfonts}

\newcommand*\diff{\mathop{}\!\mathrm{d}}
\newcommand*\Diff[1]{\mathop{}\!\mathrm{d^#1}} 
\newcommand{\N}{\mathbb{N}}
\newcommand{\Z}{\mathbb{Z}}
 
\newenvironment{problem}[2][Problem]{\begin{trivlist}
\item[\hskip \labelsep {\bfseries #1}\hskip \labelsep {\bfseries #2.}]}{\end{trivlist}}
 
\begin{document}

\title{MA1101R Tutorial 10}
\author{Lim Wen Bin \\
A0140764H\\
T02}
\maketitle

\begin{problem}{1.a}
\end{problem}
\begin{align*}
(\lambda - 1)(\lambda + 1) - 3 &= 0\\
\lambda^2 &= 4\\
\lambda &= \pm 2\\
E_2 = x :
&\left( \begin{array}{cc}
1 & 3\\
1 & 3\\
\end{array} \right) x = 0 \\
\left( \begin{array}{cc}
1 & 3\\
1 & 3\\
\end{array} \right)
&\xrightarrow{G.E}
\left( \begin{array}{cc}
1 & 3\\
0 & 0\\
\end{array} \right)\\
x &= 
s\left( \begin{array}{c}
-3\\
1\\
\end{array} \right) , s \in \mathbb{R}\\
E_2 &= \text{span}\left\{
\left( \begin{array}{c}
-3\\
1\\
\end{array} \right) \right\} \\
E_{-2} = x :
&\left( \begin{array}{cc}
-3 & 3\\
1 & -1\\
\end{array} \right) x = 0 \\
\left( \begin{array}{cc}
-3 & 3\\
1 & -1\\
\end{array} \right)
&\xrightarrow{G.E}
\left( \begin{array}{cc}
1 & -1\\
0 & 0\\
\end{array} \right)\\
x &= 
s\left( \begin{array}{c}
1\\
1\\
\end{array} \right) , s \in \mathbb{R}\\
E_{-2} &= \text{span}\left\{
\left( \begin{array}{c}
1\\
1\\
\end{array} \right) \right\} \\
\intertext{$A$ has two independent eigenvectors $\Rightarrow$ $A$ is diagonalizable.}
P &= 
\left( \begin{array}{cc}
1 & -3\\
1 & 1\\
\end{array} \right)\\
\end{align*}
\filbreak

\begin{problem}{1.b}
\end{problem}
\begin{align*}
(\lambda - 1)^3(\lambda - 2) &= 0\\
\lambda &= \pm 1, \lambda = 2 \\
E_2 = x :
&\left( \begin{array}{cccc}
1 & 0 & 0 & 0\\
1 & 1 & 0 & 0\\
2 & 0 & 1 & 0\\
5 & 0 & 1 & 0\\
\end{array} \right) x = 0 \\
\left( \begin{array}{cccc}
1 & 0 & 0 & 0\\
1 & 1 & 0 & 0\\
2 & 0 & 1 & 0\\
5 & 0 & 1 & 0\\
\end{array} \right)
&\xrightarrow{G.E}
\left( \begin{array}{cccc}
1 & 0 & 0 & 0\\
0 & 1 & 0 & 0\\
0 & 0 & 1 & 0\\
0 & 0 & 0 & 0\\
\end{array} \right)\\
x &= 
s\left( \begin{array}{c}
0\\
0\\
0\\
1\\
\end{array} \right) , s \in \mathbb{R}\\
E_2 &= \text{span}\left\{
\left( \begin{array}{c}
0\\
0\\
0\\
1\\
\end{array} \right) \right\} \\
E_{1} = x :
&\left( \begin{array}{cccc}
0 & 0 & 0 & 0\\
1 & 0 & 0 & 0\\
2 & 0 & 0 & 0\\
5 & 0 & 1 & 1\\
\end{array} \right) x = 0 \\
\left( \begin{array}{cccc}
0 & 0 & 0 & 0\\
1 & 0 & 0 & 0\\
2 & 0 & 0 & 0\\
5 & 0 & 1 & 1\\
\end{array} \right)
&\xrightarrow{G.E}
\left( \begin{array}{cccc}
1 & 0 & 0 & 0\\
0 & 0 & 1 & 1\\
0 & 0 & 0 & 0\\
0 & 0 & 0 & 0\\
\end{array} \right)\\
x &= 
s\left( \begin{array}{c}
0\\
1\\
0\\
0\\
\end{array} \right) + 
t\left( \begin{array}{c}
0\\
0\\
-1\\
1\\
\end{array} \right) , s , t \in \mathbb{R}\\
E_{-2} &= \text{span}\left\{
\left( \begin{array}{c}
0\\
1\\
0\\
0\\
\end{array} \right),
\left( \begin{array}{c}
0\\
0\\
-1\\
1\\
\end{array} \right)
 \right\} \\
\intertext{$A$ only has three independent eigenvectors $\Rightarrow$ $A$ is not diagonalizable.}
\end{align*}
\filbreak

\end{document}