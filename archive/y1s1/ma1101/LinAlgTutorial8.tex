\documentclass[12pt]{article}
\usepackage[margin=1in]{geometry} 
\usepackage{amsmath,amsthm,amssymb,amsfonts}
 
\newcommand{\N}{\mathbb{N}}
\newcommand{\Z}{\mathbb{Z}}
 
\newenvironment{problem}[2][Problem]{\begin{trivlist}
\item[\hskip \labelsep {\bfseries #1}\hskip \labelsep {\bfseries #2.}]}{\end{trivlist}}
 
\begin{document}

\title{MA1101R Linear Algebra Tutorial 8}
\author{Lim Wen Bin}
\maketitle
 
\begin{problem}{1.a}
\end{problem}
\begin{align*}
\intertext{Let $S = \{s_0, s_1, s_2, s_3\}$, $T = \{t_0, t_1, t_2, t_3\}$ and $V = \{v_0, v_1, v_2, v_3\}$ be a orthogonal basis for $S$,}
v_0 &= (0, 1, 0, 1)\\
t_0 &= (0, 1/\sqrt{2}, 0, 1/\sqrt{2}) \\
v_1 &= s_1 - \frac{s_1 \cdot v_0}{v_0 \cdot v_0}v_0\\
&= (-4, 0, 3, 0) - (0, 0, 0, 0)\\
&= (-4, 0, 3, 0) \\
t_1 &= (-4/5, 0, 3/5, 0)\\ 
v_2 &= s_2 - \frac{s_2 \cdot v_0}{v_0 \cdot v_0}v_0 - \frac{s_2 \cdot v_1}{v_1 \cdot v_1}v_1 \\
&= (3, -1, 4, -1) - (0, -1, 0 , -1) - (0, 0, 0, 0)\\
&= (3, 0, 4, 0)\\
t_2 &= (3/5, 0, 4/5, 0)\\
v_3 &=  s_3 - \frac{s_3 \cdot v_0}{v_0 \cdot v_0}v_0 - \frac{s_3 \cdot v_1}{v_1 \cdot v_1}v_1 - \frac{s_3 \cdot v_2}{v_2 \cdot v_2}v_2 \\
&= (0, 0, 0, 2) - (0, 1, 0, 1) - (0 , 0, 0, 0)- (0 , 0, 0, 0)\\
&= (0, -1, 0, 1)\\
t_3 &= (0, -1/\sqrt{2}, 0, 1/\sqrt{2}) \\
\end{align*}

\begin{problem}{1.b}
\end{problem}
\begin{align*}
\left( \begin{array}{cccc}
0 & -4/5 & 3/5 & 0 \\
1/\sqrt{2} & 0 & 0 & -1/\sqrt{2}\\ 
0 & 3/5 & 4/5 & 0\\
1/\sqrt{2} & 0 & 0 & 1/\sqrt{2} \\
\end{array}
\right)
\left( \begin{array}{c}
u_0\\ 
u_1\\ 
u_2\\ 
u_3\\ 
\end{array}
\right) &= 
\left( \begin{array}{c}
w\\ 
x\\ 
y\\ 
z\\ 
\end{array}
\right)\\
\left( \begin{array}{c}
u_0\\ 
u_1\\ 
u_2\\ 
u_3\\ 
\end{array}
\right) &=
\left( \begin{array}{cccc}
0 & -4/5 & 3/5 & 0 \\
1/\sqrt{2} & 0 & 0 & -1/\sqrt{2}\\ 
0 & 3/5 & 4/5 & 0\\
1/\sqrt{2} & 0 & 0 & 1/\sqrt{2} \\
\end{array}
\right)^T
\left( \begin{array}{c}
w\\ 
x\\ 
y\\ 
z\\ 
\end{array}
\right)\\
&= \left( \begin{array}{c}
x/\sqrt{2} + z/\sqrt{2}\\ 
-4w/5 + 3y/5\\ 
3w/5 + 4y/5\\ 
-x/\sqrt{2} + z/\sqrt{2}\\ 
\end{array}
\right)
\end{align*}

\begin{problem}{2.a}
\end{problem}
\begin{align*}
\text{$W$ is a subspace of $\mathbb{R}$} &\Rightarrow W = \text{span}\{v_1, v_2, \ldots, v_k\} \\
\text{$u$ is orthogonal to $W$} &\Leftrightarrow \text{$u.v_i = 0$ for $i = 1, 2, \ldots, k$}\\
\text{Let }A &= \left( \begin {array}{c} v_1 \\ v_ 2 \\ \vdots \\ v_4 \end{array} \right)\\
\forall u \in W^\bot, \quad Au &= 0
\intertext{$\because u$ is a solution to a homogeneous linear system, $W^\bot$ is a subspace of $\mathbb{R}^n$.}
\end{align*}

\begin{problem}{2.b.i}
\end{problem}
\begin{align*}
(a, a + b, b, b + c, c) &= a(1, 1, 0, 0, 0) + b(0, 1, 1, 1, 0,) + c(0, 0, 0, 1, 1)\\
W &= \text{span}\{(1, 1, 0, 0, 0), (0, 1, 1, 1, 0,), (0, 0, 0, 1, 1)\} \\
A &= \left( \begin{array}{ccccc}
1 & 1 & 0 & 0 & 0 \\
0 & 1 & 1 & 1 & 0 \\
0 & 0 & 0 & 1 & 1
\end{array}
\right)\\
W^\bot &= \{u | Au = 0 \}\\
\text{rref}(A) &= 
\left( \begin{array}{ccccc}
1 & 0 & -1 & 0 & 1 \\
0 & 1 & 1 & 0 & -1 \\
0 & 0 & 0 & 1 & 1
\end{array}
\right)\\
W^\bot &= \{ (d-e, e-d, d, -e, e)| d, e \in \mathbb{R}\}
\end{align*}

\begin{problem}{2.b.ii}
\end{problem}
\begin{align*}
S &= \{ (1/\sqrt{2}, 1/\sqrt{2}, 0, 0, 0), \\
& \qquad (-1/\sqrt{10}, 1/\sqrt{10}, \sqrt{2/5}, \sqrt{2/5}, 0), \\
& \qquad(1/2\sqrt{10}, -1/2\sqrt{10}, -1/\sqrt{10}, 3/2\sqrt{10}, \sqrt{5/2}/2)\}\\
T &= \{ (1\sqrt{3}, -1\sqrt{3}, 1\sqrt{3}, 0, 0), (-1/2\sqrt{6}, 1/2\sqrt{6}, -\sqrt{3/2}/2, 1\sqrt{6}, \sqrt{3/2}/2)\}\\
\intertext{Yes $S \cup U$ is a orthonormal basis for $\mathbb{R}^5$ as dim$(T) = 2$ and dim$(S) = 3$ and $S \bot U$. Solution: prove that $v_i \cdot v_j = 0$ for $i \ne j$ and $v_i \cdot v_j = 1$ for $i = j$}
\end{align*}

\begin{problem}{3.a}
\end{problem}
\begin{align*}
\left( \begin{array}{ccc}
1 & 0 & 1 \\
2 & 1 & 0 \\
0 & 1 & -2 \\
1 & 2 & -3 
\end{array} \right)
&\xrightarrow{G.E.}
\left( \begin{array}{ccc}
1 & 0 & 1 \\
0 & 1 & -2 \\
0 & 0 & 0 \\
0 & 0 & 0 
\end{array} \right)\\
\text{basis for $V$} & = \{(1, 0, 1), (0, 1, -2)\}
\end{align*}

\begin{problem}{3.b.i}
\end{problem}
\begin{align*}
\intertext{Let $S = \{s_1, s_2\}$ be an orthonormal basis for $V$.}
s_1 &= (1/\sqrt{2}, 0 ,1/\sqrt{2})\\
v_2 - \frac{v_2 \cdot s_1} {s_1 \cdot s_1} s_1 &= (0, 1, -2) + (1, 0, 1)\\
&= (1, 1, -1)\\
s_2 &= (1/\sqrt{3}, 1/\sqrt{3}, -1/\sqrt{3})\\
\text{projection of $(1, 1, 1)$ onto $V$} &= ((1,1,1) \cdot s_1)s_1 + ((1,1,1) \cdot s_2)s_2\\
&= (1, 0, 1) + (1/3, 1/3, -1/3)\\
&= (\frac{4}{3}, \frac{1}{3}, \frac{2}{3})
\end{align*}

\begin{problem}{3.b.ii}
\end{problem}
\begin{align*}
A &= \left( \begin{array}{cc}
1 & 0\\
0 & 1\\
1 & -2
\end{array} \right) \\
b &= \left( \begin{array}{cc}
1\\
1\\
1
\end{array} \right) \\
A^TAu &= A^Tb\\
\left( \begin{array}{ccc}
2 & -2\\
-2 & 5
\end{array} \right) u &=
\left( \begin{array}{c}
2\\
-1\\
\end{array} \right) \\
u &= \left( \begin{array}{c}
\frac{4}{3}\\
\frac{1}{3}\\
\end{array} \right) \\
p &= Au\\
&= \left( \begin{array}{cc}
\frac{4}{3}\\
\frac{1}{3}\\
\frac{2}{3}
\end{array} \right)
\end{align*}

\begin{problem}{4.a}
\end{problem}
\begin{align*}
\text{$S$ is a basis for $V$} &\Rightarrow \text{$\{u_1, u_2, u_3, u_4\}$ is L.I.}\\
T = AS\\
\left( \begin{array}{cccc} v_1 \\ v_2 \\ v_3 \\ v_4 \end{array} \right) &=
\left( \begin{array}{cccc} 
0.5 & 0.5 & 0.5 & 0.5 \\
0.5 & 0.5 & -0.5 & -0.5 \\
0.5 & -0.5 & -0.5 & 0.5 \\
0.5 & -0.5 & 0.5 & -0.5 \\
\end{array} \right)
\left( \begin{array}{cccc} u_1 \\ u_2 \\ u_3 \\ u_4 \end{array} \right)\\
A &\xrightarrow{G.E.} 
\left( \begin{array}{cccc} 
1 & 1 & 1 & 1 \\
0 & 1 & 1 & 0 \\
0 & 0 & 1 & 1 \\
0 & 0 & 0 & 1 \\
\end{array} \right) \Rightarrow \text{$AS = T$ is L.I.}
\intertext{$\because$ dim$(T)$ = dim$(S)$ = 4 and $T$ is L.I., T is a basis for $V$.}
\end{align*}

\begin{problem}{4.b}
\end{problem}
\begin{align*}
\text{$S$ is orthonormal} &\Leftrightarrow S^TS = I\\
T &= AS\\
T^TT &= (AS)^TAS\\
&= S^TA^TAS\\
&= S^T 
\left( \begin{array}{cccc} 
0.5 & 0.5 & 0.5 & 0.5 \\
0.5 & 0.5 & -0.5 & -0.5 \\
0.5 & -0.5 & -0.5 & 0.5 \\
0.5 & -0.5 & 0.5 & -0.5 \\
\end{array} \right)^T
\left( \begin{array}{cccc} 
0.5 & 0.5 & 0.5 & 0.5 \\
0.5 & 0.5 & -0.5 & -0.5 \\
0.5 & -0.5 & -0.5 & 0.5 \\
0.5 & -0.5 & 0.5 & -0.5 \\
\end{array} \right) S\\
&= S^T I S\\
&= S^TS\\
&= I
\intertext{$\therefore T$ is orthonormal.}
\end{align*}

\begin{problem}{5.a.i}
\end{problem}
\begin{align*}
\left( \begin{array}{cc|c}
1 & 1 & 3\\
1 & 2 & 4\\
1 & 0 & 2
\end{array} \right) &\xrightarrow{G.E.}
\left( \begin{array}{cc|c}
1 & 0 & 2\\
0 & 1 & 1\\
0 & 0 & 0
\end{array} \right)\\
x &= \left( \begin{array}{c}
2\\
1\\
0
\end{array} \right)
\end{align*}

\begin{problem}{5.a.ii}
\end{problem}
\begin{align*}
A^TAx &= A^Tb\\
\left( \begin{array}{cc}
3 & 3\\
3 & 5\\
\end{array} \right)
x &= 
\left( \begin{array}{ccc}
1 & 1 & 1\\
1 & 2 & 0
\end{array} \right)
\left( \begin{array}{c}
3\\
4\\
2
\end{array} \right)\\
x = \left( \begin{array}{cc}
3 & 3\\
3 & 5\\
\end{array} \right)^{-1} 
\left( \begin{array}{c}
9\\
11
\end{array} \right)\\
&= \left( \begin{array}{c}
2\\
1
\end{array} \right)
\end{align*}

\begin{problem}{5.b}
\end{problem}
\begin{align*}
\intertext{To prove solution set of (1) $Ax = b$ equal to solution set of (2) $A^TAb = A^Tb$ $\Leftrightarrow$ (1) $\subseteq$ (2) $\land$ (2) $\subseteq$ (1).}
\intertext{(1) $\subseteq$ (2):}
Ax &= b\\
A^TAx &= A^Tb \quad \text{(Taking pre-multiplication by $A^T$ both sides.)}
\intertext{(2) $\subseteq$ (1):}
A^TAx &= b \\ 
x &= x_p + x_g\\
\intertext{For particular solution $x_p$, $Ax = b$ is consistent $\Rightarrow$ b is in column space of $A$. Then a particular solution is equal to the solution for the projection of $b$ onto the column space of $A$.}
\therefore A^TAx_p = A^Tb &\Rightarrow Ax_p = b \\
\intertext{For general solution $x_g$,}
A^TAx_g &= 0\\
x_g^TA^TAx_g &= 0\\
(Ax_g)^TAx_g  &= 0\\
||Ax_g|| &= 0\\
Ax_g &= 0\\
\therefore A^TA(x_p+x_g) = A^Tb &\Rightarrow Ax = b
\end{align*}

\end{document}