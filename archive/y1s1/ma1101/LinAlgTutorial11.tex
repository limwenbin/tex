\documentclass[12pt]{article}
\usepackage[margin=1in]{geometry} 
\usepackage{mathtools,amsmath,amsthm,amssymb,amsfonts}

\newcommand*\diff{\mathop{}\!\mathrm{d}}
\newcommand*\Diff[1]{\mathop{}\!\mathrm{d^#1}} 
\newcommand{\N}{\mathbb{N}}
\newcommand{\Z}{\mathbb{Z}}
 
\newenvironment{problem}[2][Problem]{\begin{trivlist}
\item[\hskip \labelsep {\bfseries #1}\hskip \labelsep {\bfseries #2.}]}{\end{trivlist}}
 
\begin{document}

\title{MA1101R Tutorial 11}
\author{Lim Wen Bin \\
A0140764H\\
T02}
\maketitle

\begin{problem}{1.a}
\end{problem}
\begin{align*}
\intertext{Let $A$ be the standard matrix for the linear transformation $T_1$.}
A &=
\left( \begin{array}{cc}
1 & 0\\
2 & 0\\
\end{array} \right)\\
\end{align*}
\filbreak

\begin{problem}{1.b}
\end{problem}
\begin{align*}
T_2 (\mathbf{0}) 
&= 
\left( \begin{array}{c}
2\\
1\\
\end{array} \right)\\
&\ne
\left( \begin{array}{c}
0\\
0\\
\end{array} \right)
\intertext{$\therefore T_2$ is not a linear transformation.}
\end{align*}
\filbreak

\begin{problem}{1.c}
\end{problem}
\begin{align*}
\intertext{Let $A$ be the standard matrix for the linear transformation $T_3$.}
A &=
\left( \begin{array}{ccc}
0 & 0 & 0\\
-1 & 1 & 0\\
0 & 1 & -1\\
\end{array} \right)\\
\end{align*}
\filbreak

\begin{problem}{1.d}
\end{problem}
\begin{align*}
T_4 (ax_1 + bx_2) &= (ax_1 + bx_2)^T A y \quad a, b \in \mathbb{R}\\
&= (ax_1^T + bx_2^T) A y\\
&= ax_1^T A y + bx_2^T A y\\
&= a(x_1^T A y) + b(x_2^T A y)\\
&= T_4(x_1) + T_4(x_2) \\
\intertext{$\therefore T_4$ is a linear transformation.}
\intertext{Let $B$ be the standard matrix for the linear transformation $T_4$.}
T_4(x) &= x^T A y\\
&= Bx\\
x^T A y &= (x^T A y)^T \quad (\because x^T A y \in \mathbb{R})\\
&= y^T A^T x\\
\therefore B &= y^T A^T
\end{align*}
\filbreak

\begin{problem}{1.a}
\end{problem}
\begin{align*}
T_5 (ax_1 + bx_2) &= (ax_1 + bx_2)^T A (ax_1 + bx_2) \quad a, b \in \mathbb{R}\\
&= (ax_1^T + bx_2^T) A (ax_1 + bx_2)\\
&= ax_1^T A (ax_1 + bx_2) + bx_2^T A (ax_1 + bx_2) \\
&= ax_1^T A ax_1 + ax_1^T A bx_2 + bx_2^T A ax_1 + bx_2^T A bx_2 \\
&= a(x_1^T A x_1) + ax_1^T A bx_2 + bx_2^T A ax_1 + b(x_2^T A x_2) \\
&= aT_5(x_1) + ax_1^T A bx_2 + bx_2^T A ax_1 + bT_5(x_2) \\
&\ne aT_5(x_1) + bT_5(x_2) \\
\intertext{$\therefore T_4$ is not a linear transformation.}
\end{align*}
\filbreak

\begin{problem}{2.a}
\end{problem}
\begin{align*}
(T_1 + T_2)(au + bv) &= T_1(au + bv) + T_2(au + bv) \quad a, b \in \mathbb{R}\\
&= aT_1(u) + bT_1(v) + aT_2(u) + bT_2(v)\\
&= a[T_1(u) + T_2(u)] + b[T_1(v) + T_2(v)]\\
&= a(T_1 + T_2)(u) + b(T_1 + T_2)(v)\\
\intertext{$\therefore (T_1+T_2)$ is a linear transformation.}
\intertext{Let $C$ be the standard matrix for the linear transformation $(T_1 + T_2)$.}
(T_1 + T_2)(u) &= T_1(u) + T_2(u)\\ 
&= Au + Bu\\
&= (A + B)u\\
\therefore C &= A+B
\end{align*}
\filbreak

\begin{problem}{2.b}
\end{problem}
\begin{align*}
(\lambda T)(au + bv) &= \lambda T(au + bv) \quad a, b \in \mathbb{R}\\
&= \lambda (aT(u) + bT(v)) \\
&= a\lambda T(u) + b\lambda T(v) \\
&= a(\lambda T)(u) + b(\lambda T)(v) \\
\intertext{$\therefore (\lambda T)$ is a linear transformation.}
\intertext{Let $B$ be the standard matrix for the linear transformation $(\lambda T)$.}
(\lambda T)(u) &= \lambda T(u)\\ 
&= \lambda Au\\
\therefore B &= \lambda A
\end{align*}
\filbreak

\begin{problem}{3.a}
\end{problem}
\begin{proof}
\begin{align*}
P(ax_1 + bx_2) &= (ax_1 + bx_2) - (n \cdot (ax_1 + bx_2)) n\quad a, b \in \mathbb{R}\\
&= ax_1 - (n \cdot (ax_1)) n + bx_2 - (n \cdot (bx_2)) n\\
&= a(x_1 - (n \cdot x_1) n) + b(x_2 - (n \cdot x_2) n)\\
&= aP(x_1) + bP(x_2)\\
\intertext{$\therefore P$ is a linear transformation.}
\intertext{Let $A$ be the standard matrix for the linear transformation $P$.}
P(x) &= Ax\\
P(x) &= x - (n \cdot x)n\\
&= x - n(n \cdot x) \quad (\text{$\because n \cdot x$ is a scalar}) \\
&= x - n(n^T x) \\
&= Ix - nn^T x\\
&= (I - nn^T) x\\
\therefore A &= (I - nn^T)
\end{align*}
\end{proof}
\filbreak

\begin{problem}{3.b}
\end{problem}
\begin{proof}
\begin{align*}
(P \circ P)(x) &= P (P(x))\\
&= P(x - (n \cdot x)n))\\
&= x - (n \cdot x)n - (n \cdot (x - (n \cdot x)n))n \\
&= x - (n \cdot x)n - (n \cdot x)n + [n \cdot ((n \cdot x)n)]n \\
&= x - (n \cdot x)n - (n \cdot x)n + (n \cdot x)[n \cdot n]n \quad (\text{$\because n \cdot x$ is a scalar}) \\
&= x - (n \cdot x)n - (n \cdot x)n + (n \cdot x)n \quad (\because n \cdot n = 1) \\
&= x - (n \cdot x)n\\
&= P(X)\\
\therefore P \circ P &= P
\end{align*}
\end{proof}
\filbreak

\begin{problem}{4.a}
\end{problem}
\begin{align*}
\intertext{Let $A$ be the standard matrix for the linear transformation $T$.}
\text{rref}(A) &=
\left( \begin{array}{ccccc}
1 & 2 & 0 & 0 & -3\\
0 & 0 & 1 & 0 & 0\\
0 & 0 & 0 & 1 & 2\\
0 & 0 & 0 & 0 & 0\\
\end{array} \right)\\
\text{Range}(T) &= \text{C}(A)\\
&= \text{span} \left\{
\left( \begin{array}{c}
1\\
2\\
-1\\
0\\
\end{array} \right) ,
\left( \begin{array}{c}
3\\
2\\
-3\\
2\\
\end{array} \right) ,
\left( \begin{array}{c}
4\\
4\\
-2\\
2\\
\end{array} \right) 
\right\}\\
\end{align*}
\filbreak

\begin{problem}{4.b}
\end{problem}
\begin{align*}
\text{Ker}(T) &= \text{N}(A)\\
&= \text{span} \left\{
\left( \begin{array}{c}
-2\\
1\\
0\\
0\\
0\\
\end{array} \right) ,
\left( \begin{array}{c}
3\\
0\\
0\\
-2\\
1\\
\end{array} \right) 
\right\}\\
\end{align*}
\filbreak

\begin{problem}{4.c}
\end{problem}
\begin{align*}
\text{dim(Range($T$))} + \text{dim(Ker($T$))} &= 3 + 2\\
&= 5 = n
\end{align*}
\filbreak

\begin{problem}{5.a}
\end{problem}
\begin{align*}
T(e_n) &= Ae_n\\
&= a_n \text{, where $a_n$ is the $n$th column of A.}\\
\{T(e_i)&:i \in \mathbb{Z}, i \in [1,n]\} \text{ is orthogonal}\\
\Rightarrow T(e_i) \cdot T(e_j) &= \begin{cases}
0, & i \ne j\\
1, & i=j\\
\end{cases}\\
\Rightarrow a_i \cdot a_j &= \begin{cases}
0, & i \ne j\\
1, & i=j\\
\end{cases}\\
A^T A &= I
\intertext{$\therefore A$ is orthogonal.}
\end{align*}
\filbreak

\begin{problem}{5.b}
\end{problem}
\begin{align*}
T(u) &= u_1T(e_1) + u_2T(e_2)\\
\intertext{$\therefore T$ is a rotation on $\mathbb{R}^2$}
\end{align*}
\filbreak

\end{document}