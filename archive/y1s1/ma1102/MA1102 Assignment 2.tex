\documentclass[12pt]{article}
\usepackage[margin=1in]{geometry} 
\usepackage{mathtools,amsmath,amsthm,amssymb,amsfonts}

\newcommand*\diff{\mathop{}\!\mathrm{d}}
\newcommand*\Diff[1]{\mathop{}\!\mathrm{d^#1}} 
\newcommand{\N}{\mathbb{N}}
\newcommand{\Z}{\mathbb{Z}}
 
\newenvironment{problem}[2][Problem]{\begin{trivlist}
\item[\hskip \labelsep {\bfseries #1}\hskip \labelsep {\bfseries #2.}]}{\end{trivlist}}
 
\begin{document}

\title{MA1102R Homework Assignment 2}
\author{Lim Wen Bin \\
A0140764H\\
T03}
\maketitle

\begin{problem}{1}
\end{problem}
\begin{proof}
\begin{align*}
\intertext{Let $f(x) = x^5 + 4x^2 - 2$, $f$ is continuous on $\mathbb{R} \because$ it is a polynomial.}
f'(x) &= 5x^4 + 8x\\
&= x(5x^3 + 8)\\
f'(x) &= 0 \Leftrightarrow x = 0 \text{ or } x = -\frac{2}{5^{1/3}} \tag{1}\label{1.1}\\
f'(x) &\begin{cases}
	> 0, & x < -\frac{2}{5^{1/3}}\\
	\le 0, & -\frac{2}{5^{1/3}} \le x \le 0\\
	> 0, & 0 < x
\end{cases}\\
\intertext{By IVT,}
f(-3) = -209 \land f(-1) = 1 &\Rightarrow \exists c_0 \in [-3,-1] \text{ such that } f(c_0) = 0\\
f(-1) = 1 \land f(0) = -2 &\Rightarrow \exists c_1 \in [-1,0] \text{ such that } f(c_1) = 0\\
f(0) = -2 \land f(1) = 3 &\Rightarrow \exists c_2 \in [0,1] \text{ such that } f(c_2) = 0\\
\intertext{$\therefore$ $f$ has at least three real roots. Suppose $f$ has another real root at $c_3 \ne c_0, c_1, c_2$, and let $V_0, V_1, V_2$ denote the open interval between successive real roots. Then by Rolle's Theorem,}
\exists s_i \in V_i, i \in \{0, 1, 2\} &: f'(s_i)= 0\\
\intertext{However, this contradicts \eqref{1.1}. $\therefore$ $f$ has exactly three real roots.}
\end{align*}
\end{proof}
\filbreak

\begin{problem}{2.i}
\end{problem}
\begin{align*}
f(x) &= (x+2)^{1/3} (x-7)^{2/3} \\
f'(x) &= \frac{(x-7)^{2/3}}{3(x+2)^{2/3}} + \frac{2(x+2)^{1/3}}{3(x-7)^{1/3}} \\
f'(x) &= \frac{(x-7) + 2(x+2)} {3(x+2)^{2/3} (x-7)^{1/3}} \\
f'(x) &= \frac{x-1} {[(x+2)^2 (x-7)]^{1/3}} \quad x \ne -2, x \ne 7 \\
[(x+2)^2 (x-7)]^{1/3} &< 0 \text{ when } x < 7\\
x-1 &\le 0 \text{ when } x \le 1\\
f'(x) &\begin{cases}
	> 0 & x < 1\\
	= 0 & x = 1\\
	< 0 & 1 < x < 7\\
	> 0 & 7 < x
\end{cases}
\intertext{$\therefore f$ is increasing for $x \in (-\infty, 1) \cup (7, \infty)$ and decreasing for $x \in (1, 7)$.}
\end{align*}
\filbreak

\begin{problem}{2.ii}
\end{problem}
\begin{align*}
\shortintertext{At a critical point,}
f'(x) &= 0 \lor f'(x) \text{ undefined}\\
x_1 &= -2 \text{ is not a turning point as $f$ is increasing}\\
x_2 &= 1 \text{ is a maximum as $f$ changes from increasing to decreasing}\\
x_3 &= 7 \text{ is a minimum as $f$ changes from decreasing to increasing}\\
f(1) &= (3 \cdot (-6)^2)^{1/3}\\
&= 3(4)^{1/3}\\
f(7) &= 0\\
\intertext{Coordinates of maximum point$:(1, 3(4)^{1/3})$, Coordinates of minimum point$:(7, 0)$}\\
\end{align*}
\filbreak

\begin{problem}{2.iii}
\end{problem}
\begin{align*}
f''(x) &= \frac{d}{dx} \frac{x-1} {[(x+2)^2 (x-7)]^{1/3}} \\
&= \frac{ [(x+2)^2 (x-7)]^{1/3} - \frac{1}{3} (x-1) [ 2(x+2)(x-7) + (x+2)^2 ] [(x+2)^2 (x-7)]^{-2/3} } {[(x+2)^2 (x-7)]^{2/3}} \\ 
&= \frac{ [(x+2)^2 (x-7)]^{-2/3} \{(x+2)^2 (x-7)- \frac{1}{3} (x-1) [ 2(x+2)(x-7) + (x+2)^2 ]\}} {[( x+2)^2 (x-7)]^{2/3}} \\ 
&= \frac{ (x+2)^2 (x-7) - \frac{1}{3} (x-1) [ 2(x+2)(x-7) + (x+2)^2 ]} {[(x+2)^2 (x-7)]^{4/3}} \\ 
&= \frac{ (x+2)^2 (x-7) - \frac{1}{3} (x-1) (x+2) [ 2(x-7) + x + 2 ]} {[(x+2)^2 (x-7)]^{4/3}} \\ 
&= \frac{ (x+2) [ 3(x+2)(x-7) - (x-1) (3x-12)]} {3[(x+2)^2 (x-7)]^{4/3}} \\ 
&= \frac{  3(x^2 + 2x -7x -14) - (3x^2 -3x - 12x + 12)} {3 (x+2)^{5/3} (x-7)^{4/3}} \\ 
&= \frac{ 3x^2  -15x -42 - 3x^2 +15x - 12} {3 (x+2)^{5/3} (x-7)^{4/3}} \\ 
&= \frac{ -54} {3 (x+2)^{5/3} (x-7)^{4/3}} \\
&= -\frac{18} {(x+2)^{5/3} (x-7)^{4/3}} \\
&(x+2)^{5/3} (x-7)^{4/3} > 0 \text{ when } x < -2\\
&(x+2)^{5/3} (x-7)^{4/3} < 0 \text{ when } x > -2\\
f''(x) &\begin{cases}
	> 0, & x < -2\\
	< 0, & -2 > x\\
\end{cases}
\intertext{$\therefore f$ is concave up on $(-\infty, -2)$ and concave down on $(-2, \infty)$.}
\end{align*}
\filbreak

\begin{problem}{2.iv}
\end{problem}
\begin{align*}
\intertext{$f$ changes from concave up to concave down at $x=-2$.}
f(-2) &= 0\\
&= \lim_{x \to -2} f(x)\\
\intertext{$\therefore f$ is continuous at $x=-2$ and coordinates of inflexion point $= (-2, 0)$.}
\end{align*}
\filbreak

\begin{problem}{3.i}
\end{problem}
\begin{proof}
\begin{align*}
\intertext{Let $g(x) = f(x)\sin x + f'(x)\cos x$.}
g(0) &= f(0) \sin 0 + f'(0) \cos 0\\
&= 0 + 1\\
&= 1\\
g'(x) &= f'(x) \sin x + f(x) \cos x + f''(x) \cos x - f'(x) \sin x\\
g'(x) &= f'(x) \sin x + f(x) \cos x - f(x) \cos x - f'(x) \sin x \qquad \text{($\because f''(x) = -f(x)$)}\\
&= 0
\intertext{$\therefore g$ is a constant function and $g(x) = 1$, $\forall x \in \mathbb{R}$.}
\end{align*}
\end{proof}
\filbreak

\begin{problem}{3.ii}
\end{problem}
\begin{proof}
\begin{align*}
\intertext{Let $h(x) = f(x)\cos x - f'(x)\sin x$.}
h(0) &= f(0) \cos 0 - f'(0) \sin 0\\
&= 1 + 0\\
&= 1\\
h'(x) &= f'(x) \cos x - f(x) \sin x - f''(x) \sin x - f'(x) \cos x\\
h'(x) &= f'(x) \cos x - f(x) \sin x + f(x) \sin x - f'(x) \cos x \sin x \qquad \text{($\because f''(x) = -f(x)$)}\\
&= 0
\intertext{$\therefore h$ is a constant function and $h(x) = 1$, $\forall x \in \mathbb{R}$.}
\end{align*}
\end{proof}
\filbreak

\begin{problem}{3.iii}
\end{problem}
\begin{proof}
\begin{align*}
\shortintertext{From (ii),}
f(x)\cos x - f'(x)\sin x &= 1, \forall x \in \mathbb{R}\\
f(x) &= \frac{1 + f'(x)\sin x}{\cos x}\\
\shortintertext{From (i),}
f(x)\sin x + f'(x)\cos x &= 1, \forall x \in \mathbb{R}\\
f'(x) &= \frac{1-f(x)\sin x}{\cos x}\\
\shortintertext{Therefore,}
f(x) &= \frac{1 + \frac{1-f(x)\sin x}{\cos x}\sin x}{\cos x}\\
\cos^2 x f(x) &= \cos x + \sin x - \sin^2 x f(x)\\
\cos^2 x f(x) + \sin^2 x f(x) &= \cos x + \sin x \\
f(x) (\cos^2 x + \sin^2 x) &= \cos x + \sin x \\
f(x) &= \cos x + \sin x, \forall x \in \mathbb{R}\\
\end{align*}
\end{proof}
\filbreak

\begin{problem}{4.a}
\end{problem}
\begin{align*}
\shortintertext{Let radius of cylinder be $r$, height of cylinder be $h$, lateral surface area of cylinder be $A$.}
r &\in (0, 1)\\
h &\in (0, 2)\\
h &= 2\sqrt{1 - r^2}\\
A &= 4 \pi r \cdot \sqrt{1 - r^2}\\
\frac{dA}{dr} &= 4\pi \sqrt{1 - r^2} - \frac{4\pi r^2}{\sqrt{1 - r^2}}\\
\frac{dA}{dr} &= \frac{4\pi( 1 - r^2) - 4\pi r^2}{\sqrt{1 - r^2}}\\
\frac{dA}{dr} &= \frac{4\pi - 4\pi(r^2) - 4\pi r^2}{\sqrt{1 - r^2}}\\
\frac{dA}{dr} &= \frac{4\pi (1 - 2r^2)}{\sqrt{1 - r^2}}\\
\shortintertext{At a critical point,}
\frac{dA}{dr} &= 0\\ 
1 - 2r^2 &= 0\\
(1 - \sqrt{2}r) (1 + \sqrt{2}r) &= 0\\
r &= \frac{1}{\sqrt{2}} \text{ or } r = -\frac{1}{\sqrt{2}} \text{ (rejected $\because r \in (0,1)$)}\\
A \big|_{r=1/\sqrt{2}} &=  4 \pi \frac{1}{\sqrt{2}} \cdot \sqrt{1 - \left(\frac{1}{\sqrt{2}}\right)^2}\\
&= 2\pi\\
A \big|_{r=1/1.5} &\approx 6.2442 < 2\pi\\
A \big|_{r=1/1.4} &\approx 6.2818 < 2\pi\\
\intertext{$\therefore$ at $r = \frac{1}{\sqrt{2}}$, $A$ is maximum.}
h \big|_{r=1/\sqrt{2}} &= 2\sqrt{1-\left(\frac{1}{\sqrt{2}}\right)^2}\\
&= 1
\intertext{$\therefore$ When $A$ is maximum, $r = \frac{1}{\sqrt{2}}$, $h = 1$.}
\end{align*}
\filbreak

\begin{problem}{4.b}
\end{problem}
\begin{align*}
\shortintertext{Let radius of cone be $r$, height of cone be $h$, lateral surface area of cone be $A$.}
r &\in (0, 1)\\
h &\in (0, 	2)\\
r &= \sqrt{1 - (h-1)^2}\\
&= \sqrt{2h - h^2}\\
A &= \pi r l\\
&= \pi \sqrt{2h - h^2} \cdot \sqrt{2h - h^2 + h^2}\\
&= \pi \sqrt{2h - h^2}  \cdot \sqrt{2h}\\
&= \pi \sqrt{4h^2 - 2h^3}\\
\frac{dA}{dh} &= \pi \frac{8h - 6h^2} {2\sqrt{4h^2 - 2h^3}}\\
&= \pi \frac{h(4 - 3h)} {h\sqrt{4 - 2h}}\\
&= \pi \frac{4 - 3h} {\sqrt{4 - 2h}}\\
\shortintertext{At a critical point,}
\frac{dA}{dh} &= 0\\ 
4 - 3h &= 0\\
h = \frac{4}{3}\\
A \big|_{h=4/3} &= \pi \sqrt{4 \left(\frac{4}{3}\right)^2 - 2 \left(\frac{4}{3}\right)^3}\\
&= \pi \sqrt{\frac{64}{27}} \\
&= \frac{8}{3\sqrt{3}} \pi \approx 4.8367\\
A \big|_{r=1} &\approx 4.4428 < \frac{8}{3\sqrt{3}} \pi \\
A \big|_{r=5/3} &\approx 4.2751 < \frac{8}{3\sqrt{3}} \pi \\
\intertext{$\therefore$ at $h = \frac{3}{2} $, $A$ is maximum.}
r \big|_{h=4/3} &= \sqrt{\frac{8}{3} - \left(\frac{4}{3}\right)^2}\\
&= \frac{2\sqrt{2}}{3}
\intertext{$\therefore$ When $A$ is maximum, $r = \frac{2\sqrt{2}}{3}$, $h = \frac{4}{3}$.}
\end{align*}
\filbreak

\begin{problem}{5.i}
\end{problem}
\begin{align*}
\shortintertext{Let area of $\triangle OMN$ be $A$.}
y &= x^2 - 1\\
\frac{dy}{dx} &= 2x\\
\frac{dy}{dx}\Bigl|_{x=a} &= 2a\\
\text{Equation of tangent at $a$} : y-(a^2-1) &= 2a(x-a)\\
y &= 2ax-2a^2 + a^2 - 1\\
y &= 2ax - a^2 - 1\\
\text{$y$-intercept} &= -a^2 -1\\
\text{$x$-intercept} &= \frac{a}{2} + \frac{1}{2a}\\
A &= \frac{1}{2}|-a^2 -1|(\frac{a}{2} + \frac{1}{2a})\\
&= \frac{1}{2}(a^2 + 1)(\frac{a}{2} + \frac{1}{2a})\\
&= \frac{1}{2}(\frac{a^3}{2} + \frac{1}{2a} + \frac{a}{2} + \frac{a}{2})\\
&= \frac{a^3}{4} + \frac{a}{2} + \frac{1}{4a}\\
\end{align*}
\filbreak

\begin{problem}{5.ii}
\end{problem}
\begin{align*}
\frac{dA}{da} &= \frac{3a^2}{4} + \frac{1}{2} - \frac{1}{4a^2}\\
\intertext{At a critical point,}
\frac{dA}{da} &= 0\\
\frac{3a^2}{4} + \frac{1}{2} - \frac{1}{4a^2} &= 0\\ 
3a^4 + 2a^2 - 1 &= 0\\ 
a &= \frac{1}{\sqrt{3}} \text{ or } a = -\frac{1}{\sqrt{3}} \text{ (rejected $\because a > 0$)}\\
A \bigl|_{a=1/\sqrt{3}} &= \frac{1}{12\sqrt{3}} + \frac{1}{2\sqrt{3}} + \frac{\sqrt{3}}{4}\\
&= \frac{4}{3\sqrt{3}} \approx 0.7698\\
A \bigl|_{a=1/1.7} &= 0.7700\\
A \bigl|_{a=1/1.8} &= 0.7706\\
\intertext{$\therefore A$ is minimum when $a = \frac{1}{\sqrt{3}}$.}
P &= \left(\frac{1}{\sqrt{3}}, -\frac{2}{3}\right)
\end{align*}
\filbreak

\begin{problem}{6}
\end{problem}
\begin{align*}
\shortintertext{Let number of passengers per day be $n$, fare in dollars be $p$, revenue be $r$.}
\frac{dn}{dp} &= \frac{-200}{0.10} \qquad \text{(assuming linearity)}\\
&= -2000\\
n &= \int -2000 \diff p \\
&= -2000p + C\\
n_0 &= -2000p_0 + C\\
6000 &= -2000(2.00) + C\\
C &= 10000\\
\therefore n &= 10000 - 2000p\\
r &= np\\
&= 10000p - 2000p^2 \\
\frac{dr}{dp} &= 10000 -4000p\\
\intertext{At a critical point,}
\frac{dr}{dp} &= 0\\
10000 -4000p &= 0\\
p &= 2.50\\
\frac{d^2r}{dp^2}\biggl|_{p=2.50} &= -4000 < 0 \Rightarrow \text{$r$ is maximum at $p =2.50$}
\intertext{$\therefore$ revenue is maximized when fare is \$2.50.}
\end{align*}
\filbreak

\begin{problem}{7}
\end{problem}
\begin{align*}
\int_{0}^{1} \cos2x \diff x &= \lim_{n \to \infty} \frac{1}{n} \sum_{i = 1}^n \cos\frac{2i}{n} \\
&= \lim_{n \to \infty} \frac{1}{n} \left( \cos\frac{2}{n} + \cos\frac{4}{n} + \ldots + \cos\frac{2n}{n} \right) \\
&= \lim_{n \to \infty} \frac{1}{n} \left( \frac{\sin \left[ (n + \frac{1}{2})\frac{2}{n} \right]}{2\sin \frac{1}{n}} -\frac{1}{2} \right)\\
&= \lim_{n \to \infty} \frac{\sin \left(2 + \frac{1}{n}\right)}{2n\sin \frac{1}{n}} - \frac{1}{2n}\\
&= \lim_{n \to \infty} \frac{\sin \left(2 + \frac{1}{n}\right) - \sin\frac{1}{n}}{2n\sin \frac{1}{n}}\\
&= \lim_{n \to \infty} \frac{\sin \left(2 + \frac{1}{n}\right) - \sin\frac{1}{n}}{2} \cdot \lim_{n \to \infty}\frac{1}{n\sin \frac{1}{n}}\\
&= \frac{\sin (2 + 0) - \sin0}{2} \cdot \lim_{n \to \infty}\frac{1}{n\sin \frac{1}{n}}\\
&= \frac{\sin (2)}{2} \cdot \lim_{k \to 0}\frac{k}{\sin k} \quad \text{(sub. $k= \frac{1}{n}$)}\\
&= \frac{\sin (2)}{2} \cdot 1\\
&= \frac{\sin (2)}{2} 
\end{align*}
\filbreak

\begin{problem}{8.a}
\end{problem}
\begin{align*}
&\lim_{n \to \infty} n \left( \frac{1}{n^2+1^2} + \frac{1}{n^2+2^2} + \ldots + \frac{1}{n^2+n^2} \right)\\
&= \lim_{n \to \infty} \frac{1}{n} \left( \frac{n^2}{n^2+1^2} + \frac{n^2}{n^2+2^2} + \ldots + \frac{n^2}{n^2+n^2} \right)\\
&= \lim_{n \to \infty} \frac{1}{n} \sum_{i=1}^{n} \frac{1}{1 + \left(\frac{i}{n}\right)^2}\\
&= \int_{0}^{1} \frac{1}{1 + x^2} \diff x\\
&= \tan^{-1}x \bigl|_{0}^{1}\\
&= \frac{\pi}{4}
\end{align*}
\filbreak

\begin{problem}{8.b}
\end{problem}
\begin{align*}
&\lim_{n \to \infty} n \left( \frac{1}{(n+1)^2} + \frac{1}{(n+2)^2} + \ldots + \frac{1}{(n+n)^2} \right)\\
&= \lim_{n \to \infty} \frac{1}{n} \left( \frac{n^2}{(n+1)^2} + \frac{n^2}{(n+2)^2} + \ldots + \frac{n^2}{(n+n)^2} \right)\\
&= \lim_{n \to \infty} \frac{1}{n} \sum_{i=1}^{n} \frac{1}{(\frac{i}{n} + 1)^2}\\
&= \int_{0}^{1} \frac{1}{(x+1)^2} \diff x\\
&= -\frac{1}{x+1} \Bigl|_{0}^{1}\\
&= -\frac{1}{2} + 1\\
&= \frac{1}{2}
\end{align*}
\filbreak

\begin{problem}{9}
\end{problem}
\begin{align*}
F'(x) &= \frac{d}{dx} \int_{0}^{x^2} (x^2-t) f(t) \diff t\\
&= \frac{d}{dx} \left[ x^2 \int_{0}^{x^2} f(t) \diff t - \int_{0}^{x^2} t f(t) \diff t \right]\\
&=  x^2 \frac{du}{dx}\frac{d}{du} \int_{0}^{u} f(t) \diff t - \frac{du}{dx}\frac{d}{du} \int_{0}^{u} t f(t) \diff t \quad \text{(sub. $u = x^2$)}\\
&=  x^2 (2x)f(u) +2x \int_{0}^{x^2} f(t) \diff t- 2x u f(u) \\
&=  2x \int_{0}^{x^2} f(t) \diff t + 2x^3 f(x^2) - 2x^3 f(x^2) \\
&=  2x \int_{0}^{x^2} f(t) \diff t \\
F''(x) &= \frac{d}{dx} 2x \int_{0}^{x^2} f(t) \diff t \\
&= 2 \int_{0}^{x^2} f(t) \diff t + 2x (2x) f(x^2) \\
&= 2 \int_{0}^{x^2} f(t) \diff t +  4x^2 f(x^2) \\
\end{align*}
\filbreak

\begin{problem}{10.a}
\end{problem}
\begin{proof}
\begin{align*}
\intertext{Suppose $f(c) = k$, $k \in \mathbb{R}^+$, then by MVT,}
\exists a \in (0, c) &: f'(a) = k\\
\exists b \in (c, 1) &: f'(b) = -k\\
\intertext{$f'(x)$ is continuous and differentiable. By MVT,}
\exists d \in (a, b) &: f''(a) = -2k < 0\\
\intertext{$\therefore \exists x_0 = d \in (0,1)$ such that $f''(x) < 0$.}
\end{align*}
\end{proof}
\filbreak

\begin{problem}{10.b}
\end{problem}
\begin{proof}
\begin{align*}
\intertext{Let $g(t) = \int_{0}^{t} f(x) \diff x$, $t \in [0,1]$.}
g'(t) &= \frac{d}{\diff t} \int_{0}^{1} f(x) \diff x\\
&= f(t) \\
g(0) &= \int_{0}^{0} f(x) \diff x\\
&= 0\\
g(1) &= \int_{0}^{1} f(x) \diff x\\
&= 0\\
\intertext{$g$ is continuous and differentiable. By MVT,}
\exists a \in (0, 1) &: g'(a) = 0\\
f(a) &= 0
\intertext{By MVT,}
\exists b \in (0, a) &: f'(b) = 0\\
\exists c \in (a, 1) &: f'(c) = 0\\
\intertext{$f'(x)$ is continuous and differentiable. By MVT,}
\exists d \in (b, c) &: f''(d) = 0\\
\intertext{$\therefore \exists x_0 = d \in (0,1)$ such that $f''(x) = 0$.}
\end{align*}
\end{proof}
\filbreak

\end{document}