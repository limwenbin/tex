\documentclass[12pt]{article}
\usepackage[margin=1in]{geometry} 
\usepackage{amsmath,amsthm,amssymb,amsfonts}
 
\newcommand{\N}{\mathbb{N}}
\newcommand{\Z}{\mathbb{Z}}
 
\newenvironment{problem}[2][Problem]{\begin{trivlist}
\item[\hskip \labelsep {\bfseries #1}\hskip \labelsep {\bfseries #2.}]}{\end{trivlist}}
 
\begin{document}

\title{MA1102R Homework Assignment 1}
\author{Lim Wen Bin}
\maketitle

\begin{problem}{3.b}
\end{problem}
\begin{align*}
\intertext{To prove: $\lim_{x \to -1} \frac{1}{\sqrt{x^2 + 3}} = \frac{1}{2}$}
\text{i.e. } \forall \epsilon > 0, \exists \delta > 0 \text{ s.t. } |x - (-1)| < \delta \Rightarrow |\frac{1}{\sqrt{x^2 + 3}} - \frac{1}{2} | < \epsilon
\end{align*}

\begin{problem}{4}
\end{problem}
\begin{align*}
\text{Coordinates of P: } & (x, x^2)
\intertext{Let gradient of OP be $m_{OP}$.}
m_{OP} &= \frac{x^2}{x} \\
\intertext{Let gradient of perpendicular bisector of OP be $m_{Q}$.}
m_Q &= \frac{-1}{m_{OP}} \\
&= -\frac{1}{x} \\
\intertext{For arbitrary $x = a$,}
\text{Coordinates of midpoint of OP: } & (\frac{a}{2}, \frac{a^2}{2}) \\
\text{Equation of perpendicular bisector of OP: } & y - \frac{a^2}{2} = -\frac{1}{a} (x - \frac{a}{2}) \\
\text{: } & y = -\frac{1}{a} (x - \frac{a}{2}) + \frac{a^2}{2} \\
\text{Coordinates of Q: } & (0, -\frac{1}{a} (0 - \frac{a}{2}) + \frac{a^2}{2}) \\
\text{: } & (0, \frac{1}{2} + \frac{a^2}{2}) \\
\intertext{As P approaches O,}
\text{$y$-coordinate of Q} &= \lim_{a \to 0} (\frac{1}{2} + \frac{a^2}{2}) \\
&= \frac {1}{2} + 0 \\
&= \frac {1}{2}
\end{align*}

\begin{problem}{8}
\end{problem}
\begin{align*}
2x^2 + 2xy + xy^2 - 3x + 3y + 7 &= 0 \\
4x + 2y +2x\frac{dy}{dx} + y^2 +x(2y)\frac{dy}{dx} -3 + 3\frac{dy}{dx} &= 0 \\
\frac{dy}{dx} (2x + 2xy + 3) &= -4x -2y - y^2 + 3 \tag{1}\label{8.1} \\
\frac{dy}{dx} &= \frac{3 - y^2 - y - 4x}{2x + 2xy + 3}
\intertext{At $(1, -2)$,}
\frac{dy}{dx} & = \frac{3 - (-2)^2 - (-2) - 4}{2 + 2(-1) + 3} \\
& = -1 
\intertext{Differentiating \eqref{8.1} w.r.t. $x$,}
\frac{d^2y}{dx^2} (2x + 2xy + 3) + \frac{dy}{dx} (2 + 2y +2x \frac{dy}{dx}) &= -4 - 2 \frac{dy}{dx} - 2y \frac{dy}{dx} + 3 \\
\intertext{At $(1, -2)$,}
\frac{d^2y}{dx^2} (2 + 2(-2) + 3) - (2 + 2(-2) - 2) &= -4 + 2  + 2(-2) + 3 \\
\frac{d^2y}{dx^2} &= -7
\end{align*}

\begin{problem}{9}
\end{problem}
\begin{align*}
f(x) &= (x^2 - x - 5)x^{2/5} \\
f'(x) &= (2x - 1)x^{2/5} + \frac{2}{5} (x^2 - x - 5) x^{-3/5} \\
&= x^{-3/5} ((2x - 1)x + \frac{2}{5} (x^2 - x - 5)) \\
f(-1) = -3, & \qquad f(2) = (-3)(\sqrt[5]{4}) \approx -3.958
\intertext{At a critical point, $f'(x) = 0$, or $f'(x)$ does not exist.}
x^{-3/5} ((2x - 1)x + \frac{2}{5} (x^2 - x - 5)) &= 0 \\
(2x - 1)x + \frac{2}{5} (x^2 - x - 5) &= 0 \\
2x^2 - x + \frac{2}{5} x^2 - \frac{2}{5} x - 2 &= 0 \\
12x^2 -7x -10 &= 0 \\
x = 5/4, & \qquad x = -\frac{2}{3} \\
f(\frac{5}{4}) = (-\frac{76}{16} )( \sqrt[5]{\frac{25}{16}} ) \approx -5.125, & \qquad f(-\frac{2}{3}) = (-\frac{35}{9} )( \sqrt[5]{\frac{4}{9}} ) \approx -3.306 \\
f(0) = 0
\intertext{$f$ is maximum when $x=0$ where $f(0)=0$ and minimum at $x=\frac{5}{4}$ where $f(\frac{5}{4})=(-\frac{76}{16} )( \sqrt[5]{\frac{25}{16}} )$.}
\end{align*}

\begin{problem}{10}
\end{problem}
\begin{proof}
\begin{align*}
\intertext{To prove: $\lim_{x \to a} (f(x) + g(x)) = -\infty$}
\text{i.e. } \forall M < 0, \exists \delta > 0 &\qquad \text{s.t.} \qquad |x - a| < \delta \Rightarrow f(x) + g(x) < M \\
\lim_{x \to a} f(x) = -\infty \Rightarrow \forall K < 0, \exists \delta > 0 &\qquad \text{s.t.} \qquad |x - a| < \delta \Rightarrow f(x) < K \\
\lim_{x \to a} g(x) = c \Rightarrow \forall \epsilon > 0, \exists \delta > 0 &\qquad \text{s.t.} \qquad |x - a| < \delta \Rightarrow |g(x) - c| < \epsilon \\
\intertext{Choose $\epsilon = 1$,}
\exists \delta_g : |x - a| < \delta_g & \Rightarrow |g(x) - c| < 1 \\
-1 < g(x) + c &< 1 \\
-1 + c < g(x) &< 1 + c \\
\intertext{For any $M$, choose $K < M - c - 1$.}
\exists \delta_f : |x - a| < \delta_f & \Rightarrow f(x) < K \\
\intertext{Choose $\delta = \min \{ \delta_f, \delta_g \}$,}
|x - a| < \delta & \Rightarrow f(x) + g(x) < K + 1 + c \\
f(x) + g(x) &< M \\
\therefore \lim_{x \to a} (f(x) + g(x)) &= -\infty
\end{align*}
\end{proof}

\begin{problem}{11}
\end{problem}
\begin{proof}
\begin{align*}
y &= x^3 + px +q \\
\frac{dy}{dx} &= 3x^2 + p \\
\intertext{$y$ is tangent to x-axis $\Rightarrow$ when $x = x_0$, $y = 0$ and $\frac{dy}{dx} = 0$.}
0 &= 3x_0^2 + p \\
x_0 &= \sqrt{-\frac{p}{3}}, \qquad (p<0)\\
0 &= x_0^3 + px_0 +q \\
0 &= x_0(x_0^2 + p) +q \\
0 &= \sqrt{-\frac{p}{3}}(-\frac{p}{3} + p) +q \\
0 &= (\sqrt{-\frac{p}{3}}(-\frac{p}{3} + p) +q)(\sqrt{-\frac{p}{3}}(-\frac{p}{3} - p) +q) \\
0 &= -\frac{p}{3}(-\frac{p}{3} + p)^2 - q^2 \\
p(\frac{2p}{3})^2 + 3q^2 &= 0 \\
p(\frac{4p^2}{9}) + 3q^2 &= 0 \\
4p^3 + 27q^2 &= 0 \\
\end{align*}
\end{proof}

\begin{problem}{12}
\end{problem}
\begin{proof}
\begin{align*}
\intertext{Assume, for contradiction, that $f$ is unbounded above i.e. $\exists \alpha \in [a,b] \Rightarrow f(\alpha) = \infty$ is undefined. However,}
\text{continuity of f} &\Rightarrow \text{$f$ is defined for all $x \in [a,b]$}  
\intertext{This is a contradiction and hence $f$ is bounded above by $d$. In addition, assume, for contradiction, that $d$ is not in the range of $f$,}
f(x) &< d, \forall x \in [a,b] \\
\forall \sigma \in \mathbb{R}^+, \exists x \in [a,b] : d - f(x) &< \sigma,  \\
\frac{1}{d - f(x)} &> \frac{1}{\sigma}
\intertext{This implies that $g(x) = \frac{1}{d-f(x)}$ is unbounded above. However g(x) is continuous and as previously proved, bounded. A contradiction is obtained and thus $d$ is in range of f.}
\intertext{Similarly, assume $f$ is unbounded below i.e. $\exists \beta \in [a,b] \Rightarrow f(\beta) = -\infty$ is undefined. However,}
\text{continuity of f} &\Rightarrow \text{$f$ is defined for all $x \in [a,b]$}  
\intertext{This is a contradiction and hence $f$ is bounded below by $c$ which is similarly contained in the range of $f$ by considering $h(x) = \frac{1}{f(x) -c} > \frac{1}{\gamma}, \forall \gamma \in \mathbb{R}^+$ which yields a contradiction. \newline $\therefore$ Range of $f$ is $[c,d]$.}
\end{align*}
\end{proof}

\end{document}