\documentclass[12pt]{article}
\usepackage[margin=1in]{geometry} 
\usepackage{mathtools,amsmath,amsthm,amssymb,amsfonts}

\newcommand*\diff{\mathop{}\!\mathrm{d}}
\newcommand*\Diff[1]{\mathop{}\!\mathrm{d^#1}} 
\newcommand{\N}{\mathbb{N}}
\newcommand{\Z}{\mathbb{Z}}
 
\newenvironment{problem}[2][Problem]{\begin{trivlist}
\item[\hskip \labelsep {\bfseries #1}\hskip \labelsep {\bfseries #2.}]}{\end{trivlist}}
 
\begin{document}

\title{MA1104 Multivariable Calculus Tutorial 1}
\author{Lim Wen Bin \\
A0140764H\\
T01}
\maketitle

\begin{problem}{1}
\end{problem}
\begin{align*}
a &= \left( \begin{array}{c}
	3\\
	-2\\
	1\\
\end{array} \right) \\
b &= \left( \begin{array}{c}
	2\\
	3\\
	-4\\
\end{array} \right) \\
\text{comp}_a b &= \frac{a \cdot b}{||a||} \\
&= -\frac{4}{\sqrt{14}} \\
\text{proj}_a b &= \text{proj}_a b \cdot \frac{a}{||a||} \\
&= -\frac{4}{\sqrt{14}} 
\cdot 
\frac{1}{\sqrt{14}}
\left( \begin{array}{c}
	3\\
	-2\\
	1\\
\end{array} \right) \\
&=
\left( \begin{array}{c}
	-1/2\\
	6/7\\
	-4/7\\
\end{array} \right) \\
\theta &= \cos^{-1} \frac{a \cdot b}{||a||||b||}\\
&= 101.5 ^\circ
\end{align*}
\filbreak

\begin{problem}{2}
\end{problem}
\begin{align*}
x &= \text{proj}_a b \\
&= \frac{a \cdot b}{a \cdot a} a \\
&= \frac{9}{6} a \\
&= 
\left( \begin{array}{c}
	3\\
	-3/2\\
	3/2\\
\end{array} \right) \\
x + y &= b\\
y &= b - x \\
&= 
\left( \begin{array}{c}
	3\\
	-1\\
	2\\
\end{array} \right) 
-
\left( \begin{array}{c}
	3\\
	-3/2\\
	3/2\\
\end{array} \right) \\
&=
\left( \begin{array}{c}
	0\\
	1/2\\
	1/2\\
\end{array} \right) \\
\end{align*}
\filbreak

\begin{problem}{3}
\end{problem}
\begin{proof}
\begin{align*}
\text{comp}_c (a+b) &= \frac{(a+b) \cdot c}{||c||} \\
&= \frac{a \cdot c  + b \cdot c}{||c||} \\
&= \frac{a \cdot c}{||c||}  + \frac{b \cdot c}{||c||} \\
&= \text{comp}_c a + \text{comp}_c b
\end{align*}
\end{proof}
\filbreak

\begin{problem}{4}
\end{problem}
\begin{align*}
\vec{AD} &= \vec{OD} - \vec{OA} \\
&= 
\left( \begin{array}{c}
	5\\
	0\\
	6\\
\end{array} \right)
-
\left( \begin{array}{c}
	5\\
	2\\
	0\\
\end{array} \right)\\
&=
\left( \begin{array}{c}
	0\\
	-2\\
	6\\
\end{array} \right)\\
\vec{AB} &= \vec{OB} - \vec{OA} \\
&= 
\left( \begin{array}{c}
	2\\
	6\\
	1\\
\end{array} \right)
-
\left( \begin{array}{c}
	5\\
	2\\
	0\\
\end{array} \right)\\
&=
\left( \begin{array}{c}
	-3\\
	4\\
	1\\
\end{array} \right)\\
\vec{AB} \times \vec{AD}
&= 
\left( \begin{array}{c}
	-3\\
	4\\
	1\\
\end{array} \right) 
\times
\left( \begin{array}{c}
	0\\
	-2\\
	6\\
\end{array} \right)\\
&=
\left( \begin{array}{c}
	24 + 2\\
	0 + 18\\
	6\\
\end{array} \right)\\
&=
\left( \begin{array}{c}
	26\\
	18\\
	6\\
\end{array} \right)\\
\text{Area} &= 
\left\lVert
\left( \begin{array}{c}
	26\\
	18\\
	6\\
\end{array} \right)
\right\rVert \\
&= \sqrt{26^2 + 18^2 + 6^2} \\
&= 32.2 
\end{align*}
\filbreak

\begin{problem}{5}
\end{problem}
\begin{align*}
\vec{QR} &= \vec{OR} - \vec{OQ} \\
&= 
\left( \begin{array}{c}
	3\\
	1\\
	1\\
\end{array} \right)
-
\left( \begin{array}{c}
	0\\
	1\\
	2\\
\end{array} \right) \\
&=
\left( \begin{array}{c}
	3\\
	0\\
	-1\\
\end{array} \right) \\
\vec{PQ} &= \vec{OQ} - \vec{OP} \\
&= 
\left( \begin{array}{c}
	1\\
	2\\
	0\\
\end{array} \right)
-
\left( \begin{array}{c}
	0\\
	1\\
	2\\
\end{array} \right) \\
&=
\left( \begin{array}{c}
	1\\
	1\\
	-2\\
\end{array} \right) \\
\text{Distance} &= 
\left\lVert
\left( \begin{array}{c}
	1\\
	1\\
	-2\\
\end{array} \right)
\times
\left( \begin{array}{c}
	3\\
	0\\
	-1\\
\end{array} \right)
\right\rVert \\
&= 
\left\lVert
\left( \begin{array}{c}
	-1\\
	-5\\
	-3\\
\end{array} \right)
\right\rVert \\
&= \sqrt{1 + 25 + 9} \\
&= \sqrt{35} \\
\text{Distance} &= \frac{\sqrt{35}}{\sqrt{10}} \\
&= \sqrt{7/2}
\end{align*}
\filbreak

\begin{problem}{6}
\end{problem}
\begin{proof}
\begin{align*}
\intertext{Let $A, B, C, D$ be the vertices of the rhombus. Let $\vec{AB} = (x, y)^T$. Then,}
\vec{AC} &= \vec{AB} + \vec{BC} \\
&=
\left( \begin{array}{c}
	x\\
	y\\
\end{array} \right)
+
\left( \begin{array}{c}
	x\\
	-y\\
\end{array} \right) \\
&=
\left( \begin{array}{c}
	2x\\
	0\\
\end{array} \right) \\
\vec{BD} &= \vec{BA} + \vec{AD} \\
&=
\left( \begin{array}{c}
	-x\\
	-y\\
\end{array} \right)
+
\left( \begin{array}{c}
	x\\
	-y\\
\end{array} \right) \\
&=
\left( \begin{array}{c}
	0\\
	-2y\\
\end{array} \right) \\
\vec{AC} \cdot \vec{BD} &= 0\\
\intertext{$\therefore$ the diagonals of rhombu are $\bot$.}
\end{align*}
\end{proof}
\filbreak

\begin{problem}{7}
\end{problem}
\begin{align*}
(v + w) \cdot (v + w) &= \lVert v \rVert ^2 + \lVert w \rVert ^2 + 2 v \cdot w\\
&= 2\lVert v + w \rVert ^2 + 2 v \cdot w\\
&= 2 (v + w) \cdot (v + w) + 2 v \cdot w\\
\label{7}
v \cdot w &= -\frac{1}{2} (v + w) \cdot (v + w) \tag{1}
\intertext{Let $\theta$ be the angle between $v$ and $w$.}
\lVert v + w \rVert &= \lVert v \rVert + \lVert w \rVert \\
\cos\theta &= \frac{v \cdot w}{\lVert v \rVert \lVert w \rVert}\\
\cos\theta &= \frac{v \cdot w}{\lVert v + w \rVert ^2}\\
\cos\theta &= \frac{v \cdot w}{(v+w)\cdot(v+w)} \\
&= -\frac{1}{2} \qquad \text{(by \eqref{7})}\\\theta &= \frac{2\pi}{3}
\end{align*}
\filbreak

\begin{problem}{8}
\end{problem}
\begin{proof}
\begin{align*}
\text{Let } a &=
\left( \begin{array}{c}
	\sqrt{y + z}\\
	\sqrt{x + z}\\
	\sqrt{x + y}\\
\end{array} \right) \text{, where } (y + z), (x + z), (x + y) \ge 0, \\
\text{and }
b &= 
\left( \begin{array}{c}
	1\\
	1\\
	1\\
\end{array} \right) 
\shortintertext{Then, by triangle inequality,}
a \cdot b &\le \lVert a \rVert \lVert b \rVert \\
\sqrt{y + z} +\sqrt{x + z} + \sqrt{x + y} &\le \sqrt{2(x+y+z)}\sqrt{3} \\
\sqrt{y + z} +\sqrt{x + z} + \sqrt{x + y} &\le \sqrt{x+y+z}\sqrt{6} \\
\sqrt{\frac{y + z}{x+y+z}} + \sqrt{\frac{x + z}{x+y+z}} 
+ \sqrt{\frac{x + y}{x+y+z}} &\le \sqrt{6} \\
\end{align*}
\end{proof}
\filbreak

\begin{problem}{9}
\end{problem}
\begin{proof}
\begin{align*}
\lVert a \rVert &= \lVert (a-b)+b \rVert \\
&\le \lVert (a-b)\rVert +\lVert b \rVert \quad \text{(by triangle inequality)}\\
\lVert a \rVert - \lVert b \rVert &\le \lVert (a-b)\rVert \\
\end{align*}
\end{proof}
\filbreak

\begin{problem}{10}
\end{problem}
\begin{proof}
\begin{align*}
a \cdot b &= 1 - 1\\
&= 0 \Rightarrow a \bot b \\
\shortintertext{Furthermore,}
c &= 
\left( \begin{array}{c}
	0\\
	2\\
	2\\
\end{array} \right) \\
&=
\left( \begin{array}{c}
	-1\\
	1\\
	0\\
\end{array} \right) 
-
\left( \begin{array}{c}
	-1\\
	-1\\
	-2\\
\end{array} \right) \\
&= a - b
\intertext{$\therefore$ $c$ is the hypotenuse to right angled triangle with 
	sides $a$ and $b$.}
\end{align*}
\end{proof}
\filbreak

\begin{problem}{11.a}
\end{problem}
\begin{proof}
\begin{align*}
\vec{EQ} &= \alpha \vec{AQ} \qquad \text{(they are collinear)} \\
&= \alpha (\frac{1}{2} (2a - 4b)) \\
&= \alpha (a - 2b)
\end{align*}
\end{proof}
\filbreak

\begin{problem}{11.b}
\end{problem}
\begin{proof}
\begin{align*}
\vec{DE} &= \beta \vec{DB} \qquad \text{(they are collinear)} \\
&= \beta (2a + 2b)
\end{align*}
\end{proof}
\filbreak

\begin{problem}{11.c}
\end{problem}
\begin{proof}
\begin{align*}
\vec{DQ} &= \vec{DE} + \vec{EQ} \\
a &= \beta (2a + 2b) + \alpha (a - 2b) \\
\label{11.c}
2\beta + \alpha &= 1 \tag{1} \\
2\beta - 2\alpha &= 0 \tag{2}
\shortintertext{Solving (1) and (2), }
\alpha = \beta &= 1/3 \\
\therefore \vec{DE} &= \frac{1}{3}\vec{DB}
\shortintertext{Similiarly,}
\vec{PB} &= \vec{PF} + \vec{FB} \\
a &= \gamma ((-2a) + 2b) + \delta (2a + b) \\
-2\gamma + 2\delta &= 1 \tag{3} \\
2\gamma + \delta &= 0 \tag{4}
\shortintertext{Solving (3) and (4), }
\gamma = \delta &= 1/3 \\
\vec{PB} &= \frac{1}{3}\vec{DB} 
\intertext{$\therefore$ $AP$ and $AQ$ trisect diagonal $BD$ at points 
	E and F.}
\end{align*}
\end{proof}
\filbreak

\end{document}
