\documentclass[12pt]{article}
\usepackage[margin=1in]{geometry} 
\usepackage{mathtools,amsmath,amsthm,amssymb,amsfonts}

\AtBeginDocument{%
 \abovedisplayskip=4pt plus 5pt minus 3pt
 \abovedisplayshortskip=0pt plus 5pt
 \belowdisplayskip=4pt plus 5pt minus 3pt 
 \belowdisplayshortskip=4pt plus 5pt minus 4pt
}

\newcommand*\diff{\mathop{}\!\mathrm{d}}
\newcommand*\Diff[1]{\mathop{}\!\mathrm{d^#1}} 
\newcommand{\N}{\mathbb{N}}
\newcommand{\Z}{\mathbb{Z}}
 
\newenvironment{problem}[2][Problem]{\begin{trivlist}
\item[\hskip \labelsep {\bfseries #1}\hskip \labelsep {\bfseries #2.}]}{\end{trivlist}}
 
\begin{document}

\title{MA1104 Multivariable Calculus Tutorial 4}
\author{Lim Wen Bin \\
A0140764H\\
T01}
\maketitle

\begin{problem}{1.a}
\end{problem}
\begin{align*}
	f(x,y) &= 4x^3 -3x^2y^2 +2x +3y \\
	f_x &= 12x^2 -6y^2x +2\\
	f_y &= -6x^2y +3 \\
	f_{xy} &= -12xy\\
	f_{yx} &= -12xy \\
	f_{xx} &= 24x -6y^2\\
	f_{yy} &= -6x^2
\end{align*}
\filbreak

\begin{problem}{1.b}
\end{problem}
\begin{align*}
	f(x,y,u,v) &= \frac{ x^2yv^2 +(\sin xy)v } { u +x^3 +y^3u } \\
	f_{uvxyvuv} &= f_{vvvxyuu} \\
	f_v &= \frac{ 2x^2yv +(\sin xy) } { u +x^3 +y^3u } \\
	f_{vv} &= \frac{ 2x^2y } { u +x^3 +y^3u } \\
	f_{vvv} &= 0 \\
	f_{uvxyvuv} &= 0 
\end{align*}
\filbreak

\begin{problem}{2}
\end{problem}
\begin{align*}
	\intertext{Let $F(x,y,z) = 4x^2y -z$. The given surface is a level surface of where 
	$F = 0$.}
	\nabla F(x,y,z) &= \langle 8yx, 4x^2, -1 \rangle \\
	\nabla F(1,3,13) &= \langle 24, 4, -1 \rangle \\
	\intertext{Equation of tangent plane to given surface:}
	\nabla F(1,3,13) \cdot \langle x-1, y-3, z-12 \rangle &= 0\\
	x-1 + 3(y-3) -z-12 &= 0\\
	x+ 3(y-3) -z-13 &= 0\\
\end{align*}
\filbreak

\begin{problem}{3}
\end{problem}
\begin{align*}
	\intertext{On the plane $x=3$,}
	z &= \frac{9}{y^2 -3} \\
	9 &= y^2z - 3z \qquad \text{($y \ne \pm \sqrt{3}$)}
	\intertext{Let $f(y,z) = y^2z -3z$.}
	\nabla f(y,z) &= \langle 2zy, y^2 -3 \rangle \\
	\nabla f(2,9) &= \langle 36, 1 \rangle \\
	\intertext{Equation of tangent line to curve:}
	\langle x, y, z \rangle &= \langle 3, 2, 9 \rangle 
		+ t\langle 0,1,-36 \rangle, t \in \mathbb{R} \\
\end{align*}
\filbreak

\begin{problem}{4.a}
\end{problem}
\begin{align*}
	f_{x} (x,y) &= \sqrt{y} \\
	f_{y} (x,y) &= \frac{x}{2\sqrt{y}} \\
	\intertext{$f_x$ and $f_y$ are both continuous at $(1,4) \Rightarrow f$ 
		is differentiable.}
\end{align*}
\filbreak

\begin{problem}{4.b}
\end{problem}
\begin{gather*}
	\intertext{In general,}
	\begin{align*}
		g_x (x,y) &= \frac{\partial}{\partial x} \frac{x^3 + y^3} {2 + \cos(1/(x+y))} \\
		&=  \frac{3x^2(2+\cos(1/(x+y)) + [1/(x+y)^2]\sin[1/(x+y)](x^3 + y^3)} 
			{(2 + \cos(1/(x+y)))^2} 
	\end{align*}
	\intertext{$g_x$ and $g_y$ exist and are continuous on $\mathbb{R}^2 \setminus 
		\{(0,0)\}$ as they are rational functions.  It remains to prove that 
		$g_x$ and $g_y$ exist and are continuous at $(0,0)$.}
	\begin{align*}
		g_x (0,0) &= \lim_{h\to0} \frac{g(0+h,0) - g(0,0)} {h} &\\
		&= \lim_{h\to0} \frac{(0+h)^3}{h(2+\cos(1/(0+h)))} \\
		&= \lim_{h\to0} \frac{h^2}{2+\cos(1/(h))} \\
	\end{align*}
	\\
	\left| \frac{h^2}{2+\cos(1/(h))} \right| \le \left| h^2 \right| = h^2
	\intertext{By Squeeze Theorem,}
	g_x (0,0) = \lim_{h\to0} h^2 = 0
	\intertext{By the same argument, both $g_x$ and $g_y$ exist at $(0,0)$. To prove 
		their continuity at the said point, consider $\lim_{(x,y)\to(0,0)} g_x (x,y)$.}
	|g_x| \le |3x^2(2+\cos(1/(x+y)) + [1/(x+y)^2]\sin[1/(x+y)](x^3 + y^3)| \\
	|g_x| \le 9x^2 + \left| x^3 + y^3 \right| \\
	\lim_{(x,y)\to(0,0)} \left( 9x^2 + \left| x^3 + y^3 \right| \right) = 0
	\intertext{By Squeeze Theorem,}
	\lim_{(x,y)\to(0,0)} g_x (x,y) = 0 = g_x (0,0)
	\intertext{By the same argument, both $g_x$ and $g_y$ are continuous at $(0,0)$. 
		Therefore, $g$ is differentiable on \mathbb{R}^2.}
\end{gather*}
\filbreak

\begin{problem}{5}
\end{problem}
\begin{gather*}
	\intertext{$f$ is differentiable at $(a,b)$ implies that:}
	\lim_{(\Delta x, \Delta y)\to(0,0)} f(a + \Delta x, b + \Delta y) - f(a,b) = 0 \\
	\Rightarrow \lim_{(\Delta x, \Delta y)\to(0,0)} f(a + \Delta x, b + \Delta y) = f(a,b) 
	\intertext{Substituting $c$ for $a+\Delta x$ and $d$ for $b+\Delta y$,}
	\lim_{(c, d)\to(a,b)} f(c, d) = f(a,b) 
	\intertext{$\Rightarrow f$ is continuous at $(a,b)$.}
\end{gather*}
\filbreak

\begin{problem}{6}
\end{problem}
\begin{align*}
	\intertext{For $(x,y) \ne (0,0)$, $f_x$ and $f_y$ exist as $f$ is a rational 
		function. Checking whether the partial derivatives exist at $(0,0)$,}
	f_x (0,0) &= \lim_{h\to0} \frac{f(0+h,0) - f(0,0)} {h} \\
	&= \lim_{h\to0} \frac{f(0+h,0) - 0} {h} \\
	&= \lim_{h\to0} \frac{2(0+h)(0)} {h(0+h)^2} \\
	&= \lim_{h\to0} \frac{0} {h^2} \\
	&= 0 \\
	\shortintertext{Similarly,}
	f_y (0,0) &= 0 \\
	\intertext{Hence, both $f_x (0,0)$ and $f_y (0,0)$ exist. In general,}
	f_x (x,y) &= \frac{\partial}{\partial x} \left(
		\frac{2xy}{x^2 + y^2}
	\right) \\
	&= \frac{2y(x^2 + y^2) + 2xy(2x)}{(x^2 + y^2)^2} \\
	&= \frac{2yx^2 + 2y^3 + 4x^2y}{(x^2 + y^2)^2} \\
	\left| \frac{2yx^2 + 2y^3 + 4x^2y}{(x^2 + y^2)^2} \right| 
	&\le |2yx^2 + 2y^3 + 4x^2y|
	\intertext{By Squeeze Theorem,}
	\lim_{(x,y)\to(0,0)} f_x (x,y) &= \lim_{(x,y)\to(0,0)} |2yx^2 + 2y^3 + 4x^2y| \\
	&= 0 = f_x(0,0) 
	\intertext{By the same argument, both $f_x$ and $f_y$ are continuous at
		 $(0,0)$ and thus, $f$ is differentiable at $(0,0)$.}
\end{align*}
\filbreak

\begin{problem}{7}
\end{problem}
\begin{align*}
	\intertext{Let $f(x,y,z) = x^3y^4z^{-1}$.}
	f_x &= 3x^2y^4z^{-1} \\
	f_y &= 4x^3y^3z^{-1} \\
	f_z &= -x^3y^4z^{-2} \\
	f(4,1,2) &= 32 \\
	\Delta f &= f_x (4,1,2) \Delta x + f_y (4,1,2) \Delta y + f_z (4,1,2) \Delta z \\
	&= 24 (-0,01) + 128 (0.01) + 16 (0.02) \\
	&= 1.36 \\
	f(3.99^3, 1.01^4, 1.98^{-1}) &\approx 32 + 1.36 \\
	&= 33.36
\end{align*}
\filbreak

\end{document}
