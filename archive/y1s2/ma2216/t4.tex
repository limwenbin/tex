\documentclass[12pt]{article}
\usepackage[margin=1in]{geometry} 
\usepackage{mathtools,amsmath,amsthm,amssymb,amsfonts}

\newcommand*\diff{\mathop{}\!\mathrm{d}}
\newcommand*\Diff[1]{\mathop{}\!\mathrm{d^#1}} 
\newcommand{\N}{\mathbb{N}}
\newcommand{\Z}{\mathbb{Z}}
 
\newenvironment{problem}[2][Problem]{\begin{trivlist}
\item[\hskip \labelsep {\bfseries #1}\hskip \labelsep {\bfseries #2.}]}{\end{trivlist}}
 
\begin{document}

\title{ST2131 Probability Tutorial 3}
\author{Lim Wen Bin \\
A0140764H\\
T16}
\maketitle

\begin{problem}{1.i}
\end{problem}
\begin{align*}
	P(\text{2 day, original price}) &= \binom{2}{1} p(1-p) \\
	&= 2 p(1-p) \\
\end{align*}
\filbreak

\begin{problem}{1.ii}
\end{problem}
\begin{align*}
	P(\text{3 day, 1 unit increase}) &= \binom{3}{1} p^2(1-p)\\
	&= 3 p^2(1-p)\\
\end{align*}
\filbreak

\begin{problem}{1.iii}
\end{problem}
\begin{align*}
	P(\text{day 1 increase$|$3 day, 1 unit increase}) &= 
		\frac{P(\text{day 1 increase$\cap$3 day, 1 unit increase})}
		{P(\text{3 day, 1 unit increase})} \\
	&= \frac{p(\binom{2}{1}p(1-p)} {3 p^2(1-p)} \\
	&= \frac{2p^2(1-p)} {3 p^2(1-p)} \\
	&= \frac{2} {3}
\end{align*}
\filbreak

\begin{problem}{2}
\end{problem}
\begin{align*}
	P(AB) &= 1/4 \\
	&= (1/2) (1/2) \\
	&= P(A) P(B)
	\intertext{$\therefore A$ and $B$ are independent.}
	P(AC) &= 1/4 \\
	&= (1/2) (1/2) \\
	&= P(A) P(C)
	\intertext{$\therefore A$ and $C$ are independent.}
	P(BC) &= 1/4 \\
	&= (1/2) (1/2) \\
	&= P(B) P(C)
	\intertext{$\therefore B$ and $C$ are independent. However,}
	P(ABC) &= 1/4 \\
	P(A) P(B) P(C) &= 1/2 (1/2) (2/4) \\
	&= 1/8 \ne P(ABC)
	\intertext{$\therefore A, B$ and $C$ are not independent.}
\end{align*}
\filbreak

\begin{problem}{3}
\end{problem}
\begin{gather*}
	\intertext{$A$ is independent.}
	\Rightarrow P(AA) = P(A) P(A) \\
	P(A) = P(A)^2 \\
	P(A) = 1 \lor P(A) = 0
\end{gather*} 
\filbreak

\begin{problem}{4}
\end{problem}
\begin{proof}
\begin{gather*}
	\intertext{Let $A_i$ denote the probability that the first head occurs on the 
		$i$th toss of the coin. Then}
	\sum_{i=1}^{\infty} \left[ a_1 \prod_{j=1}^{i-1} (1-a_j) \right] 
		= \sum_{i=1}^{\infty} P(A_i) = P\left(\bigcup_{i=1}^{\infty}A_i\right) \\
	P\left(\bigcup_{i=1}^{\infty}A\right) 
		+ P\left( \left[ \bigcup_{i=1}^{\infty}A_i \right]^C \right) = 1 \\
	P\left( \left[ \bigcup_{i=1}^{\infty}A_i \right]^C \right) 
		= \prod_{i=1}^{\infty} (1-a_i) \\
	\therefore \sum_{i=1}^{\infty} \left[ a_1 \prod_{j=1}^{i-1} (1-a_j) \right] 
		+ \prod_{i=1}^{\infty} (1-a_i) = 1\\
\end{gather*}  
\end{proof}  
\filbreak

\begin{problem}{5.i}
\end{problem}
\begin{align*}
	P(AB) &= 1/4 \\
	&= (1/2) (1/2) \\
	&= P(A) P(B)
	\intertext{$\therefore A$ and $B$ are independent.}
	P(AC) &= 1/4 \\
	&= (1/2) (1/2) \\
	&= P(A) P(C)
	\intertext{$\therefore A$ and $C$ are independent.}
	P(BC) &= 1/4 \\
	&= (1/2) (1/2) \\
	&= P(B) P(C)
	\intertext{$\therefore B$ and $C$ are independent.}
\end{align*}  
\filbreak

\begin{problem}{5.ii}
\end{problem}
\begin{align*}
	P(C|AB) &= \frac{P(C\cap(AB))} {P(AB)} \\
	&= \frac{1/16}{1/4} \\
	&= 1/4
\end{align*}  
\filbreak

\begin{problem}{5.ii}
\end{problem}
\begin{align*}
	P(ABC) &= 1/4 \\
	P(A) P(B) P(C) &= 1/64 \ne P(ABC) \\
	\intertext{$\therefore$ The three events are not independent.}
\end{align*}  
\filbreak

\begin{problem}{6.i}
\end{problem}
\begin{gather*}
	\intertext{(i) is not true in general as having 2 pairs of independent events 
		does not imply the independece of the ensemble.}
	\text{$A$ is independent of $B$} \Rightarrow P(AB) = P(A) P(B) \\
	\text{$A$ is independent of $C$} \Rightarrow P(AC) = P(A) P(C) \\
	\begin{align*}
		P(A(B \cup C)) &= P(AB \cup AC) &\\
		&= P(AB) + P(BC) - P(ABC) \\
		&= P(A)P(B) + P(B)P(C) - P(A \cap BC) \\
		&= P(A)(P(B) + P(C)) - 0 \\
		&= P(A)(P(B) + P(C) - P(BC)) \\
		&= P(A)(P(B) \cup P(C))
	\end{align*}
	\intertext{(ii) is true.}
	\begin{align*}
	P(CAB) &= P(ABC) &\\
	&= P(A)P(BC) \\
	&= P(A)P(B)P(C) \\
	&= P(AB)P(C) \\
	\intertext{(iii) is true.}
	\end{align*}
\end{gather*}  
\filbreak

\begin{problem}{7}
\end{problem}
\begin{proof}
\begin{gather*}
	\intertext{Let $Q_n$ denote the probability that $n$ trials result in an odd number 
		of successes.}
	Q_n = 1 - P_n
	\intertext{Conditioning on the first outcome,}
	P_n = p(Q_{n-1}) + (1-p)P_{n-1} \\
	P_n = p(1 - P_{n-1}) + (1-p)P_{n-1} \\
	\intertext{Let $s_n$ denote the proposition that $P_n = f(n) = \frac{1+(1-2p)^n}{2}$. 
		When $n = 1$,}
	P_1 = 1-p \\
	f(n) = \frac{1+1-2p} {2} = 1-p
	\intertext{Therefore, $s_1$ is true. For some $k+1 \in \mathbb{N}$, assuming that 
		$s_k$ is true,}
	\begin{align*}
		P_{k+1} &= p(1-P_{k}) + (1-p)P_{k} &\\
		&= p\left(1-\frac{1+(1-2p)^k}{2} \right) + (1-p) \frac{1+(1-2p)^k}{2} \\
		&= \frac{p-p(1-2p)^k)}{2} + \frac{1-p+(1-p)(1-2p)^k}{2} \\
		&= \frac{1+(1-2p)(1-2p)^k)}{2} \\
		&= \frac{1+(1-2p)^{k+1}}{2} 
	\end{align*}
	\intertext{As $s_1$ is true and $s_k$ is true $\Rightarrow s_{k+1}$ is true, 
		$s_n$ is true for all $n \in \mathbb{N}$.}
\end{gather*}  
\end{proof}  
\filbreak

\begin{problem}{8}
\end{problem}
\begin{align*}
	S_X &= \{4, 2, 1, 0, -1, -2\} \\
	P(X=4) &= (4/14)(3/13) \\
	&= 6/91 \\
	P(X=2) &= (4/14)(2/13) + (2/14)(4/13) \\
	&= 8/91 \\ 
	P(X=1) &= (4/14)(8/13) + (8/14)(4/13) \\
	&= 32/91 \\ 
	P(X=0) &= (2/14)(1/13)\\
	&= 1/91 \\ 
	P(X=-1) &= (2/14)(8/13) + (8/14)(2/13)\\
	&= 16/91 \\ 
	P(X=-2) &= (8/14)(7/13) \\
	&= 28/91
\end{align*}  
\filbreak

\begin{problem}{9}
\end{problem}
\begin{align*}
	S_X &= \{ -n, -n+2, \ldots, n-2, n\} \\
	P(x=k) &= \frac{\binom{n}{(1/2)(n-|k|)}}{2^n}
\end{align*}  
\filbreak

\begin{problem}{10}
\end{problem}
\begin{align*}
	P(X=4) &= 4!/120 \\
	&= 24/120 \\
	P(X=3) &= 3!/120 \\
	&= 6/120\\
	P(X=2) &= \binom{3}{2}(2)/120 \\
	&= 6/120\\
	P(X=1) &= \binom{4}{2}(2)/120 \\
	&= 24/120\\
	P(X=0) &= \binom{5}{2}(3!)/120 \\
	&= 60/120\\
\end{align*}  
\filbreak

\begin{problem}{11.a}
\end{problem}
\begin{align*}
	P(X=1) &= 0 \\
	P(X=2) &= 11/12 - 1/2 \\
	&= 5/12 \\
	P(X=3) &= 1/12 \\
\end{align*}  
\filbreak

\begin{problem}{11.b}
\end{problem}
\begin{align*}
	P(1/2 < X < 3/2) &= P(X < 3) - P(X \le 1/2) \\
	&= 11/12 - 1/8 \\
	&= 19/24 
\end{align*}  
\filbreak

\end{document}
