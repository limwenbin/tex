\documentclass[12pt]{article}
\usepackage[margin=1in]{geometry} 
\usepackage{mathtools,amsmath,amsthm,amssymb,amsfonts}

\newcommand*\diff{\mathop{}\!\mathrm{d}}
\newcommand*\Diff[1]{\mathop{}\!\mathrm{d^#1}} 
\newcommand{\N}{\mathbb{N}}
\newcommand{\Z}{\mathbb{Z}}

\AtBeginDocument{%
 \abovedisplayskip=4pt plus 5pt minus 3pt
 \abovedisplayshortskip=0pt plus 5pt
 \belowdisplayskip=4pt plus 5pt minus 3pt 
 \belowdisplayshortskip=4pt plus 5pt minus 4pt
}

\newenvironment{problem}[2][Problem]{\begin{trivlist}
\item[\hskip \labelsep {\bfseries #1}\hskip \labelsep {\bfseries #2.}]}{\end{trivlist}}
 
\begin{document}

\title{MA2101 Linear Algebra II Homework 5}
\author{Lim Wen Bin \\
A0140764H\\
T01}
\maketitle

\section*{Section 6.1}

\subsubsection*{Problem 5}
\begin{align*}
	\text{Let } 
		x = \left( \begin{array}{c}
			x_1 \\
			x_2 \\
		\end{array} \right),\ 
		y &= \left( \begin{array}{c}
			y_1 \\
			y_2 \\
		\end{array} \right),\ 
		z = \left( \begin{array}{c}
			z_1 \\
			z_2 \\
		\end{array} \right)\\
	\intertext{For $c \in C$,}
		\langle cx + z, y \rangle &= (cx+z) A y^* \\
		&= (cx) A y^* + z A y^* \\
		&= c(x A y^*) + z A y^* \\
		&=\langle cx, y \rangle + \langle z, y \rangle \\
		\intertext{As $A^* = A$,}
		\overline{\langle x, y \rangle} &= \overline{xAy^*} \\
		&= \overline{\overline{y}A^Tx^T} \\
		&= y\overline{A^Tx^T} \\
		&= yAx^* \\
		&= \overline{\langle y, x \rangle}
		\intertext{Lastly, for $x \ne 0$,}
		\langle x, x \rangle &= (x_1 - ix_2, ix_1 + 2x_2)
			\left( \begin{array}{c}
				x_1 \\
				x_2 \\
			\end{array} \right)\\
		&= x_1(x_1 - ix_2) + x_2(ix_1 + 2x_2) \ne 0
	\intertext{$\therefore \langle x, y \rangle = xAy^*$ is a inner product.}
\end{align*}
\begin{align*}
	\langle (1-i, 2+3i), (2+i, 3-2i) \rangle 
	&= (1-i-i(2+3i), i(1-i)+2(2+3i))
	\left( \begin{array}{c}
		2-i \\
		3+2i \\
	\end{array} \right)\\
	&= (1-i-2i+3, i+1+4+6i)
	\left( \begin{array}{c}
		2-i \\
		3+2i \\
	\end{array} \right)\\
	&= (4-3i, 5+7i)
	\left( \begin{array}{c}
		2-i \\
		3+2i \\
	\end{array} \right)\\
	&= 5-10i+1+31i\\
	&= 6+21i
\end{align*}
\filbreak

\subsubsection*{Problem 8(a)}
\begin{gather*}
	\intertext{For $x = (1,1) \ne 0$,}
	\langle x,x\rangle = 1-1 = 0
	\intertext{$\therefore \langle(a,b),(c,d)\rangle = ac-bd$ is not an inner product.}
\end{gather*}
\filbreak

\subsubsection*{Problem 9(a)}
\begin{proof}
\begin{gather*}
	\text{Let } \beta = \{z_1,\ldots,z_n\} \\
	\langle x, z_i \rangle = 0,\ \forall z_i \in \beta
	\intertext{For an inner product space, $\langle x,x \rangle = 0 \Leftrightarrow x = 0$.
		For some scalars $(a_i)$,}
	x = \sum_{i=1}^n a_i z_i \\ 
	\begin{align*}
		\langle x, x \rangle &= \langle x, \sum_{i=1}^n a_i z_i \rangle &\\
		&= \sum_{i=1}^n \overline{a_i} \langle x, z_i \rangle \\
		&= \sum_{i=1}^n \overline{a_i} \cdot 0 \\
		&= 0
	\end{align*}
	\\
	\Rightarrow x = 0 
\end{gather*}
\end{proof}
\filbreak

\subsubsection*{Problem 9(b)}
\begin{proof}
\begin{gather*}
	\langle x, z \rangle = \langle y, z \rangle \\
	\langle x, z \rangle - \langle y, z \rangle = 0\\
	\langle x-y, z \rangle = 0,\ \forall z \in \beta \\
	\intertext{By (a),}
	x-y=0 \Rightarrow x=y
\end{gather*}
\end{proof}
\filbreak

\subsubsection*{Problem 11}
\begin{proof}
\begin{align*}
	\lVert x+y \rVert^2 + \lVert x-y \rVert^2
	&= \langle x+y,x+y \rangle + \langle x-y,x-y \rangle \\
	&= [(\langle x,x \rangle + \langle y,x \rangle) + (\langle x,y \rangle + 
		\langle y,y \rangle)] \\
	&\phantom{=}+[(\langle x,x \rangle - \langle y,x \rangle) - (\langle x,y \rangle - 
		\langle y,y \rangle)] \\
	&= \langle x,x \rangle + \langle y,x \rangle + \langle x,y \rangle + 
		\langle y,y \rangle \\
	&\phantom{=}+\langle x,x \rangle - \langle y,x \rangle - \langle x,y \rangle + 
		\langle y,y \rangle \\
	&= 2\langle x,x \rangle + 2\langle y,y \rangle \\
	&= 2\lVert x \rVert^2 + 2\lVert y \rVert^2
	\intertext{The equation states that the sum of the squares of the lengths of both 
		diagonals of a parallelogram is equal to the sum of the squares of the lengths 
		of its sides.}
\end{align*}
\end{proof}
\filbreak

\subsubsection*{Problem 12}
\begin{proof}
\begin{gather*}
	S = \{v_1, \ldots, v_k\} \text{ is orthogonal} \\
	\Rightarrow \langle v_i, v_j \rangle = 0 \text{ for } i \ne j \\
	\begin{align*}
		\left\lVert \sum_{i=1}^{k} a_i v_i \right\rVert^2 
		&= \left\langle \sum_{i=1}^{k} a_i v_i, \sum_{j=1}^{k} a_j v_j \right\rangle &\\
		&= \sum_{i=1}^{k} a_i \left\langle v_i, \sum_{j=1}^{k} a_j v_j \right\rangle \\
		&= \sum_{i=1}^{k} \sum_{j=1}^{k} a_i \overline{a_j} 
			\left\langle v_i, v_j \right\rangle \\
		&= \sum_{i=1}^{k} a_i \overline{a_i} \left\langle v_i, v_i \right\rangle\ 
			(\because \langle v_i, v_j \rangle = 0 \text{ when } i \ne j) \\
		&= \sum_{i=1}^{k} |a_i|^2 \lVert v_i \rVert^2
	\end{align*}
\end{gather*}
\end{proof}
\filbreak

\section*{Section 6.2}

\subsubsection*{Problem 6}
\intertext{To prove: for $x \not \in W$, $\exists y \in V$ s.t. $y \in W^\bot$ 
	and $\langle x,y \rangle \ne 0$}
\begin{proof}
\begin{gather*}
	\intertext{By Theorem 6.6,}
	x = z + y,\ z \in W,\ y \in W^\bot \\
	y = x-z
	\intertext{Assume, for contradiction, that $\langle x, y\rangle = 0$. Then,}
	\langle z+y, y \rangle = 0 \\
	\langle z, y \rangle + \langle y, y \rangle = 0 \\
	0 + \lVert y \rVert^2 = 0 \\
	y = 0 
	\intertext{Then,}
	x = z \Rightarrow x \in W
	\intertext{Thus yielding a contradiction. Therefore, $\exists y \in W^\bot$ s.t. 
		$\langle x, y\rangle \ne 0$}
\end{gather*}
\end{proof}
\filbreak

\subsubsection*{Problem 7} 
\intertext{To prove: $z \in W^\bot
\Leftrightarrow \langle z,v \rangle = 0,\ \forall v \in \beta$}
\begin{proof} 
\begin{gather*} 
	\intertext{$(\Rightarrow):$}
	z \in W^\bot \Rightarrow \langle z, w \rangle = 0,\ \forall w \in W \\
	v \in \beta \Rightarrow v \in W \\
	\therefore \langle z, v \rangle = 0,\ \forall v \in \beta
	\intertext{$(\Leftarrow):$}
	\intertext{For any $x \in W$, for scalars $(a_i)$,}
	\text{Let } \beta = \{v_1, v_2, \ldots\} \\
	x = a_1 v_1 + a_2 v_2 + \ldots \\
	\begin{align*}
		\langle z, x \rangle &= \langle z, a_1 v_1 \rangle 
			+ \langle z, a_2 v_2 \rangle + \ldots &\\
		&= \overline{a_1} \langle z, v_1 \rangle + 
			\overline{a_2} \langle z, v_2 \rangle + \ldots \\
		&= 0
	\end{align*}
	\intertext{i.e.} \langle z, x \rangle = 0,\ \forall z \in W
	\Rightarrow x \in W^\bot
\end{gather*} 
\end{proof} 
\filbreak

\subsubsection*{Problem 13(a)} 
\begin{proof} 
\begin{gather*} 
	\intertext{For any $v \in S^\bot$,}	
	\langle v, s \rangle = 0,\ \forall s \in S \\
	\because S_0 \subseteq S, \\
	\langle v, s' \rangle = 0,\ \forall s' \in S_0 \subseteq S \\
	\Rightarrow v \in S_0^\bot \\
	\therefore S_0 \subseteq S \Rightarrow S^\bot \subseteq S_0^\bot
\end{gather*} 
\end{proof} 
\filbreak

\subsubsection*{Problem 13(b)} 
\begin{proof} 
\begin{gather*} 
	v \in S^\bot \Rightarrow \langle v, u \rangle = 0\ \forall u \in S \\
	w \in (S^\bot)^\bot \Rightarrow \langle w, v \rangle = 0\ \forall v \in S^\bot \\
	\intertext{i.e.}
	\forall u \in S,\ v \in S^\bot,\ \langle u, v \rangle = \overline{0} = 0 \\
	\therefore u \in S \Rightarrow u \in (S^\bot))^\bot \\
	\Rightarrow S \subseteq (S^\bot)^\bot
	\intertext{Then, for any $x \in \text{span}(S)$, $v \in S^\bot$, for scalars $(a_i)$,}
	\text{Let } S = \{ s_1, s_2, \ldots\} \\
	x = a_1 s_1 + a_1 s_2 + \ldots \\
	\begin{align*}
		\langle x, v \rangle &= \langle a_1 s_1, v \rangle 
			+ \langle a_2 s_2, v \rangle + \ldots &\\
		&= \overline{a_1} \langle s_1, v \rangle + 
			\overline{a_2} \langle s_2, v \rangle + \ldots \\
		&= 0
	\end{align*}
	\\
	\Rightarrow x \in (S^\bot))^\bot \\
	\therefore \text{span}(S) \subseteq (S^\bot))^\bot 
\end{gather*} 
\end{proof} 
\filbreak

\subsubsection*{Problem 13(c)} 
\begin{proof} 
\begin{gather*} 
	\intertext{Taking the contrapositive of the result in Exercise 6,}
	\forall v \in W^\bot,\ \langle u,v \rangle = 0 \Rightarrow u \in W \\
	\intertext{For any $w \in (W^\bot)^\bot$,}
	\langle w,v \rangle = 0,\ \forall v \in W^\bot \\
	\therefore w \in (W^\bot)^\bot \Rightarrow w \in W \\
	\Rightarrow (W^\bot)^\bot \subseteq W
	\intertext{From (b),}
	W \subseteq (W^\bot)^\bot \\
	\therefore W = (W^\bot)^\bot
\end{gather*} 
\end{proof} 
\filbreak

\subsubsection*{Problem 13(d)} 
\begin{proof} 
\begin{gather*} 
	V = W \oplus W^\bot \Leftrightarrow W \cap W^\bot = \{0\} \land V = W + W^\bot
	\intertext{$(W \cap W^\bot = \{0\}):$}
	\intertext{Suppose $x \in W \cap W^\bot$. For any $v \in V$, by Theorem 6.6,}
	v = w + w',\ w \in W,\ w' \in W^\bot \\
	\begin{align*}
		\langle x, v \rangle &= \langle x, w \rangle + \langle x, w' \rangle &\\
		&= 0 + 0 = 0
	\end{align*}
	\intertext{i.e.}
	\forall v \in V,\ \langle x, v \rangle = 0 = \langle 0, v \rangle \\
	\Rightarrow x = 0
	\intertext{$(V = W + W^\bot):$}
	\intertext{Trivially, $W + W^\bot \subseteq V$. To prove the converse, for any 
		$v \in V$, by Theorem 6.6,}
	v = w + w',\ w \in W,\ w' \in W^\bot \\
	\therefore V \subseteq W + W^\bot \\
	\Rightarrow V = W \oplus W^\bot
\end{gather*} 
\end{proof} 
\filbreak

\subsubsection*{Problem 19(a)} 
\begin{gather*} 
	\intertext{Let $v$ be the orthogonal projection of $u$ on $W$.}
	W = \text{span}(w) = \text{span}((1,4)) \\
	\begin{align*} 
		v &= \frac{\langle u, w \rangle}{\langle w, w \rangle}w &\\
		&= \frac{2 + 24}{17}(1,4) \\
		&= \frac{26}{17}(1,4) \\
	\end{align*} 
\end{gather*} 
\filbreak

\subsubsection*{Problem 19(b)} 
\begin{gather*} 
	\intertext{Let $v$ be the orthogonal projection of $u$ on $W$.}
	W = \text{span}(w_1, w_2) = \text{span}((-3,1,0), (2,0,1)) \\
	\intertext{To make $\{w_1, w_2\}$ into an orthogonal basis for $W$,}
	\text{Let } w_1' = w_1 \\
	\begin{align*} 
		w_2' &= w_2 - \frac{\langle w_2,w_1, \rangle}{\langle w_1, w_1 \rangle} w_1 &\\
		&= (2,0,1) - \frac{-6}{10} (-3,1,0) \\
		&= (1/5,3/5,1) \\
		v &= \frac{\langle u, w_1' \rangle}{\langle w_1', w_1' \rangle}w_1' +
			\frac{\langle u, w_2' \rangle}{\langle w_2', w_2' \rangle}w_2' \\
		&= \frac{-6 + 1}{10}(-3,1,0) + \frac{2/5 + 3/5 + 3}{7/5}(1/5,3/5,1) \\
		&= (3/2,-1/2,0) + (4/7,12/7,20/7) \\
		&= (29/14,17/14,20/7)
	\end{align*} 
\end{gather*} 
\filbreak

\subsubsection*{Problem 19(c)} 
\begin{gather*} 
	\intertext{Let $v$ be the orthogonal projection of $h$ on $W$.}
	W = \text{span}(w_1, w_2) = \text{span}((1,0,\ldots), (0,1,\ldots)) \\
	\intertext{To make $\{w_1, w_2\}$ into an orthogonal basis for $W$,}
	\text{Let } w_1' = w_1 \\
	\begin{align*} 
		w_2' &= w_2 - \frac{\langle w_2,w_1 \rangle}{\langle w_1, w_1 \rangle} w_1 &\\
		&= (0,1,\ldots) - \frac{1}{2}(1,0,\ldots) \\
		&= (-1/2,1,\ldots) \\
		\langle w_2', w_2' \rangle &= \int_0^1 1/4 -x +x^2 \diff x \\
		&= 1/4 -1/2 +1/3 \\
		&= 1/12\\
		v &= \frac{\langle u, w_1' \rangle}{\langle w_1', w_1' \rangle}w_1' +
			\frac{\langle u, w_2' \rangle}{\langle w_2', w_2' \rangle}w_2' \\
		&= \left( \int_0^1 4+3x-2x^2 \diff x \right)(1,0,\ldots) \\
		&\phantom{=} + 12\left( \int_0^1 (x-1/2)(4+3x-2x^2) \diff x \right) 
			(-1/2,1,\ldots) \\
		&= (4+3/2-2/3)(1,0,\ldots) \\
		&\phantom{=} + 12\left( \int_0^1 4x+3x^2-2x^3 -2-(3/2)x+x^2 \diff x \right) 
			(-1/2,1,\ldots) \\
		&= (29/6,0,\ldots) + 12(2 +1 -1/2 -2 -3/4 +1/3) (-1/2,1,\ldots) \\
		&= (29/6,0,\ldots) + (-1/2,1,\ldots) \\
		&= (13/3,1,\ldots) 
	\end{align*} 
\end{gather*} 
\filbreak

\section*{Section 6.3}

\subsubsection*{Problem 2(b)} 
\begin{align*} 
	\intertext{$\beta = \{(1,0),(0,1)\}$ is an orthonomal basis for $C^2$. Then,}
	y &= g((1,0)) \cdot (1,0) +  g((0,1)) \cdot (0,1) \\
	&= (1, -2)
\end{align*} 
\filbreak

\subsubsection*{Problem 8} 
\begin{proof} 
\begin{gather*} 
	\text{$T$ is invertible} \Rightarrow \exists T^{-1} \text{ s.t. } TT^{-1} = T^{-1}T = I
	\intertext{For any $x, y \in V$,}
	\langle T(x), y \rangle = \langle x, T^*(y) \rangle \\
	\langle T^{-1}(x), y \rangle = \langle x, (T^{-1})^*(y) \rangle \\
	\intertext{Considering the composition,}
	\langle T^{-1}(T(x)), y \rangle = \langle x, (T^{-1})^*(T^*(y)) \rngle \\
	\langle x, y \rangle = \langle x, (T^{-1})^*(T^*(y)) \rngle \\
	\therefore (T^{-1})^*T^* = I
	\intertext{A similiar argument taking the composition in the other order shows that 
		$T^* (T^{-1})^*= I$. Therefore, $T^*$ is invertible. Furthermore,}
	T^* (T^{-1})^* = (T^{-1})^*T^* = I \\
	\Rightarrow (T^*)^{-1} = (T^{-1})^* 
\end{gather*}
\end{proof} 
\filbreak

\subsubsection*{Problem 12(a)} 
\begin{proof} 
\begin{gather*} 
	\intertext{For any $u \in N(T)$,}
	T(u) = 0 \\
	\forall y \in V, \langle T(u), y \rangle = \langle u, T^*(y) \rangle \\
	\langle 0, y \rangle = \langle u, T^*(y) \rangle \\
	\langle u, T^*(y) \rangle = 0 \\
	\Rightarrow u \in R(T^*)^\bot \\
	\therefore N(T) \subseteq R(T^*)^\bot \\
	\intertext{For any $v \in R(T^*)^\bot$,}
	\forall x\in V, \langle v, T^*(x) \rangle = 0 \\
	\langle T(v), x \rangle = \langle v, T^*(x) \rangle = 0 \\
	\intertext{i.e.}
	\langle T(v), x \rangle = 0, \forall x \in V \\
	\Rightarrow T(v) = 0 \text{ and } v \in N(T) \\
	\therefore R(T^*)^\bot \subseteq N(T) \Rightarrow R(T^*)^\bot = N(T)
\end{gather*}
\end{proof} 
\filbreak

\subsubsection*{Problem 12(b)} 
\begin{proof} 
\begin{align*} 
	\intertext{In general, from (a),}
	R(T^*)^\bot &= N(T)
	\intertext{When $V$ is finite dimensional, using Exercise 13(c) of section 6.2}
	R(T^*) &= (R(T^*)^\bot)^\bot \\
	&= N(T)^\bot \\
\end{align*}
\end{proof} 
\filbreak

\end{document}
