\documentclass[10pt,landscape]{article}
\usepackage{multicol}
\usepackage{calc}
\usepackage{ifthen}
\usepackage[landscape]{geometry}
\usepackage{mathtools,amsmath,amsthm,amsfonts,amssymb}
\usepackage{color,graphicx,overpic}
\usepackage{hyperref}


\pdfinfo{
  /Title (example.pdf)
  /Creator (TeX)
  /Producer (pdfTeX 1.40.0)
  /Author (Seamus)
  /Subject (Example)
  /Keywords (pdflatex, latex,pdftex,tex)}

% This sets page margins to .5 inch if using letter paper, and to 1cm
% if using A4 paper. (This probably isn't strictly necessary.)
% If using another size paper, use default 1cm margins.
\ifthenelse{\lengthtest { \paperwidth = 11in}}
    { \geometry{top=.5in,left=.5in,right=.5in,bottom=.5in} }
    {\ifthenelse{ \lengthtest{ \paperwidth = 297mm}}
        {\geometry{top=1cm,left=1cm,right=1cm,bottom=1cm} }
        {\geometry{top=1cm,left=1cm,right=1cm,bottom=1cm} }
    }

% Turn off header and footer
\pagestyle{empty}

% Redefine section commands to use less space
\makeatletter
\renewcommand{\section}{\@startsection{section}{1}{0mm}%
                                {-1ex plus -.5ex minus -.2ex}%
                                {0.5ex plus .2ex}%x
                                {\normalfont\large\bfseries}}
\renewcommand{\subsection}{\@startsection{subsection}{2}{0mm}%
                                {-1explus -.5ex minus -.2ex}%
                                {0.5ex plus .2ex}%
                                {\normalfont\normalsize\bfseries}}
\renewcommand{\subsubsection}{\@startsection{subsubsection}{3}{0mm}%
                                {-1ex plus -.5ex minus -.2ex}%
                                {1ex plus .2ex}%
                                {\normalfont\small\bfseries}}
\makeatother

% Define BibTeX command
\def\BibTeX{{\rm B\kern-.05em{\sc i\kern-.025em b}\kern-.08em
    T\kern-.1667em\lower.7ex\hbox{E}\kern-.125emX}}

% Don't print section numbers
\setcounter{secnumdepth}{0}


\setlength{\parindent}{0pt}
\setlength{\parskip}{0pt plus 0.5ex}

%My Environments
\newtheorem{example}[section]{Example}
% -----------------------------------------------------------------------

\begin{document}
\raggedright
\footnotesize
\begin{multicols}{3}


% multicol parameters
% These lengths are set only within the two main columns
%\setlength{\columnseprule}{0.25pt}
\setlength{\premulticols}{1pt}
\setlength{\postmulticols}{1pt}
\setlength{\multicolsep}{1pt}
\setlength{\columnsep}{2pt}

\begin{center}
     \Large{\underline{MA3269 Mathematical Finance I}} \\
\end{center}

Lim Wen Bin A0140764H

\subsection{Chapter 0: Revision}

\subsubsection{Series}
$\sum_{i=0}^n y^i = \frac{1-y^{n+1}} {1-y}$;
$\sum_{i=0}^\infty y^i = \frac{1} {1-y}$ for $|y|<1$;
$\sum_{i=1}^n iy^i = \frac{1-y^n (1+-ny)} {(1-y)^2}$;
$\sum_{i=1}^\infty iy^i = \frac{1} {(1-y)^2}$ for $|y|<1$;
$\sum_{i=1}^n i = \frac{1}{2} n(n+1)$;
$\sum_{i=1}^n i^2 = \frac{1}{6} n(n+1)(2n+1)$;
$\sum_{i=1}^n i^3 = \frac{1}{4} n^2(n+1)^2$;
$\sum_{i=m}^n u_i = \sum_{i=k}^n u_i = \sum_{i=k}^{m-1} u_i$;

\subsubsection{Calculus}
AM-GM Inequality: $\frac{x+y}{2} \ge \sqrt{xy} \Leftrightarrow \frac{\sqrt{xy}}{x+y} \le \frac{1}{2}$;
$f$ continuous on $[a,b] \land f(a)f(b)<0 \Rightarrow \exists x \in (a,b) : f(x)=0$.
For injective and differentiable $f$ on open interval $I$, $f^{-1}$ is
differentiable on its domain and $(f^{-1}(y))' = \frac{1} {f'(f^{-1}(y))}$;
L'Hopital's Rule: for $\frac{0}{0}$ or $\frac{\infty}{\infty}$, for $c=\infty$ or $c=a^+/a^-$
$\lim_{x\to c} \frac{f(x)}{g(x)} = \lim_{x\to c} \frac{f'(x)}{g'(x)}$;
Taylor's Theorem: $f(b) = \sum_{k=0}^{n} \frac{f^{(k)}(a)}{k!} (b-a)^k +
\frac{f^{(n+1)}(c)}{(n+1)!} (b-a)^{n+1}$ for some $c \in (a,x)$;
Mean Value Theorem: for continuous $f$ on $[a,b]$, $\exists c \in (a,b) : 
f'(c) = \frac{f(b)-f(a)}{b-a}$;
$f$ is increasing if $a<b \Rightarrow f(a)<f(b)$ or $f'(x)>0\ \forall x \in (a,b)$;
$f$ is convex at $c$ if it is differentiable and its graph lies below the tangent line;
Jensen's Inequalities: for convex(concave) $f$ over $(a,b)$, $(x_n) \in (a,b)$
and $(\alpha_n) \in [0,1] : \sum_i \alpha_i = 1$, 
$f\left( \sum_{i=1}^n \alpha_i x_i \right) \le (\ge)\ f \sum_{i=1}^n \alpha_i f(x_i)$;

\subsubsection{Derivatives}
$f'(x) = \lim_{h\to0} \frac{f(x+h)-f(x)} {h}$;
$\frac{d}{dx} \tan(f(x)) = f'(x)\sec^2(f(x))$;
$\frac{d}{dx} \csc(f(x)) = f'(x)\csc(f(x))\cot(f(x))$;
$\frac{d}{dx} \sec(f(x)) = f'(x)\sec(f(x))\tan(f(x))$;
$\frac{d}{dx} \sin^{-1}(f(x)) = \frac{f'(x)} {\sqrt{1-(f(x))^2}}$;
$\frac{d}{dx} \cos^{-1}(f(x)) = -\frac{f'(x)} {\sqrt{1-(f(x))^2}}$;
$\frac{d}{dx} \tan^{-1}(f(x)) = \frac{f'(x)} {1+(f(x))^2}$;
$\frac{d}{dx} (f(x))^{g(x)} = (f(x))^{g(x)} \frac{d}{dx} (g(x) \ln f(x))$;
Quotient Rule: $\frac{d}{dx} \frac{f(x)}{g(x)} = \frac{f'(x)g(x)-g'(x)f(x)}{(g(x))^2}$;

\subsubsection{Integrals}
$\int \frac{1} {a^2+(x+b)^2} dx = \frac{1}{a} \tan^{-1} \left( \frac{x+b}{a} \right)$;
$\int \frac{1} {\sqrt{a^2+(x+b)^2}} dx = \sin^{-1} \left( \frac{x+b}{a} \right)$;
$\int -\frac{1} {\sqrt{a^2+(x+b)^2}} dx = \cos^{-1} \left( \frac{x+b}{a} \right)$;
$\int \frac{1} {a^2-(x+b)^2} dx = \frac{1}{2a} \ln \left| \frac{x+b+a}{x+b-a} \right|$;
$\int \frac{1} {(x+b)^2-a^2} dx = \frac{1}{2a} \ln \left| \frac{x+b-a}{x+b+a} \right|$; \\
Substitution I: $u=g(x), \int f(g(x))g'(x) dx = \int f(u)du$;
Substitution II: $x=g(t), \int f(x)dx = \int f(g(t))g'(t)dt$;
Integration by parts: $\int fg' dx = fg - \int f'g dx$;
FTC: $\int_{a}^{b} F'(x) dx = F(b) - F(a)$;

\subsection{Chapter 1: Theory of Interest}

\subsubsection{Interest}
Simple interest: $a(t) = 1 + rt$;
Compound interest: $a(t) = (1 + r)^t$;
Frequency compounding: $a(t) = \left(1 + \frac{r^{(p)}}{p} \right)^p$;
Effective annual interest: $1+r_e = \left(1 + \frac{r^{(p)}}{p} \right)^p$;
Continuous compounding: $a(t) = \lim_{p\to\infty} \left(1 +
	\frac{r^{(\infty)}}{p} \right)^p = e^{r^{(\infty)}}$;
$\forall p \in \mathbb{Z}^+, r>0, e^r > \left(1 + \frac{r}{p} \right)^p$;

\subsubsection{Present Value and Time Value}
$\text{PV}(C) = \sum_{i=1}^n \frac{c_i}{a(t_i)}$;
$\text{TV}_t(C) = \text{PV}(C) \cdot a(t)$;
$\text{TV}(C,t) = \frac{a(t)}{a(s)} \cdot \text{TV}(C,s)$;
$C_1$ and $C_2$ are equivalent if $PV(C_1) = PV(C_2)$;
Deferred Cash Flow: $\frac{\text{PV}(C)}{\text{PV}(C_{(+k)})} = (1+r)^k$;
Equation of Value: $\text{PV}(C) = \sum_{i=1}^n \frac{c_i}{(1+r)^{t_i}} = 0$;
Internal Rate 0f Return: $r \ge 0 : PV(C) = 0$;

\subsubsection{Newton Raphson Iterative Method}
$\alpha_{n+1} = \alpha_n - \frac{f(\alpha_n)}{f'(\alpha_n)}$, for $\alpha_0$ close
		to $\alpha$ (with IVT);
Initial estimate $\alpha_0$ can be $\frac{a+b}{2}$.

\subsubsection{Annuities}
A series of payments at regular intervals. 
Annuity-due: beginning of period. 
Annuity immediate (ordinary/in arrears): end of period. 
Perpetual annuity : infinite number of payments. 

\subsubsection{Loans}
For loan $L$ to be repaid with $C$, $L = \text{PV}(C) = \sum_i^n \frac{c_i}{a(t_i)}$;
	$L_m^\text{Balance}$ after and installment payment = PV of remaining payments.
When $n \not\in \mathbb{Z}$, $L = \text{PV($\lfloor n \rfloor$pmts)} + \text{PV}(B,T)$, 
where $B$ is made at $T > \lfloor n \rfloor$.

\subsubsection{Force of Interest}
$\delta (t) = \frac{a'(t)}{a(t)} = \frac{d(\ln a(t))}{dt}$;
$a(t) = \text{exp} \int_0^t \delta(s) ds$;
$a(p,q) = \frac{a(q)}{a(p)} = \text{exp} \int_p^q \delta(t) dt$;

\subsection{Chapter 2: Bonds and Term Structure}

\subsubsection{Bond Valuations}
For $R$ = redemption value, $m$ = \#coupons/yr, $n$ = \#payments, $\lambda$ =
	nominal yield, $c$ = nominal coupon rate, $P = \frac{R}{1 + \lambda/m)^n} +
	\sum_{i=1}^n \frac{(c/m)F}{(1+\lambda/m)^i}$.
When $F = R$, $P = F + F(\frac{c - \lambda}{\lambda}) \left[ 1 -
	\frac{1}{(1+\lambda/m)^n} \right]$;
$P = F \Leftrightarrow c = \lambda$;
$P < F \Leftrightarrow c < \lambda$;
For $P_k$ := price after $k$th payment, $P = \frac{R}{1 + \lambda/m)^{n-k}} +
	\sum_{i=1}^{n-k} \frac{(c/m)F}{(1+\lambda/m)^i}$,
$P_{k+1} = P_k(1+\lambda/m) - Fc/m$;
Zero coupon bonds: $P = \frac{F}{(1+\lambda)^n}$;
Between coupon payments: $0 \le \epsilon < 1$, $P_{k+\epsilon} = (1+\lambda/m)^k$;

\subsubsection{Macaulay Duration}
$D = \frac{1}{\text{PV}(C)} \sum_{i=0}^n t_i PV(c_i) = \sum_{i=1}^n t_i w_i$,
where $w_i = \frac{\text{PV}(c_i)}{\text{PV}(C)}$;
$c_i \ge 0 \forall i \Rightarrow t_0 \le D \le t_n$;
Zero-coupon bonds: $D = T$;
Coupon bonds: $D = \frac{1}{P} \left[ \sum_{i=1}^{n}
	\frac{(i/m)(cF/m)}{(1+\lambda/m)^i} + \frac{Fn/m}{(1+\lambda/m)^n} \right]$;
For $r$ = effective annual rate, $\frac{dP}{dr} = - \frac{D}{1+r} P$;

\subsubsection{Duration and Sensitivity}
$\mu := \lambda/m \land \gamma = c/m$;
$D = \frac{1+\mu}{m\mu} - \frac{1+\mu+n(\gamma - \mu)}{m\gamma[(1+\mu)^n -1] + m\mu}$;
$= \frac{1+\mu}{m\mu} \left( 1 - \frac{1}{(1+\mu)^n} \right)$ if $\mu = \gamma$;
$\lim_{n\to\infty} D = \frac{1+\mu}{m\mu} = \frac{1+\lambda/m}{\lambda}$;
$D_m = \frac{1}{1+\lambda/m}$;
$\frac{dP}{d\lambda} = \left(- \frac{1}{1+\lambda/m} D \right) P = -D_m P$;
$P(\lambda + \Delta\lambda) \approx P(\lambda) - (D_mP)\Delta\lambda$;
$\Delta P \approx (-D_mP)\Delta\lambda$;
\% change in price: $\frac{\Delta P}{P} \times 100\% \approx
	(-D_m)\Delta\lambda \times 100\%$;

\subsubsection{Duration of Bond Portfolio}
$D_p = \sum_{i=1}^n \alpha_i D_i$ where $\alpha_i =$ fraction invested in bond $i$.

\subsubsection{Convexity}
$P(\lambda + \Delta\lambda) \approx P(\lambda) + \frac{dP}{d\lambda} \Delta\lambda 
	+ \frac{1}{2} \frac{d^2P}{d\lambda^2}(\Delta\lambda)^2$;
Convexity: $C = \frac{1}{P} \cdot \frac{d^2P}{d\lambda^2}$;
$\Delta P = P(\lambda + \Delta\labda) - P(\lambda) \approx P \left[ 
	-D_m \Delta\lambda + \frac{C}{2}(\Delta\lambda)^2 \right]$;
$C = \frac{F}{P} \left[ \frac{2c}{\lambda^3} \left( 1 -
	\frac{1}{(1+\lambda/m)^n} \right) - \frac{2nc}{m\lambda^2 (1+\lambda/m)^{n+1}}
	- \frac{n(n+1)(c-\lambda)}{\lambda m^2(1+\lambda/m)^{n+2}} \right]$;

\subsubsection{Yield Curves and Term Structure of Interest Rates}
$f_{0,t} = s_t$;
Given a zero coupon bond: $s_n = \left( \frac{F}{P} \right)^{1/n} - 1$;
$(1+s_k)^k = (1+s_j)^j(1+f_{j,k})^{k-j}$;
$(1+s_n)^n = (1+s_1)(1+f_{1,2})\ldots(1+f_{n-1,n})$;
Given $(s_1, \ldots, s_n)$, $\text{PV}(C) = \sum_{i=1}^n \frac{c_i}{(1+s_i)^i}$;
Bootstrap Method: To find $s_n$ given $(s_1, \ldots, s_n)$ and $n$-year coupon
	bond, solve $P = \frac{c_1}{1+s_1} + \ldots + \frac{c_n}{1+s_n}$.

% You can even have references
%\rule{0.3\linewidth}{0.25pt}
%\scriptsize
%\bibliographystyle{abstract}
%\bibliography{refFile}

\end{multicols}
\end{document}
