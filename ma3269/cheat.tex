\documentclass[10pt,landscape]{article}
\usepackage{multicol}
\usepackage{calc}
\usepackage{ifthen}
\usepackage[landscape]{geometry}
\usepackage{mathtools,amsmath,amsthm,amsfonts,amssymb}
\usepackage{color,graphicx,overpic}
\usepackage{hyperref}
\usepackage{bm} % for \bm


\pdfinfo{
  /Title (example.pdf)
  /Creator (TeX)
  /Producer (pdfTeX 1.40.0)
  /Author (Seamus)
  /Subject (Example)
  /Keywords (pdflatex, latex,pdftex,tex)}

% This sets page margins to .5 inch if using letter paper, and to 1cm
% if using A4 paper. (This probably isn't strictly necessary.)
% If using another size paper, use default 1cm margins.
\ifthenelse{\lengthtest { \paperwidth = 11in}}
    { \geometry{top=.5in,left=.5in,right=.5in,bottom=.5in} }
    {\ifthenelse{ \lengthtest{ \paperwidth = 297mm}}
        {\geometry{top=1cm,left=1cm,right=1cm,bottom=1cm} }
        {\geometry{top=1cm,left=1cm,right=1cm,bottom=1cm} }
    }

% Turn off header and footer
\pagestyle{empty}

% Redefine section commands to use less space
\makeatletter
\renewcommand{\section}{\@startsection{section}{1}{0mm}%
                                {-1ex plus -.5ex minus -.2ex}%
                                {0.5ex plus .2ex}%x
                                {\normalfont\large\bfseries}}
\renewcommand{\subsection}{\@startsection{subsection}{2}{0mm}%
                                {-1explus -.5ex minus -.2ex}%
                                {0.5ex plus .2ex}%
                                {\normalfont\normalsize\bfseries}}
\renewcommand{\subsubsection}{\@startsection{subsubsection}{3}{0mm}%
                                {-1ex plus -.5ex minus -.2ex}%
                                {1ex plus .2ex}%
                                {\normalfont\small\bfseries}}
\makeatother

% Define BibTeX command
\def\BibTeX{{\rm B\kern-.05em{\sc i\kern-.025em b}\kern-.08em
    T\kern-.1667em\lower.7ex\hbox{E}\kern-.125emX}}

% Don't print section numbers
\setcounter{secnumdepth}{0}


\setlength{\parindent}{0pt}
\setlength{\parskip}{0pt plus 0.5ex}

%My Environments
\newtheorem{example}[section]{Example}
% -----------------------------------------------------------------------

\begin{document}
\raggedright
\footnotesize
\begin{multicols}{3}


% multicol parameters
% These lengths are set only within the two main columns
%\setlength{\columnseprule}{0.25pt}
\setlength{\premulticols}{1pt}
\setlength{\postmulticols}{1pt}
\setlength{\multicolsep}{1pt}
\setlength{\columnsep}{2pt}

\begin{center}
     \Large{\underline{MA3269 Mathematical Finance I}} \\
\end{center}

Lim Wen Bin A0140764H

\subsection{Chapter 0: Revision}

\subsubsection{Series}
$\sum_{i=0}^n y^i = \frac{1-y^{n+1}} {1-y}$;
$\sum_{i=0}^\infty y^i = \frac{1} {1-y}$ for $|y|<1$;
$\sum_{i=1}^n iy^{i-1} = \frac{1-y^n (1+n-ny)} {(1-y)^2}$;
$\sum_{i=1}^\infty iy^{i-1} = \frac{1} {(1-y)^2}$ for $|y|<1$;
$\sum_{i=1}^n i = \frac{1}{2} n(n+1)$;
$\sum_{i=1}^n i^2 = \frac{1}{6} n(n+1)(2n+1)$;
$\sum_{i=1}^n i^3 = \frac{1}{4} n^2(n+1)^2$;
$\sum_{i=m}^n u_i = \sum_{i=k}^n u_i = \sum_{i=k}^{m-1} u_i$;

\subsubsection{Calculus}
AM-GM Inequality: $\frac{x+y}{2} \ge \sqrt{xy} \Leftrightarrow \frac{\sqrt{xy}}{x+y} \le \frac{1}{2}$;
$f$ continuous on $[a,b] \land f(a)f(b)<0 \Rightarrow \exists x \in (a,b) : f(x)=0$.
For injective and differentiable $f$ on open interval $I$, $f^{-1}$ is
differentiable on its domain and $(f^{-1}(y))' = \frac{1} {f'(f^{-1}(y))}$;
L'Hopital's Rule: for $\frac{0}{0}$ or $\frac{\infty}{\infty}$, for $c=\infty$ or $c=a^+/a^-$
$\lim_{x\to c} \frac{f(x)}{g(x)} = \lim_{x\to c} \frac{f'(x)}{g'(x)}$;
Taylor's Theorem: $f(b) = \sum_{k=0}^{n} \frac{f^{(k)}(a)}{k!} (b-a)^k +
\frac{f^{(n+1)}(c)}{(n+1)!} (b-a)^{n+1}$ for some $c \in (a,x)$;
Mean Value Theorem: for continuous $f$ on $[a,b]$, $\exists c \in (a,b) : 
f'(c) = \frac{f(b)-f(a)}{b-a}$;
$f$ is increasing if $a<b \Rightarrow f(a)<f(b)$ or $f'(x)>0\ \forall x \in (a,b)$;
$f$ is convex at $c$ if it is differentiable and its graph lies below the tangent line;
Jensen's Inequalities: for convex(concave) $f$ over $(a,b)$, $(x_n) \in (a,b)$
and $(\alpha_n) \in [0,1] : \sum_i \alpha_i = 1$, 
$f\left( \sum_{i=1}^n \alpha_i x_i \right) \le (\ge)\ f \sum_{i=1}^n \alpha_i f(x_i)$;

\subsubsection{Derivatives}
$f'(x) = \lim_{h\to0} \frac{f(x+h)-f(x)} {h}$;
$\frac{d}{dx} \tan(f(x)) = f'(x)\sec^2(f(x))$;
$\frac{d}{dx} \csc(f(x)) = f'(x)\csc(f(x))\cot(f(x))$;
$\frac{d}{dx} \sec(f(x)) = f'(x)\sec(f(x))\tan(f(x))$;
$\frac{d}{dx} \sin^{-1}(f(x)) = \frac{f'(x)} {\sqrt{1-(f(x))^2}}$;
$\frac{d}{dx} \cos^{-1}(f(x)) = -\frac{f'(x)} {\sqrt{1-(f(x))^2}}$;
$\frac{d}{dx} \tan^{-1}(f(x)) = \frac{f'(x)} {1+(f(x))^2}$;
$\frac{d}{dx} (f(x))^{g(x)} = (f(x))^{g(x)} \frac{d}{dx} (g(x) \ln f(x))$;
Quotient Rule: $\frac{d}{dx} \frac{f(x)}{g(x)} = \frac{f'(x)g(x)-g'(x)f(x)}{(g(x))^2}$;

\subsubsection{Integrals}
$\int \frac{1} {a^2+(x+b)^2} dx = \frac{1}{a} \tan^{-1} \left( \frac{x+b}{a} \right)$;
$\int \frac{1} {\sqrt{a^2+(x+b)^2}} dx = \sin^{-1} \left( \frac{x+b}{a} \right)$;
$\int -\frac{1} {\sqrt{a^2+(x+b)^2}} dx = \cos^{-1} \left( \frac{x+b}{a} \right)$;
$\int \frac{1} {a^2-(x+b)^2} dx = \frac{1}{2a} \ln \left| \frac{x+b+a}{x+b-a} \right|$;
$\int \frac{1} {(x+b)^2-a^2} dx = \frac{1}{2a} \ln \left| \frac{x+b-a}{x+b+a} \right|$; 
\\
Substitution I: $u=g(x), \int f(g(x))g'(x) dx = \int f(u)du$;
Substitution II: $x=g(t), \int f(x)dx = \int f(g(t))g'(t)dt$;
Integration by parts: $\int fg' dx = fg - \int f'g dx$;
FTC: $\int_{a}^{b} F'(x) dx = F(b) - F(a)$;

\subsection{Chapter 1: Theory of Interest}

\subsubsection{Interest}
Simple interest: $a(t) = 1 + rt$;
Compound interest: $a(t) = (1 + r)^t$;
Frequency compounding: $a(t) = \left(1 + \frac{r^{(p)}}{p} \right)^p$;
Effective annual interest: $1+r_e = \left(1 + \frac{r^{(p)}}{p} \right)^p$;
Continuous compounding: $a(t) = \lim_{p\to\infty} \left(1 +
	\frac{r^{(\infty)}}{p} \right)^p = e^{r^{(\infty)}}$;
$\forall p \in \mathbb{Z}^+, r>0, e^r > \left(1 + \frac{r}{p} \right)^p$;

\subsubsection{Present Value and Time Value}
$\text{PV}(C) = \sum_{i=1}^n \frac{c_i}{a(t_i)}$;
$\text{TV}_t(C) = \text{PV}(C) \cdot a(t)$;
$\text{TV}(C,t) = \frac{a(t)}{a(s)} \cdot \text{TV}(C,s)$;
$C_1$ and $C_2$ are equivalent if $PV(C_1) = PV(C_2)$;
Deferred Cash Flow: $\frac{\text{PV}(C)}{\text{PV}(C_{(+k)})} = (1+r)^k$;
Equation of Value: $\text{PV}(C) = \sum_{i=1}^n \frac{c_i}{(1+r)^{t_i}} = 0$;
Internal Rate 0f Return: $r \ge 0 : PV(C) = 0$;

\subsubsection{Newton Raphson Iterative Method}
$\alpha_{n+1} = \alpha_n - \frac{f(\alpha_n)}{f'(\alpha_n)}$, for $\alpha_0$ close
		to $\alpha$ (with IVT);
Initial estimate $\alpha_0$ can be $\frac{a+b}{2}$.

\subsubsection{Annuities}
A series of payments at regular intervals. 
Annuity-due: beginning of period. 
Annuity immediate (ordinary/in arrears): end of period. 
Perpetual annuity : infinite number of payments. 

\subsubsection{Loans}
For loan $L$ to be repaid with $C$, $L = \text{PV}(C) = \sum_i^n \frac{c_i}{a(t_i)}$;
	$L_m^\text{Balance}$ after and installment payment = PV of remaining payments.
When $n \not\in \mathbb{Z}$, $L = \text{PV($\lfloor n \rfloor$pmts)} + \text{PV}(B,T)$, 
where $B$ is made at $T > \lfloor n \rfloor$.

\subsubsection{Force of Interest}
$\delta (t) = \frac{a'(t)}{a(t)} = \frac{d(\ln a(t))}{dt}$;
$a(t) = \text{exp} \int_0^t \delta(s) ds$;
$a(p,q) = \frac{a(q)}{a(p)} = \text{exp} \int_p^q \delta(t) dt$;

\subsection{Chapter 2: Bonds and Term Structure}

\subsubsection{Bond Valuations}
For $R$ = redemption value, $m$ = \#coupons/yr, $n$ = \#payments, $\lambda$ =
	nominal yield, $c$ = nominal coupon rate, $P = \frac{R}{1 + \lambda/m)^n} +
	\sum_{i=1}^n \frac{(c/m)F}{(1+\lambda/m)^i}$.
When $F = R$, $P = F + F(\frac{c - \lambda}{\lambda}) \left[ 1 -
	\frac{1}{(1+\lambda/m)^n} \right]$;
$P = F \Leftrightarrow c = \lambda$;
$P < F \Leftrightarrow c < \lambda$;
For $P_k$ := price after $k$th payment, $P = \frac{R}{1 + \lambda/m)^{n-k}} +
	\sum_{i=1}^{n-k} \frac{(c/m)F}{(1+\lambda/m)^i}$,
$P_{k+1} = P_k(1+\lambda/m) - Fc/m$;
Zero coupon bonds: $P = \frac{F}{(1+\lambda)^n}$;
Between coupon payments: $0 \le \epsilon < 1$, $P_{k+\epsilon} = (1+\lambda/m)^k$;

\subsubsection{Macaulay Duration}
$D = \frac{1}{\text{PV}(C)} \sum_{i=0}^n t_i PV(c_i) = \sum_{i=1}^n t_i w_i$,
where $w_i = \frac{\text{PV}(c_i)}{\text{PV}(C)}$;
$c_i \ge 0 \forall i \Rightarrow t_0 \le D \le t_n$;
Zero-coupon bonds: $D = T$;
Coupon bonds: $D = \frac{1}{P} \left[ \sum_{i=1}^{n}
	\frac{(i/m)(cF/m)}{(1+\lambda/m)^i} + \frac{Fn/m}{(1+\lambda/m)^n} \right]$;
For $r$ = effective annual rate, $\frac{dP}{dr} = - \frac{D}{1+r} P$;

\subsubsection{Duration and Sensitivity}
$\mu := \lambda/m \land \gamma = c/m$;
$D = \frac{1+\mu}{m\mu} - \frac{1+\mu+n(\gamma - \mu)}{m\gamma[(1+\mu)^n -1] + m\mu}$;
$= \frac{1+\mu}{m\mu} \left( 1 - \frac{1}{(1+\mu)^n} \right)$ if $\mu = \gamma$;
$\lim_{n\to\infty} D = \frac{1+\mu}{m\mu} = \frac{1+\lambda/m}{\lambda}$;
$D_m = \frac{1}{1+\lambda/m}$;
$\frac{dP}{d\lambda} = \left(- \frac{1}{1+\lambda/m} D \right) P = -D_m P$;
$P(\lambda + \Delta\lambda) \approx P(\lambda) - (D_mP)\Delta\lambda$;
$\Delta P \approx (-D_mP)\Delta \lambda$;
\% change in price: $\frac{\Delta P}{P} \times 100\% \approx
	(-D_m)\Delta \lambda \times 100\%$;

\subsubsection{Duration of Bond Portfolio}
$D_p = \sum_{i=1}^n \alpha_i D_i$ where $\alpha_i =$ fraction invested in bond $i$.

\subsubsection{Convexity}
$P(\lambda + \Delta\lambda) \approx P(\lambda) + \frac{dP}{d\lambda} \Delta\lambda 
	+ \frac{1}{2} \frac{d^2P}{d\lambda^2}(\Delta\lambda)^2$;
Convexity: $C = \frac{1}{P} \cdot \frac{d^2P}{d\lambda^2}$;
$\Delta P = P(\lambda + \Delta\lambda) - P(\lambda) \approx P \left[ 
	-D_m \Delta\lambda + \frac{C}{2}(\Delta\lambda)^2 \right]$;
$C = \frac{F}{P} \left[ \frac{2c}{\lambda^3} \left( 1 -
	\frac{1}{(1+\lambda/m)^n} \right) - \frac{2nc}{m\lambda^2 (1+\lambda/m)^{n+1}}
	- \frac{n(n+1)(c-\lambda)}{\lambda m^2(1+\lambda/m)^{n+2}} \right]$;

\subsubsection{Yield Curves and Term Structure of Interest Rates}
$f_{0,t} = s_t$;
Given a zero coupon bond: $s_n = \left( \frac{F}{P} \right)^{1/n} - 1$;
$(1+s_k)^k = (1+s_j)^j(1+f_{j,k})^{k-j}$;
$(1+s_n)^n = (1+s_1)(1+f_{1,2})\ldots(1+f_{n-1,n})$;
Given $(s_1, \ldots, s_n)$, $\text{PV}(C) = \sum_{i=1}^n \frac{c_i}{(1+s_i)^i}$;
Bootstrap Method: To find $s_n$ given $(s_1, \ldots, s_n)$ and $n$-year coupon
	bond, solve $P = \frac{c_1}{1+s_1} + \ldots + \frac{c_n}{1+s_n}$.

\subsection{Chapter 3: Expected Utility Theory}
$E[U(W)] = E[U(X+w_0)] = \sum_{i=1}^n p_i U(x_i + w_0)$;
Accept risky prospect if: $E[U(X+w_o)] > U(w_0)$;
Risk adverse: $U$ is strictly concave $\Leftrightarrow U'' < 0 \Leftrightarrow
	E[U(X)] < U(E[X])$ (Jensen's inequality);
Risk neutral: $U$ is linear $\Leftrightarrow U'' = 0 \Leftrightarrow
	E[U(X)] = U(E[X])$;
Risk seeking: $U$ is strictly convex $\Leftrightarrow U'' > 0 \Leftrightarrow
	E[U(X)] > U(E[X])$;
Positive affine transformations: For $\alpha > 0, \beta \in \mathbb{R}$,
	$U \sim V \Leftrightarrow U = \alpha V + \beta$;

\subsubsection{Certainty Equivalent and Risk Premium}
Certainty equivalent: $U(CE) = E[U(w_0 + X))$, $CE > w_0
	\Rightarrow$ invest in $X$;
$U \sim V \Rightarrow CE(X, V) = CE(X, U)$;
Risk premium: $U(w_0 - RP) = E[U(w_0 + X)], RP = w_0 - CE$, $RP > w_0
	\Rightarrow$ invest in $X$;

\subsubsection{Arrow-Pratt Measures of Risk Aversion}
Absolute risk aversion function:
	$U_{\text{ARA}}(w) = -\frac{U''(w)}{U'(w)} = -[\ln(U')]'$;
Relative risk aversion function:
	$U_{\text{RRA}}(w) = - w\frac{U''(w)}{U'(w)}$;
$U_{\text{ARA}} > V_{\text{ARA}} \forall w \Leftrightarrow \exists g, g'' < 0 \text{
	s.t. } U = g(V)\Leftrightarrow U$ is globally more risk
	averse than $V$;
$U_{\text{ARA}} = V_{\text{ARA}} \Leftrightarrow U \sim V$;

Portfolio Selection: For random rate of return r, maximize $E[U(w_x(1 + \alpha r))]$ 
using closed interval method;
For $U(x) = a-be^{-\lambda x}$, two investments with normally distributed returns, 
	maximize $f(\alpha) = \alpha \mu_1 + (1-\alpha) \mu_2 - \frac{\lambda w_0}{2}
	(\alpha^2 \sigma_1^2 + (1-\alpha)^2 \sigma_2^2)$;

\subsection{Chapter 4: Mean-Variance Analysis}
Rate of return: $r = \frac{W_1 - W_0}{W_0}$;
Correlation: $\rho_{i,j} = \frac{\sigma_{i,j}}{\sigma_i \sigma_j}$;

\subsubsection{Portfolio Mean and Variance}
$\mu_p = \bm{w}^T \bm{u} = \sum_{i=1}^n w_i \mu_i$;
$\sigma_p^2 = \bm{w}^T\bm{C}\bm{w} 
	= \sum_{j=1}^n \sum_{i=1}^n w_i w_j \sigma_{i,j} 
	= \sum_{i=1}^n w_i^2 \sigma_{i}^2
	+ \sum_{j=1}^n \sum_{\substack{i=1,\\i\ne j}}^n w_i w_j \sigma_{i,j}$;
$\bm{C}$ is symmetric, invertible and positive definite.
Cov$(r_1,r_2) = \bm{w}_1^T\bm{C}\bm{w}_2$;
$\frac{\partial}{\partial\bm{w}} (\bm{w}^T\bm{C}\bm{w})
	= 2\bm{C}\bm{w}$;
$\frac{\partial}{\partial\bm{w}} (\bm{a}^T\bm{w}) = \bm{a}$;

\subsubsection{Portfolios of Two Assets}
$\sigma_p^2 = \alpha^2 \sigma_1^2 + (1-\alpha)^2 \sigma_2^2 +
	2\alpha(1-\alpha)\rho_{1,2}\sigma_1\sigma_2$;
$\alpha_{\text{GMVP}} = \alpha^* = \frac{\sigma_2(\sigma_2 - \rho_{1,2}\sigma_1)}
	{\sigma_1^2 + \sigma_2^2 - 2\rho_{1,2}\sigma_1\sigma_2}$;
$(\sigma^2)_{\text{GMVP}} = \frac{\sigma_1^2\sigma_2^2 (1 - \rho_{1,2}^2)}
	{\sigma_1^2 + \sigma_2^2 - 2\rho_{1,2}\sigma_1\sigma_2}$;
$\mu_{\text{GMVP}} = \alpha^* \mu_1 + (1-\alpha^*)\mu_2$;
Completing the square: $ax^2 + bx + c = a\left( x+\frac{b}{2a}
	\right)^2 + c - \frac{b^2}{4a}$;
Shortselling not allowed $\Rightarrow 0 \le \alpha \le 1$;
Feasible set $F$: $\rho = 1 \Rightarrow F$ is a line, $\rho = -1 \Rightarrow F$
	is V-shaped graph of two straight lines meeting at the GMVP, $|\rho| < 1
	\Rightarrow F$ is a curve connecting the two points.

Feasible set of portfolio with $n > 2$ assets: $F$ is convex; $\forall \mu \in
	\mathbb{R}, \exists \sigma^* : (\sigma^*, \mu) \in F, (\sigma,\mu) \in F \Rightarrow
	\sigma^* \le \sigma$;

\subsection{Chapter 5: Portfolio Theory \& CAPM}
\subsubsection{Lagrange Multiplier Derivation}
Solve $\frac{\partial L}{\partial w_i} = 0$ i.e.
	$\bm{C} \bm{w} = \lambda_1 \bm{1} + \lambda_2 \bm{\mu}$,
	$\bm{1}^T \bm{w} = 1$,
	and $\bm{\mu}^T \bm{w} = \mu$.
\[
	A 
	= \left( \begin{array}{cc}
		a & b\\
		b & c\\
	\end{array} \right)
	=
	\left( \begin{array}{cc}
		\bm{1}^T \bm{C}^{-1} \bm{1} & \bm{1}^T \bm{C}^{-1} \bm{\mu}\\
		\bm{1}^T \bm{C}^{-1} \bm{\mu} & \bm{\mu}^T \bm{C}^{-1} \bm{\mu}\\
	\end{array} \right),\
	A
	\bm{\lambda}
	=
	\left( \begin{array}{c}
		1\\
		\mu\\
	\end{array} \right)
\]
$A^{-1}$ yields $\lambda_1 = \frac{c-b\mu}{ac-b^2},\ 
	\lambda_2 = \frac{a\mu-b}{ac-b^2}$;
$ac-b^2 = \Delta > 0$;

\subsubsection{Minimum Variance Frontier}
$\bm{w}_{\mu}^* = \lambda_1 \bm{C}^{-1} \bm{1} + \lambda_2 \bm{C}^{-1} \bm{\mu}$;
$\sigma_p^2 = \frac{a\mu_p^2 - 2b\mu_p + c}{ac - b^2}$;
Completing the square, minimum-variance frontiers curve: $\sigma_p^2 =
	\frac{a}{\Delta} \left( \mu - \frac{b}{a} \right)^2 + \frac{1}{a}$;
Equivalently, $a\sigma^2 - \frac{a^2}{\Delta} \left( \mu - \frac{b}{a}
	\right)^2 = 1$;
$\sigma_{GMVP} = \sqrt{\frac{1}{a}}$;
$\mu_{\text{GMVP}} = \frac{b}{a}$;
$\bm{w}_{\text{GMVP}} = \frac{\bm{C}^{-1} \bm{1}}{\bm{1}^T \bm{C}^{-1} \bm{1}}$;
Asymptotes: $\sigma = \sqrt{\frac{a}{\Delta}} \left| \mu - \frac{b}{a} \right|$;
$\forall$ portfolio $p$, Cov$(r_p, r_{\text{GMVP}}) = \sigma^2_{\text{GMVP}}$;
The upper half of the minimum variance frontier is the efficient frontier 
$\mu_p > \frac{b}{a}$.

\subsubsection{Two Fund Theorem}
$w_{f1} = \frac{\bm{C}^{-1} \bm{1}}{\bm{1}^T \bm{C}^{-1} \bm{1}} (\text{GMVP}),\ 
	w_{f2} = \frac{\bm{C}^{-1} \bm{\mu}}{\bm{1}^T \bm{C}^{-1} \bm{\mu}}$;
$\mu_{f1} = \frac{b}{a},\ \mu_{f2} = \frac{c}{b}$;
$\sigma_{f1} = \frac{1}{a},\ \sigma_{f2} = \frac{c}{b^2}$;
$b > 0 \Rightarrow \frac{c}{b} > \frac{b}{a} (\because \Delta > 0) \Rightarrow$ 
	fund 2 is efficient;
For $\bm{w}_1 \ne \bm{w}_2$ on the MVF, $\bm{w}$ is on the MVF $\Leftrightarrow
	\exists \alpha \in \mathbb{R} \text{ s.t. } \bm{w} = \alpha \bm{w}_1 +
	(1-\alpha) \bm{w}_2$;
If $\{\bm{w}_1, \bm{w}_2\}$ are efficient and $\bm{w} \in
	\text{span}\{\bm{w}_1, \bm{w}_2\}$, $\bm{w}$ is efficient.
A portfolio on the MVF is efficient if $\mu > \frac{b}{a}$;

\subsubsection{Portfolios with a Risk-free Asset}
$\mu_p = (1 - \bm{1}^T \bm{w}) r_f + \bm{w}^T \bm{\mu}$;
$\sigma_p^2 = \bm{w}^T \bm{C} \bm{w}$;
MVP with mean $\mu$:
	$\bm{w} = \frac{ (\mu-r_f) \bm{C}^{-1} (\bm{\mu} - r_f\bm{1}) }
		{(\bm{\mu}-r_f\bm{1})^T \bm{C}^{-1} (\bm{\mu} - r_f\bm{1})}$,
	$\sigma^2 = \frac{(\mu - r_f)^2}{c - 2br_f + ar_f^2}$;
Tangency portfolio (only portfolio that lies on MVF and CML):
	$\bm{w}_{\tan} = \frac{ \bm{C}^{-1} (\bm{\mu} - r_f\bm{1}) }
		{\bm{1}^T \bm{C}^{-1} (\bm{\mu} - r_f\bm{1})}$,
	$\mu_{\tan} = \frac{\bm{\mu}^T \bm{C}^{-1} (\bm{\mu} - r_f\bm{1})}
		{\bm{1}^T \bm{C}^{-1} (\bm{\mu} - r_f\bm{1})}$
		$= \frac{c-r_f b}{b-r_f a}$,
	$\sigma_{\tan} = \frac{(\bm{\mu}-r_f\bm{1})^T \bm{C}^{-1} (\bm{\mu} - r_f\bm{1})}
		{[\bm{1}^T \bm{C}^{-1} (\bm{\mu} - r_f\bm{1})]^2}
		= \frac{c-2r_f b+ar_f^2}{(b-r_f a)^2}$;
In equilibrium, market portfolio = tangency portfolio.

\subsubsection{Capital Market Line}
CML is tangent to MVF and is the efficient frontier for portfolio involving
	both risky and risk-free assets.
Equation for MVF of $(n+1)$ assets:
	$\sigma = (c-2br_f+ar_f^2)^{-1/2} |\mu-r_f|$;
Equation of CML:
	$\mu_p = r_f + \frac{\sigma_p}{\sigma_{\tan}}(\mu_{\tan}-r_f)$;
Sharpe ratio: $SR_p = \frac{\mu_p - r_f}{\sigma_p}$, portfolios on the CML have the highest $SR$.
One fund theorem: Investors only hold portfolios on the CML, differing only in
	the portion $\alpha$ invested in the tangency portfolio.

\subsubsection{Capital Asset Pricing Model}
Market portfolio: $w_i = \frac{u_i p_i}{\sum_{i=1}^n u_i p_i}$, in equilibrium its equal to the tangency portfolio; \\
CML: $\mu_p - r_f = \frac{\sigma_p}{\sigma_m}(\mu_m - r_f)
	= \frac{\sigma_p}{\sigma_m} RP$; \\
CAPM: $\mu_i - r_f = \frac{\sigma_{i,m}}{\sigma_m^2}(\mu_m - r_f)
	= \beta(\mu_m - r_f)$;
$\beta_p = \sum_{i=1}^n w_i \beta_i = \bm{w}^T \bm{\beta}$;
$\forall p$ on the CML, $r_p = \alpha r_m + (1-\alpha)r_f,\ \alpha > 0$,
$\beta_p = \alpha,\ \sigma_p = \alpha \sigma_m$;
$\mu_i = \mu_j \Leftrightarrow \beta_i = \beta_j$;
An asset if undervalued (underpriced) if estimated return $>$ CAPM return.
SML: For all assets $i$, $\mu_i r_f + \beta_i(\mu_m - r_f)$;
Systematic risk:
	Var$(r_i) = \beta_i^2 \sigma_m^2 + \text{Var}(\epsilon_i)$,
	$\epsilon_i = r_i - [r_f + \beta_i (r_m - r_f)]$, $E[\epsilon_i] = 0$, 
	Cov$(\epsilon_i,r_m) = 0$;
An asset has systematic risk $\Leftrightarrow$ it lies on the CML;

% You can even have references
%\rule{0.3\linewidth}{0.25pt}
%\scriptsize
%\bibliographystyle{abstract}
%\bibliography{refFile}

\end{multicols}
\end{document}
