\documentclass[10pt,landscape]{article}
\usepackage{multicol}
\usepackage{calc}
\usepackage{ifthen}
\usepackage[landscape]{geometry}
\usepackage{mathtools,amsmath,amsthm,amsfonts,amssymb}
\usepackage{color,graphicx,overpic}
\usepackage{hyperref}


\pdfinfo{
  /Title (example.pdf)
  /Creator (TeX)
  /Producer (pdfTeX 1.40.0)
  /Author (Seamus)
  /Subject (Example)
  /Keywords (pdflatex, latex,pdftex,tex)}

% This sets page margins to .5 inch if using letter paper, and to 1cm
% if using A4 paper. (This probably isn't strictly necessary.)
% If using another size paper, use default 1cm margins.
\ifthenelse{\lengthtest { \paperwidth = 11in}}
    { \geometry{top=.5in,left=.5in,right=.5in,bottom=.5in} }
    {\ifthenelse{ \lengthtest{ \paperwidth = 297mm}}
        {\geometry{top=1cm,left=1cm,right=1cm,bottom=1cm} }
        {\geometry{top=1cm,left=1cm,right=1cm,bottom=1cm} }
    }

% Turn off header and footer
\pagestyle{empty}

% Redefine section commands to use less space
\makeatletter
\renewcommand{\section}{\@startsection{section}{1}{0mm}%
                                {-1ex plus -.5ex minus -.2ex}%
                                {0.5ex plus .2ex}%x
                                {\normalfont\large\bfseries}}
\renewcommand{\subsection}{\@startsection{subsection}{2}{0mm}%
                                {-1explus -.5ex minus -.2ex}%
                                {0.5ex plus .2ex}%
                                {\normalfont\normalsize\bfseries}}
\renewcommand{\subsubsection}{\@startsection{subsubsection}{3}{0mm}%
                                {-1ex plus -.5ex minus -.2ex}%
                                {1ex plus .2ex}%
                                {\normalfont\small\bfseries}}
\makeatother

% Define BibTeX command
\def\BibTeX{{\rm B\kern-.05em{\sc i\kern-.025em b}\kern-.08em
    T\kern-.1667em\lower.7ex\hbox{E}\kern-.125emX}}

% Don't print section numbers
\setcounter{secnumdepth}{0}


\setlength{\parindent}{0pt}
\setlength{\parskip}{0pt plus 0.5ex}

%My Environments
\newtheorem{example}[section]{Example}
% -----------------------------------------------------------------------

\begin{document}
\raggedright
\footnotesize
\begin{multicols}{3}


% multicol parameters
% These lengths are set only within the two main columns
%\setlength{\columnseprule}{0.25pt}
\setlength{\premulticols}{1pt}
\setlength{\postmulticols}{1pt}
\setlength{\multicolsep}{1pt}
\setlength{\columnsep}{2pt}

\begin{center}
     \Large{\underline{FIN2004X Finance}} \\
\end{center}

Lim Wen Bin A0140764H

\subsection{Lecture 1: Financial Management Overview}

Finance is a discipline concerned with \textbf{determining value} and
\textbf{making decisions based on that value assessment}. \\

Main areas of finance include: 1. Investments, 2. Financial Markets and
Intermediaries and 3. Corporate Finance. \\

\textbf{Investment Vehicle Model}: Investors provide financing in exchange
for financial securities, Firms invest funds in assets, Income generated
is distributed to the investors. \\

\textbf{Balance Sheet/Accounting Model}: Investment decisions are
represented on the asset side of the balance sheet, Financing decisions
are represented on the liabilities and equity side. \\

Corporate Finance can be thought as the analysis of \textbf{Capital
structure}, \textbf{Capital budgeting} and \textbf{Net working capital}. \\

\subsubsection{Forms of Business Organisation} 

1. \textbf{Sole proprietorship} Advantages: Easy to start, least
   regulated, single owner keeps profits, taxed once as personal income.
   Disadvantages: Limited to life of owner, capital limited to owner
   wealth, unlimited liability, difficult to sell ownership. 
   
2. \textbf{Partnership} Advantages: Two or more owners, more capital,
   relatively easy to start, Income taxed once as personal income.
   Disadvantages: Unlimited liablity (General vs. Limited partnership),
   Dissolved when one owners dies or sells out, Difficult to transfer
   ownership.

3. \textbf{Corporation} Advantages: Limited liability, unlimited life,
   seperation of ownership and management, easy transfer of ownershipo,
   easy capital raising. Disadvantages: Agency costs, double taxation
   (corporate and personal income)

\subsubsection{Sources of External Corporate Financing} 

1. \textbf{Debt} Lender lend money with debt contracts and become the
   corporations's creditors, possessing higher claim priority in
   a liquidation event. 

2. \textbf{Equity} By buying shares, shareholders become residual
   claimants of the firm, receiving payoffs in terms of dividends sand
   capital gains.

\textbf{Goal of Financial Management}: Maximize stock prices, i.e.
increase the intrinsic value of the stock my influencing the amout, timing
and riskiness of cash flow streams. \\

\subsubsection{The Agency Problem}

Conflicts of interests between: Shareholders and managers, shareholders
and creditors. \\

Costs: Direct agency costs (management expenses and monitoring costs),
Indirect agency costs (lost opportunities).

Solutions: Appropriate compensation plans, monitoring by lenders and
investors, threat of firing, awareness of good corporate governance.

\subsubsection{Financial Markets}

\textbf{Money markets}: Trades debt securities of one year maturity or
less, involves loosely connected dealer markets, major bank involvement.

\textbf{Capital markets}: Trades equity and long-term debt claims, auction
markets like SGX.

\textbf{Primary markets}: Goverment and corporate issued securities,
public offerings, private offerings (to FIs and HNWI)

\textbf{Secondary markets}: Trades existing claims, Dealer markets (OTC),
Auction market (SGX), Easy retrieval of market value.

\subsection{Lecture 2: Financial Statement Analysis}

The annual report issued to stockholders includes the: Balance sheet,
Income statement, Statement of retained earning, Statement of cash flows.

\subsubsection{Balance Sheet}

Assets = Liailities + Equity: Lists item from decreasing liquidity, valued
at original cost (except for marketable securities and inventories). \\

\textbf{Market value of shareholders' equity} = Share price * Number of
outstanding shares;
\textbf{Enterprise value} = Market value of assets + Debt - Excess cash
(and other non-operating assets); 
\textbf{Market to book ratio} = Market value of X / Book value of X;

The \textbf{Income Statement} shows revenues and expenses and taxes
associated. 
\textbf{EBIT} = Earnings before interest and tax = Net sales - COGS -
Depreciation;
\textbf{Net income} = Net sales (revenue) - COGS - Expense;

\textbf{Statement of Retained Earning}: EOY RE = Initial RE + NI
- Dividends. \\

\subsubsection{Statement of Cash Flows}

Changes in cash divided into 3 categories: \textbf{Operating activities},
\textbf{Investment activites}, \textbf{Financing activites}. \\

NWC = Cash + Other CA - CL (CL = non-interest CL + interest CL); 
$\Rightarrow$ $\Delta$Cash = $\Delta$RE + $\Delta$CL - $\Delta$Non-cash CA
- $\Delta$Net fixed assets + $\Delta$LT-debt + $\Delta$Common stock; \\

\textbf{After tax cash flow} = Net income after tax + depreciation +
amortisation + other non-cash changes;

\textbf{Operating working capital} stems from operations and is removed
from financing decisions (excludes interest bearing CL i.e. notes payable). \\

\textbf{Cash flow from assets (CFFA/FCF)} = Operating cash flow (OCF)
- Net capital spending (NCS) - $\Delta$Net operating working capital
(NOWC);
Alternatively, CFFA = NI + depreciation - AP - AR + Inventory - NCS;

\textbf{OCF} = EBIT + Depreciation - (EBIT * Tax rate);
\textbf{NCS} = Ending fixed assets - Beginning fixed assets + Depreciation;
\textbf{NOWC} = CA - CL (Excluding notes payable);
$\Delta$\textbf{NOWC} = Ending NOWC - Beginning NOWC;
\textbf{Interest tax shield} = Cash saved from reduction in taxes due to
tax-free interest = Interest * Tax rate;
\textbf{Cash flow to creditors} = Interest paid - Net new borrowing (LT-Debt
and Note payable);
\textbf{Cash flow to stockholders} = Dividends - Net new equity raised;
\textbf{Cash flow to creditors and stockholders} = CFFA + Interest tax shield;

\subsubsection{Financial Ratios}

1. Liquidity Ratios: 
\textbf{Current ratio} = $\frac{\text{CA}}{\text{CL}}$;
\textbf{Quick ratio} = $\frac{\text{CA - Inventory}}{\text{CL}}$;
\textbf{Cash ratio} = $\frac{\text{Cash}}{\text{CL}}$;
\textbf{NWC to total assets} = $\frac{\text{NWC}}{\text{TA}}$;
\textbf{Interval measure} = $\frac{\text{CA}}{\text{Average daily operating cost}} =
\frac{\text{CA}}{\text{(COGS + Other expenses)/365}}$;
\\

2. LT solvency: 
\textbf{Total debt ratio} = $\frac{\text{TD}}{\text{TA}}$;
\textbf{Debt/equity ratio} = $\frac{\text{TA - TE}}{\text{TE}}$;
\textbf{Equity multiplier} = $\frac{\text{TA}}{\text{TE}}$
= 1 + Debt/equity ratio;
\textbf{LT debt ratio} = $\frac{\text{LT debt}}{\text{LT debt + TE}}$;
\textbf{Times interest earned ratio} = $\frac{\text{EBIT}}{\text{Interest}}$;
\textbf{Cash coverage ratio} = $\frac{\text{EBIT + Depreciation}}
{\text{Interest}}$;
\\

3. Asset managment ratios: 
\textbf{Inventory turnover} = $\frac{\text{COGS}}{\text{Inventory}}$;
\textbf{Days' sales in inventory} = $\frac{\text{365}}{\text{Inventory turnover}}$;
\textbf{Rec. turnover} = $\frac{\text{Sales}}{\text{Receivables}}$
\textbf{Days sales outstanding} = $\frac{\text{AR}}{\text{Average daily sales}}$;
= $\frac{\text{365}}{\text{Rec. turnover}}$;
\textbf{FA turnover} = $\frac{\text{Sales}}{\text{Net fixed assets}}$;
\textbf{TA turnover} = $\frac{\text{Sales}}{\text{Total assets}}$;
\\

4. Profitability: 
\textbf{Profit margin} = $\frac{\text{NI}}{\text{Sales}}$;
\textbf{Basic earning power} = $\frac{\text{EBIT}}{\text{TA}}$;
\textbf{ROA} = $\frac{\text{NI}}{\text{TA}}$
\textbf{ROE} = $\frac{\text{NI - Preferred dvds}}{\text{Total common equity}}$;
\\

5. Market value ratios: 
\textbf{P/E} = $\frac{\text{Price}}{\text{EPS}}$;
\textbf{M/B} = $\frac{\text{Market share price}}{\text{Book share price}}$;
\\

\textbf{Dupont Identity}: ROE = Profit margin * TA turnover * Equity multiplier
= ROA * EM;

\subsection{Lecture 3: Time Value of Money}

\subsubsection{Annuities & Perpetuities}

\textbf{Ordinary annuities} have cash flows at the end of each period.
\textbf{Annuity dues} have cash flows at the beginning of each period.
\textbf{Perpetuities} are a set of payments paid forever, growing perpetuities
grow at a constant rate each period. 
\\

\textbf{Simple Interest}: FV = PV$(1 + rt)$;
\textbf{Compound Interest}: FV = PV$(1 + r)^t$,
$r = (\frac{\text{FV}}{\text{PV}})^{1/t} - 1$,
$t = \ln(\frac{\text{FV}}{\text{PV}}) / \ln(1+r)$;
% TODO: derive this
\textbf{PV(Annuity)} = PMT * $\frac{1 - (1/(1 + r)^t)}{r}$;
\textbf{PV(Annuity due)} = PV(Annuity) * (1+r);
\textbf{FV(Annuity)} = PMT * $\frac{(1+r)^n - 1}{r}$;
\textbf{FV(Growing annuity)} = PMT1 * $\frac{(1+r)^n - (1+g)^n}{r-g}$;
\textbf{FV(Annuity due)} = FV(Annuity) * (1+r);
\textbf{PV(Perpetuity)} = $\frac{\text{PMT}}{r}$;
\textbf{PV(Growing perpetuity)} = $\frac{\text{C}_1}{r-g}$;
\textbf{EAR} = $\left(1 + \frac{\text{APR}}{m}\right)^m -1$;

\subsubsection{Different Types of Loans}

\textbf{Loan with fixed principal payment}: Pays fixed amount of principal per
year, interest paid is the rate multiplied by the beginning balance. \\
\textbf{Amortized loan}: Each payment first covers the interest on the
beginning balance and then is contributed to repaying the principal.

\subsection{Lecture 4 \& 5: Risk \& Return}

\textbf{\% Return} = Dvd yield + Capital gain yield = 
$\frac{\text{Dvd}}{\text{Initial share price}} + \frac{\text{Capital gain}}
{\text{Initial share price}}$;
\textbf{Real return} = $\frac{\text{1 + nominal}}{\text{1 + inflation}}-1$;
\textbf{Expected return}: $\hat{r} = \sum_i r_i P_i$;
Arithmetic mean of returns = $\frac{1}{N}\sum_i^N r_i$;
Geometric mean of returns = $[\prod_i^N (1+r_i)]^{1/n} - 1$;
When given probabilites, $\sigma = \sqrt{\sum_i (r_i - \hat{r})^2 P_i}$;
When given data, $\sigma = \sqrt{\frac{\sum_i (r_i - \hat{r})^2}{n-1}}$;
\textbf{Coeffficient of variation (CV)} = $\frac{\sigma}{\hat{r}}$;
\textbf{Risk premium} = Excess return over risk-free rate = $r - r_{rf}$;
% TODO: revisit
\textbf{Real risk premium} = $\text{Re}(r) - \text{Re}(r_{rf}) =
\frac{\text{nominal RP}}{\text{1+inflation}}$;
\textbf{Expected portfolio return}: $\hat{r_p} = \sum_i w_i \hat{r_i} = \sum_i
P_i (r_p)_i$;
$\sigma_p = \sum_i ((r_p)_i - \hat{r_p})P_i$;
\textbf{Sample Cov(X,Y)} = $\frac{\sum_i (r_x - \hat{r_x})(r_y - \hat{r_y})}{n-1}$;
Cov(X,Y) = $\rho_{XY} \sigma_X \sigma_Y$;
$\sigma(ax+by) = \sqrt{a^2 \sigma(x)^2 + b^2\sigma(y)^2 +
	2ab\sigma(x)\sigma(y)\rho_{xy}}$;
\textbf{Total risk} = Coy-specific risk + Market risk, and is measured by $\sigma$.
\textbf{Market risk} $\beta_i = \frac{\text{Cov}(r_i, r_M)}{\sigma_M^2}
= \frac{\rho_{iM} \sigma_i \sigma_M}{\sigma_M^2}
= \frac{\sigma_i}{\sigma_M}\rho_{iM}$;
\textbf{Required rate of return} (Security Market Line): $r = r_{rf} + RP_i =
	r_{rf} + (r_M - r_{rf})\beta_i$;
\textbf{Reward-to-risk ratio} = Slope of SML $= r_m - r_f$;
Portfolio beta: $\beta_p = \sum w_j \beta_j$;
Efficient Frontier extends towards the positive return direction from the
	minimum variance portfolio.
\textbf{Capital Market Line}: $r_p = r_f + \frac{r_m - r_f}{\sigma_m} \sigma_p$;

% You can even have references
%\rule{0.3\linewidth}{0.25pt}
%\scriptsize
%\bibliographystyle{abstract}
%\bibliography{refFile}

\end{multicols}
\end{document}
