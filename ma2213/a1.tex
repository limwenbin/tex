\documentclass[12pt]{article}
\usepackage[margin=1in]{geometry} 
\usepackage{mathtools,amsmath,amsthm,amssymb,amsfonts}

\newcommand*\diff{\mathop{}\!\mathrm{d}}
\newcommand*\Diff[1]{\mathop{}\!\mathrm{d^#1}} 
\newcommand{\N}{\mathbb{N}}
\newcommand{\Z}{\mathbb{Z}}
 
\newenvironment{problem}[2][Problem]{\begin{trivlist}
\item[\hskip \labelsep {\bfseries #1}\hskip \labelsep {\bfseries #2.}]}{\end{trivlist}}
 
\begin{document}

\title{MA2213 Assignment 1}
\author{Lim Wen Bin \\
A0140764H\\
T05}
\maketitle

\begin{problem}{1}
\end{problem}
\begin{proof}
\begin{gather*}
\shortintertext{Given}
	a \le x_0 < \ldots < x_k \le b,\ f(x_0) = \ldots = f(x_k) = 0
	\shortintertext{and that}
	f(x), \ldots, f^{(k)} (x)
	\intertext{are all continuous on $[a,b]$, Let $p_i$ denote the following
		proposition for $i \in \{1, \ldots, k\}$.}
	\exists \text{ pairwise distinct } \xi_{i,j} \in [a,b] \text{ for } j \in
		\{0, \ldots, k-i\} \\
	\text{ s.t. } f^{(i)} (\xi_{i,j}) = 0
	\intertext{As $f(x)$ continuous on $[x_0, x_1]$, $\ldots$, $[x_{k-1}, x_k]$,
		by the Mean Value Theorem,}
	\exists \xi_{1,0} \in (x_0, x_1), \ldots, \xi_{1,k-1} \in (x_{k-1},x_k) \\
	\text{ s.t. } f^{(1)} (\xi_{1,0}) = \ldots = f^{(1)} (\xi_{1,k-1}) = 0. \\
	\therefore p_1 \text{ is true}
	\intertext{Next, assume $p_n$ is true for some $n \in \{1, \ldots, k-1\}$, i.e.}
	\exists \xi_{n,0} \in (\xi_{n-1,0}, \xi_{n-1,1}),
		\ldots, \xi_{n,k-n} \in (x_{n-1,k-n}, x_{n-1,k-n+1}) \\
	\text{ s.t. } f^{(n)} (\xi_{n,0}) = \ldots = f^{(n)} (\xi_{n,k-n}) = 0. \\
	\intertext{Then, to prove that $p_{n+1}$ is true, by the Mean
		Value Theorem,}
	\exists \xi_{n+1,0} \in (\xi_{n,0}, \xi_{n,1}),
		\ldots, \xi_{n+1,k-n-1} \in (x_{n,k-n-1}, x_{n,k-n}) \\
	\text{ s.t. } f^{(n+1)} (\xi_{n,0}) = \ldots = f^{(n+1)} (\xi_{n,k-n}) = 0. \\
	\intertext{Therefore, by mathematical induction, $p_i$ is true for all $i
	\in \{1, \ldots, k\}$ and }
	\exists \xi = \xi_{k, 0} \in [a,b] \text{ s.t. } f^{(k)} (\xi) = 0.
\end{gather*}
\end{proof}

\begin{problem}{2}
\end{problem}
\begin{gather*}
	h_0 = h_1 = 1,\ Am = b \\
	\left( \begin{array}{ccc}
		2 & 1 & 0 \\
		1 & 4 & 1 \\
		0 & 1 & 2 \\
	\end{array} \right)
	\left( \begin{array}{c}
		M_0 \\
		M_1 \\
		M_2 \\
	\end{array} \right) = 
	\left( \begin{array}{c}
		0 \\
		0 \\
		0 \\
	\end{array} \right)  \\
	% \left( \begin{array}{ccc|c}
		% 2 & 1 & 0 & 0\\
		% 1 & 4 & 1 & 0\\
		% 0 & 1 & 2 & 0\\
	% \end{array} \right)
	% \rightarrow
	% \left( \begin{array}{ccc|c}
		% 2 & 1 & 0 & 0\\
		% 0 & 7/2 & 1 & 0\\
		% 0 & 1 & 2 & 0\\
	% \end{array} \right)
	% \rightarrow
	% \left( \begin{array}{ccc|c}
		% 2 & 1 & 0 & 0\\
		% 0 & 7/2 & 1 & 0\\
		% 0 & 0 & 12/7 & 0\\
	% \end{array} \right) \\
	% %
	% \left( \begin{array}{ccc|c}
		% 2 & 1 & 0 & 0\\
		% 0 & 7/2 & 1 & 0\\
		% 0 & 0 & 1 & 0\\
	% \end{array} \right)
	% \rightarrow
	% \left( \begin{array}{ccc|c}
		% 2 & 1 & 0 & 0\\
		% 0 & 7/2 & 0 & 0\\
		% 0 & 0 & 1 & 0\\
	% \end{array} \right)
	% \rightarrow
	% \left( \begin{array}{ccc|c}
		% 2 & 1 & 0 & 0\\
		% 0 & 1 & 0 & 0\\
		% 0 & 0 & 1 & 0\\
	% \end{array} \right) \\
	% %
	% \left( \begin{array}{ccc|c}
		% 2 & 0 & 0 & 0\\
		% 0 & 1 & 0 & 0\\
		% 0 & 0 & 1 & 0\\
	% \end{array} \right)
	% \rightarrow
	% \left( \begin{array}{ccc|c}
		% 1 & 0 & 0 & 0\\
		% 0 & 1 & 0 & 0\\
		% 0 & 0 & 1 & 0\\
	% \end{array} \right),\
	\Rightarrow \left( \begin{array}{c}
		M_0 \\
		M_1 \\
		M_2 \\
	\end{array} \right) = 
	\left( \begin{array}{c}
		0 \\
		0 \\
		0 \\
	\end{array} \right) \\
	\intertext{Then, the cubic splines, given by}
	\begin{aligned}
		s_i(x) &= \frac{1}{6h_i} [(x_{i+1} - x)^3 M_i + (x-x_i)^3 M_{i+1}] \\
		&\phantom{=} -\frac{h_i}{6} [(x_{i+1} - x) M_i + (x-x_i) M_{i+1}] \\
		&\phantom{=} +\frac{1}{h_i} [(x_{i+1} - x) f_i + (x-x_i) f_{i+1}],
	\end{aligned} \\
	\shortintertext{are}
	% TODO: correct mistake
	\begin{aligned}
		s_0(x) &= -x \\
		s_1(x) &= -(2 - x) - 2 (x - 1) \\
		&= -x
	\end{aligned} \\
	\therefore S(x) = -x,\quad 0 \le x < 2
\end{gather*}

\begin{problem}{3}
\end{problem}
\begin{gather*}
	S_0'(1) = S_1'(1) \\
	2 - 3 = b \\
	b = -1 \qed \\
	S_0''(1) = S_1''(1) \\
	-6 = 2c \\
	c = -3 \qed \\
	\intertext{As $S$ is subject to natural boundary conditions,}
	S_1''(2) = 0 \\
	2(-3) + 6d = 0 \\
	d = 1 \qed
\end{gather*}
\filbreak

\begin{problem}{4}
\end{problem}
\begin{gather*}
	S_1(2) = S_0(2) \\
	a = 3 + 2 - 1 = 4 \qed \\
	S_0'(2) = S_1'(2) \\
	3 + 4 - 3 = b \\
	b = 4 \qed \\
	S_0''(2) = S_1''(2) \\
	4 - 6 = 2c \\
	c = -1 \qed \\
	\intertext{As $S$ is subject to clamped boundary conditions,}
	S_1'(3) = f'(3) = f'(1) \\
	4 + 2(-1) + 3d = 3 \\
	d = \frac{1}{3} \qed
\end{gather*}

\begin{problem}{5}
\end{problem}
\begin{gather*}
	S_0(2) = 1 \\
	1 + B - D = 1 \\
	B - D = 0 \tag{1} \\
	S_0'(2) = S_1'(2) \\
	B - 3D = b \\
	B - 3D - b = 0 \tag{2} \\
	S_0''(2) = S_1''(2) \\
	-6 D = -\frac{3}{2} \\
	D = \frac{1}{4} \qed
	\intertext{From (1),}
	B = \frac{1}{4} \qed
	\intertext{From (2),}
	b = \frac{1}{4} - \frac{3}{4} = - \frac{1}{2} \qed \\
	S_1(3) = 0 \\
	1 - \frac{1}{2} - \frac{3}{4} + d = 0 \\
	d = \frac{1}{4} \qed \\
\end{gather*}

\begin{problem}{6}
\end{problem}
\begin{gather*}
	S_0(1) = S_1(1) \\
	1 + B + 2 - 2 = 1 \\
	B = 0 \\
	S_0'(1) = S_1'(1) \\
	4 - 6 = b \\
	b = -2 \\
	f'(0) = S_0'(0) = 0 \qed \\
	\begin{aligned}
		f'(2) &= S_1'(2) \\
		&= -2 - 8 + 21 \\
		&= 11 \qed \\
	\end{aligned}
\end{gather*}

\begin{problem}{7}
\end{problem}
\begin{proof}
\begin{gather*}
	\intertext{Consider the Taylor expansion of $f(x)$ about $x_1$, then,
	$\forall x \in [a, b] = [x_0, x_2]$, $\exists \eta(x) \in (a,b)$ such that}
	f(x) = f(x_1)
		+ f'(x_1) (x-x_1)
		+ \frac{f''}{2} (x_1) (x-x_1)^2
		+ \frac{f^{(3)}}{6} (x_1) (x-x_1)^3
		+ \frac{f^{(4)} (\eta(x))}{24} (x-x_1)^4. \\
	\intertext{Then,}
	\begin{aligned}
		\int_a^b f(x) \diff x &= \int_{x_0}^{x_2} f(x) \diff x \\
		&= f(x_1)(x_2 - x_0) 
			+ \left[
				f'(x_1) (x-x_1)^2
				+ \frac{f''}{2} (x_1) (x-x_1)^3
				+ \frac{f^{(3)}}{6} (x_1) (x-x_1)^4
			\right]_{x_0}^{x_2}
		\\
		&\phantom{=} + \frac{1}{24} \int_{x_0}^{x_2} f^{(4)}(\eta(x)) (x-x_1)^4
		\diff x \\
	\end{aligned}
	\intertext{As $(x-x_1)^4 \ge 0$ for $x \in [x_0, x_2]$, by the integral
	Mean Value Theorem, for $\eta_2 \in (a,b)$,}
	\begin{aligned}
		\int_a^b f(x) \diff x &= 2hf(x_1) + \frac{h^3}{3} f''(x_1)
			+ f^{(4)} (\eta_2) \int_{x_0}^{x_2} (x-x_1)^4 \diff x 
		\\
		&= 2hf(x_1) + \frac{h^3}{3} f''(x_1) + \frac{f^{(4)}(\eta_2)}{60}h^5.
	\end{aligned}
	\intertext{Next, considering the Taylor expansion of $f(x_0)$ and $f(x_2)$
	at $x_1$, for $\xi_1 \in (x_0,x_1)$, $\xi_2 \in (x_1,x_2)$,}
	f(x_0) = f(x_1)
		+ f'(x_1) (x_0-x_1)
		+ \frac{f''}{2} (x_1) (x_0-x_1)^2
		+ \frac{f^{(3)}}{6} (x_1) (x_0-x_1)^3
		+ \frac{f^{(4)} (\xi_1)}{24} (x_0-x_1)^4,
	\\
	f(x_2) = f(x_1)
		+ f'(x_1) (x_2-x_1)
		+ \frac{f''}{2} (x_1) (x_2-x_1)^2
		+ \frac{f^{(3)}}{6} (x_1) (x_2-x_1)^3
		+ \frac{f^{(4)} (\xi_2)}{24} (x_2-x_1)^4.
	\intertext{Simplifying,}
	f(x_0) = f(x_1)
		- f'(x_1) h
		+ \frac{f''}{2} (x_1) h^2
		- \frac{f^{(3)}}{6} (x_1) h^3
		+ \frac{f^{(4)} (\xi_1)}{24} h^4,
	\\
	f(x_2) = f(x_1)
		+ f'(x_1) h
		+ \frac{f''}{2} (x_1) h^2
		+ \frac{f^{(3)}}{6} (x_1) h^3
		+ \frac{f^{(4)} (\xi_2)}{24} h^4,
	\\
	f''(x_1) = \frac{1}{h^2} (f(x_0) -2f(x_1) +f(x_2)) -
	\frac{h^2}{24}(f^{(4)}(\xi_1) + f^{(4)}(\xi_2)).
	\intertext{Furthermore, as}
	\min_{a\le x\le b} f^{(4)}(x) 
		\le
		\frac{f^{(4)}(\xi_1) + f^{(4)}(\xi_2)}{2}
		\le
		\max_{a\le x\le b} f^{(4)}(x),
	\intertext{by the Intermediate Value Theorem, $\exists \eta_1 \in (a,b)$,}
	\frac{f^{(4)}(\xi_1) + f^{(4)}(\xi_2)}{2} = f^{(4)} (\eta_1)
	\intertext{Then,}
	f''(x_1) = \frac{1}{h^2} (f(x_0) -2f(x_1) +f(x_2)) -
		\frac{h^2}{12} f^{(4)}(\eta_1).
	\\
	\begin{aligned}
		\int_{x_0}^{x_2} f(x) \diff x 
		&= 2hf(x_1) + \frac{h^3}{3} f''(x_1) + \frac{f^{(4)}(\eta_2)}{60}h^5
		\\
		&= 2hf(x_1) + 
			\frac{h^3}{3} \left[
				\frac{1}{h^2} (f(x_0) -2f(x_1) +f(x_2)) -
				\frac{h^2}{12} f^{(4)}(\eta_1)
			\right]
			+ \frac{f^{(4)}(\eta_2)}{60}h^5
		\\
		&= \frac{h}{3} f(x_0) 
			+ \frac{4h}{3} f(x_1) 
			+ \frac{h}{3} f(x_2) 
			- \frac{h^5}{36} f^{(4)}(\eta_1)
			+ \frac{f^{(4)}(\eta_2)}{60}h^5
		\\
		&= \frac{h}{3} [
				f(x_0) + 4f(x_1) + f(x_2)
			]
			- 
			\frac{h^5}{12} \left[
				\frac{f^{(4)}(\eta_1)}{3} - \frac{f^{(4)}(\eta_2)}{5}
			\right]
	\end{aligned}
	\\
	\therefore E_2(f)
		= \frac{h^5}{12} \left[
			\frac{f^{(4)}(\eta_1)}{3} - \frac{f^{(4)}(\eta_2)}{5}
		\right]
	\intertext{Furthermore, as $|\eta_1 - \eta_2| \le b - a = 2h$, by
	considering the Taylor expansion of $f^{(4)}(\eta_2)$ at $\eta_1$,}
	f^{(4)}(\eta_2) = f^{(4)}(\eta_1) + \mathbf{O}(\eta_1-\eta_2) =
		f^{(4)}(\eta_1) + \mathbf{O}(h).
	\\
	\therefore E_2(f)
		= \frac{h^5}{12} \left[
			\frac{f^{(4)}(\eta_1)}{3} - \frac{f^{(4)}(\eta_2)}{5}
		\right]
		= -\frac{h^5}{90} f^{(4)}(\eta_1) + \mathbf{O}(h^6)
\end{gather*}
\end{proof}

\end{document}
