\documentclass[12pt]{article}
\usepackage[margin=1in]{geometry} 
\usepackage{mathtools,amsmath,amsthm,amssymb,amsfonts}

\newcommand*\diff{\mathop{}\!\mathrm{d}}
\newcommand*\Diff[1]{\mathop{}\!\mathrm{d^#1}} 
\newcommand{\N}{\mathbb{N}}
\newcommand{\Z}{\mathbb{Z}}
 
\newenvironment{problem}[2][Problem]{\begin{trivlist}
\item[\hskip \labelsep {\bfseries #1}\hskip \labelsep {\bfseries #2.}]}{\end{trivlist}}
 
\begin{document}

\title{PC2134 Assignment 9}
\author{Lim Wen Bin \\
A0140764H\\
Group 1}
\maketitle

\begin{problem}{1.a}
\end{problem}
\begin{gather*}
	(1-x^2) y'' - 2x y' + l(l+1)y = 0 \\
	y'' + \frac{2x}{x^2-1} y' + \frac{-l(l+1)}{x^2-1} y  = 0 \\
	\intertext{when $x=0$,}
	P(0) = \frac{2(0)}{1-0^2} = 0 \\
	Q(0) = \frac{l(l+1)}{1-0^2} = l(l+1) \\
	\therefore x = 0 \text{ is a ordinary point.}
	\intertext{when $x=\pm1$,}
	\lim_{x\to\pm1} P(x) = Q(x) = \infty \\
	P(\pm1) 
		= (x\mp1) \frac{2x}{(x+1)(x-1)}
		= \frac{\pm 2}{\pm 2}
		= 2
	\\
	Q(\pm1) 
		= (x\mp1) \frac{l(l+1)}{(x+1)(x-1)}
		= \frac{l(l+1)}{\pm 2}
	\\
	\therefore x = \pm1 \text{ are regular singular points.}
	\intertext{Let $u = \frac{1}{x}$, then,}
	\frac{dy}{dx} 
		= \frac{dy}{du} \frac{du}{dx}
		= -\frac{1}{x^2} \frac{dy}{du}
		= -u^2 \frac{dy}{du}
	\\
	\frac{d^2y}{dx^2} 
		= \frac{du}{dx} \frac{d}{du} \left( -u^2 \frac{dy}{du} \right)
		= -u^2 \left( -2u \frac{dy}{du} - u^2 \frac{d^2y}{du^2} \right)
		= 2u^3 \frac{dy}{du} + u^4 \frac{d^2y}{du^2}
	\\
\end{gather*}
\filbreak
\begin{gather*}
	2u^3 \frac{dy}{du} + u^4 \frac{d^2y}{du^2}
		- \frac{2u^{-1}}{u^{-2}-1} \left( u^2 \frac{dy}{du} \right)
		- \frac{l(l+1)}{u^{-2}-1} y(u)  = 0
	\\
	u^4 \frac{d^2y}{du^2}
		+ \left(
			2u^3 - \frac{2u^{3}}{1-u^{2}}
		\right) \frac{dy}{du}
		- \frac{u^2l(l+1)}{1-u^{2}} y(u)  = 0
	\\
	u^4 \frac{d^2y}{du^2}
		+ \frac{2u^{5}}{1-u^{2}} \frac{dy}{du}
		- \frac{u^2l(l+1)}{1-u^{2}} y(u)  = 0
	\\
	\frac{d^2y}{du^2}
		+ \frac{2u}{1-u^{2}} \frac{dy}{du}
		- \frac{l(l+1)}{u^2(1-u^{2})} y(u) = 0
	\\
	\intertext{At $x = +-\infty$, i.e. $u = 0$,} 
	P(u=0) = 0,\ \lim_{u\to0} Q(u) = \infty \\
	u^2 Q(u) \bigl|_{u=0} = l(l+1) \\
	\therefore x = \pm\infty \text{ are regular singular points.}
\end{gather*}

\begin{problem}{1.b}
\end{problem}
\begin{gather*}
	y'' + \frac{2x}{x^2-1} y' + \frac{-l(l+1)}{x^2-1} y  = 0 \\
	y'' + \frac{s(x)}{x} y' + \frac{t(x)}{x^2} y  = 0,\quad 
		y = x^\sigma \sum_{n=1}^\infty a_n x^n
	\\
	\text{where } s(x) = \frac{2x^2}{x^2-1},\quad 
		t(x) = -\frac{x^2l(l+1)}{x^2-1} 
	\\
	\intertext{The indicial equation is given by}
	[\sigma (\sigma-1) + s(0)\sigma + t(0)] a_0 = 0. \\
	\sigma (\sigma-1) = 0 \\
	\sigma = 0,\quad \sigma = 1 \\
	\intertext{For $\sigma = 1,}
	\sum_{n=0}^\infty [
		(n + \sigma) (n + \sigma - 1)
		+ s(x) (n + \sigma)
		+ t(x)
	] a_n x^n = 0 \\
	\sum_{n=0}^\infty \left[
		n(n + 1)
		+ \frac{2x^2}{x^2-1} (n + 1)
		-\frac{x^2l(l+1)}{x^2-1} 	
	\right] a_n x^n = 0 \\
	\sum_{n=0}^\infty \left[
		n(n + 1)
		+ \frac{x^2[2n+2 - l(l+1)]}{x^2-1} 	
	\right] a_n x^n = 0 \\
	\sum_{n=0}^\infty \left[
		n(n + 1) (x^2-1) 
		+ x^2[2n+2 - l(l+1)]
	\right] a_n x^n = 0 \\
	\sum_{n=0}^\infty \{
		x^2[(n+1)(n+2) - l(l+1)]
		- n(n + 1)
	\} a_n x^n = 0 \\
\end{gather*}
\filbreak
\begin{gather*}
	x^1:\quad -1(1+1)a_1 = 0 \Rightarrow a_1 = 0 \\
	n\ge2,\ x^n:\quad 
		[
			n(n-1) - l(l+1)
		] a_{n-2}
		-n(n+1) a_n = 0 
	\\
	a_n = 
		\frac{ n(n-1) - l(l+1) }
		{ n(n+1) } a_{n-2}
		= 
		\frac{ (n+l)(n-l-1) }
		{ n(n+1) } a_{n-2}
	\\
	\begin{aligned}
		y_1 &= x^1 \sum_{n=0}^\infty a_n x^n \\
		&= a_0 x \left( 
			1 
			+ \frac{(2+l)(1-l)}
			{ 3! } x^2
			+ \frac{(4+l)(3-l)(2+l)(1-l)}
			{ 5! } x^4
			+ \ldots
		\right)
		\\
		&= a_0 \left( 
			x
			+ \frac{(2+l)(1-l)}
			{ 3! } x^3
			+ \frac{(4+l)(3-l)(2+l)(1-l)}
			{ 5! } x^5
			+ \ldots
		\right)
	\end{aligned}
	\intertext{For $\sigma = 0,}
	\sum_{n=0}^\infty \left[
		n(n - 1)
		+ \frac{2x^2}{x^2-1} n
		-\frac{x^2l(l+1)}{x^2-1} 	
	\right] a_n x^n = 0 \\
	\sum_{n=0}^\infty \left[
		n(n - 1) (x^2-1)
		+ x^2[2n - l(l+1)]
	\right] a_n x^n = 0 \\
	\sum_{n=0}^\infty \{
		x^2[n(n+1) - l(l+1)]
		- n(n - 1)
	\} a_n x^n = 0 \\
	x^1:\quad -1(1-1)a_1 = 0 \Rightarrow \text{$a_1$ is arbitrary, set $a_1 = 0$.} \\
	n\ge2,\ x^n:\quad 
		[
			n(n+1) - l(l+1)
		] a_{n-2}
		-n(n-1) a_n = 0 
	\\
	a_n = 
		\frac{ (n-1)(n-2) - l(l+1) }
		{ n(n-1) } a_{n-2}
		= 
		\frac{ (n+l-1)(n-l-2) }
		{ n(n-1) } a_{n-2}
	\\
	\begin{aligned}
	y_2 &= a_0 \left( 
			1
			+ \frac{(1+l)(1-l)}
			{ 2! } x^2
			+ \frac{(3+l)(2-l)(1+l)(-1-l)}
			{ 4! } x^4
			+ \ldots
		\right)
	\end{aligned}
\end{gather*}

\begin{problem}{1.c}
\end{problem}
Both recurrence relations are of the form
\begin{gather*}
	a_n = 
		\frac{ (n+a)(n+b) }
		{ (n+c)(n+d) } a_{n-2}
	\intertext{for finite $a, b, c, d$. Then, the interval of convergence of the
	power series is given by}
	|x-0| < \lim_{n\to\infty} \left| \frac{a_{n-2}}{a_n} \right|
		= 
		\lim_{n\to\infty} \left| 
			\frac{ (n+a)(n+b) }
			{ (n+c)(n+d) }
		\right|
		= 1.
	\intertext{$\therefore$ the solutions diverge for $x=\pm1$ if the series
	continue to infinity.}
\end{gather*}
\filbreak

\begin{problem}{1.d}
\end{problem}
\begin{gather*}
	a_n = 
		\frac{ (n+l)(n-l-1) }
		{ n(n+1) } a_{n-2}
	,\quad
	a_n' = 
		\frac{ (n+l-1)(n-l-2) }
		{ n(n-1) } a_{n-2}'
	\\
	\begin{aligned}
		y_1 &= a_0 \left( 
			x
			+ \frac{(2+l)(1-l)}
			{ 3! } x^3
			+ \frac{(4+l)(3-l)(2+l)(1-l)}
			{ 5! } x^5
			+ \ldots
		\right)
		\\
		y_2 &= a_0' \left( 
			1
			+ \frac{(1+l)(-l)}
			{ 2! } x^2
			+ \frac{(3+l)(2-l)(1+l)(-1-l)}
			{ 4! } x^4
			+ \ldots
		\right)
	\end{aligned}
	\intertext{By setting $l = 2k + 1, k \in \Z^+ \cup \{0\}$,}
	y_1 = \sum_{m=0}^k a_{2m+1} x^{2m+1} \\
	\intertext{is a polynomial of degree $l$ while $y_2$ retains the behaviour
	of divergence at $x=\pm1$. Meanwhile, by setting $l = 2k, k \in \Z^+ \cup
	\{0\}$,}
	y_2 = \sum_{m=0}^k a_{2m} x^{2m} \\
	\intertext{is a polynomial of degree $l$ while $y_1$ retains the behaviour
	of divergence at $x=\pm1$.}
\end{gather*}

\begin{problem}{2}
\end{problem}
First, to find the general solution of the associated homogenous equation, solve
\begin{gather*}
	x y'' - (1 + x) y' + y = 0. \\
	y'' - \frac{(1 + x)}{x} y' + \frac{1}{x} y = 0 \\
	y'' + \frac{s(x)}{x} y' + \frac{t(x)}{x^2} y  = 0,\quad 
		y = x^\sigma \sum_{n=1}^\infty a_n x^n
	\\
	\text{where } s(x) = -(1+x),\quad 
		t(x) = x.
	\\
	\intertext{The indicial equation is given by}
	[\sigma (\sigma-1) + s(0)\sigma + t(0)] a_0 = 0. \\
	\sigma (\sigma-1) -\sigma = 0 \\
	\sigma (\sigma-2) = 0 \\
	\sigma = 0,\quad \sigma = 2 \\
\end{gather*}
\filbreak
\begin{gather*}
	\intertext{For $\sigma = 2$,}
	\sum_{n=0}^\infty [
		(n + \sigma) (n + \sigma - 1)
		+ s(x) (n + \sigma)
		+ t(x)
	] a_n x^n = 0. \\
	\sum_{n=0}^\infty [
		(n + 2) (n + 1)
		- (1 + x) (n + 2)
		+ x
	] a_n x^n = 0 \\
	\sum_{n=0}^\infty [
		(n + 2) (n + 1)
		- (n + 2)
		+ x (1 - n - 2)
	] a_n x^n = 0 \\
	\sum_{n=0}^\infty [
		n(n + 2)
		- x (n + 1)
	] a_n x^n = 0 \\
	n\ge1,\ x^n:\quad 
		n(n+2) a_n
		- n a_{n-1}
		= 0 
	\\
	a_n = 
		(n+2)^{-1} a_{n-1}
	\\
	\begin{aligned}
		y_1 &= x^2 \sum_{n=0}^\infty a_n x^n \\
		&= a_0 x^2 \left( 
			1 
			+ \frac{2}{3!} x
			+ \frac{2}{4!} x^2
			+ \ldots
		\right) \\
		&= a_0 \left( 
			x^2
			+ \frac{2}{3!} x^3
			+ \frac{2}{4!} x^4
			+ \ldots
		\right)
	\end{aligned}
	\intertext{Setting $a_0 = \frac{1}{2}$ yields}
	y_1 = e^x - x - 1. \\
	\intertext{For $\sigma = 0$,}
	\sum_{n=0}^\infty [
		n(n - 1)
		- n(1 + x)
		+ x
	] a_n x^n = 0. \\
	\sum_{n=0}^\infty [
		n(n - 2)
		- x(n - 1)
	] a_n x^n = 0 \\
	n\ge1,\ x^n:\quad 
		n(n-2) a_n
		- (n-2) a_{n-1}
		= 0 
	\\
	a_n = 
		n^{-1} a_{n-1}
	\\
	\begin{aligned}
		y_2 &= \sum_{n=0}^\infty a_n x^n \\
		&= a_0 \left( 
			1 
			+ \frac{1}{2!} x
			+ \frac{1}{3!} x^2
			+ \ldots
		\right) \\
	\end{aligned}
	\intertext{Setting $a_0 = 1$ yields}
	y_2 = e^x. \\
	\therefore Y_c = c_1 (-x-1) + c_2 e^x
\end{gather*}
\filbreak
\begin{gather*}
	\intertext{To find the particular integral,}
		Y_p = - \int^x \frac{
			f(u)
			[
				y_2(u) y_1(x) - y_1(u) y_2(x) 
			]}
			{
				y_1(u) y_2'(u) - y_2(u) y_1'(u) 
			}
		\diff u \\
		\intertext{redefining $y_1 = -x-1$,}
	\begin{aligned}
		Y_p &= - \int^x \frac{
			u
			[
				e^u (-x-1) - (-u-1) e^x
			]}
			{
				(-u-1) e^u - e^u (-1)
			}
		\diff u \\
		&= \int^x
			e^{-u}
			[
				e^u (-x-1)
				-
				(-u-1) e^x 
			]
		\diff u \\
		&= (-x-1) \int^x
			\diff u
			+
			e^x \int^x
				(e^{-u}u + e^{-u})
			\diff u
		\\
		&= 
			- x^2 - x
			- x - 2
		\\
		&= 
			- x^2 - 2x - 2
		\\
	\end{aligned}
	\intertext{Hence, the general solution of}
	x y'' - (1 + x) y' + y = x^2
	\intertext{is given by}
	y = Y_c + Y_p
		= c_1 e^x + c_2 (-x-1)
		- x^2 - 2x - 2
\end{gather*}

\end{document}
