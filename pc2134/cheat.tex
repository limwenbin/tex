\documentclass[10pt,landscape]{article}
\usepackage{multicol}
\usepackage{calc}
\usepackage{ifthen}
\usepackage[landscape]{geometry}
\usepackage{mathtools,amsmath,amsthm,amsfonts,amssymb}
\usepackage{color,graphicx,overpic}
\usepackage{hyperref}
\usepackage{esint} % cyclic integrals

\pdfinfo{
  /Title (example.pdf)
  /Creator (TeX)
  /Producer (pdfTeX 1.40.0)
  /Author (Seamus)
  /Subject (Example)
  /Keywords (pdflatex, latex,pdftex,tex)}

% This sets page margins to .5 inch if using letter paper, and to 1cm
% if using A4 paper. (This probably isn't strictly necessary.)
% If using another size paper, use default 1cm margins.
\ifthenelse{\lengthtest { \paperwidth = 11in}}
    { \geometry{top=.5in,left=.5in,right=.5in,bottom=.5in} }
    {\ifthenelse{ \lengthtest{ \paperwidth = 297mm}}
        {\geometry{top=1cm,left=1cm,right=1cm,bottom=1cm} }
        {\geometry{top=1cm,left=1cm,right=1cm,bottom=1cm} }
    }

% Turn off header and footer
\pagestyle{empty}

% Redefine section commands to use less space
\makeatletter
\renewcommand{\section}{\@startsection{section}{1}{0mm}%
                                {-1ex plus -.5ex minus -.2ex}%
                                {0.5ex plus .2ex}%x
                                {\normalfont\large\bfseries}}
\renewcommand{\subsection}{\@startsection{subsection}{2}{0mm}%
                                {-1explus -.5ex minus -.2ex}%
                                {0.5ex plus .2ex}%
                                {\normalfont\normalsize\bfseries}}
\renewcommand{\subsubsection}{\@startsection{subsubsection}{3}{0mm}%
                                {-1ex plus -.5ex minus -.2ex}%
                                {1ex plus .2ex}%
                                {\normalfont\small\bfseries}}
\makeatother

% Define BibTeX command
\def\BibTeX{{\rm B\kern-.05em{\sc i\kern-.025em b}\kern-.08em
    T\kern-.1667em\lower.7ex\hbox{E}\kern-.125emX}}

% Don't print section numbers
\setcounter{secnumdepth}{0}


\setlength{\parindent}{0pt}
\setlength{\parskip}{0pt plus 0.5ex}

%My Environments
\newtheorem{example}[section]{Example}
% -----------------------------------------------------------------------

\begin{document}
\raggedright
\footnotesize
\begin{multicols}{3}


% multicol parameters
% These lengths are set only within the two main columns
%\setlength{\columnseprule}{0.25pt}
\setlength{\premulticols}{1pt}
\setlength{\postmulticols}{1pt}
\setlength{\multicolsep}{1pt}
\setlength{\columnsep}{2pt}

\begin{center}
     \Large{\underline{PC2134 Mathematical Mtds}} \\
     \Large{\underline{of Physics I}} \\
\end{center}

Lim Wen Bin A0140764H

\subsection{Lecture 1: Revision of Basic Mathematics}

\subsubsection{Trigonometric identities}
$\sin^2 x + \cos^2 x = 1$;
$\tan^2 x + 1 = \sec^2 x$;
$\sin2x = 2\sin x\cos x$;
$\cos2x = 2\cos^2x-1 = 1 - 2\sin^2x$;
$\sin^2(x) = \frac{1-\cos(2x)}{2}$;
$\cos^2(x) = \frac{1+\cos(2x)}{2}$;
$\sin(x\pm y) = \sin x\cos y \pm \cos x\sin y$;
$\cos(x\pm y) = \cos x\cos y \mp \sin x\sin y$;
$\tan(x\pm y) = (\tan x \pm \tan y) / (1 \mp \tan x\tan y)$;
$\sin P + \sin Q = 2\sin\frac{1}{2}(P+Q) \cos\frac{1}{2}(P-Q)$;
$\sin P - \sin Q = 2\cos\frac{1}{2}(P+Q) \sin\frac{1}{2}(P-Q)$;
$\cos P + \cos Q = 2\cos\frac{1}{2}(P+Q) \cos\frac{1}{2}(P-Q)$;
$\cos P - \cos Q = -2\sin\frac{1}{2}(P+Q) \sin\frac{1}{2}(P-Q)$;

\subsubsection{Limits}
$\lim_{x\to a} [f(x)]^{1/n} = \left[ \lim_{x\to a} f(x) \right]^{1/n}$;

\subsubsection{Intermediate value theorem}
$f$ continuous on $[a,b] \land f(a)f(b)<0 \Rightarrow \exists x \in (a,b) : f(x)=0$.
L'Hopital's Rule: for $\frac{0}{0}$ or $\frac{\infty}{\infty}$, for $c=\infty$ or $c=a^+/a^-$
	$\lim_{x\to c} \frac{f(x)}{g(x)} = \lim_{x\to c} \frac{f'(x)}{g'(x)}$;
Leibniz's Theorem: $\frac{d^n}{dx^n} [f(x)g(x)] = \sum_{r=0}^n \binom{n}{r}
	\left[ \frac{d^r}{dx^r}f(x) \right]  \left[ \frac{d^{n-r}}{dx^{n-r}}g(x) \right]$

\subsubsection{Derivatives}
$f'(x) = \lim_{h\to0} \frac{f(x+h)-f(x)} {h}$;
$\frac{d^n}{dx^n} \sin(x) = \sin(\frac{n\pi}{2} + x)$;
$\frac{d^n}{dx^n} \cos(x) = \cos(\frac{n\pi}{2} + x)$;
$\frac{d \tan(f(x))}{dx} = f'(x)\sec^2(f(x))$;
$\frac{d}{dx} \tan(f(x)) = f'(x)\sec^2(f(x))$;
$\frac{d}{dx} \csc(f(x)) = f'(x)\csc(f(x))\cot(f(x))$;
$\frac{d}{dx} \sec(f(x)) = f'(x)\sec(f(x))\tan(f(x))$;
$\frac{d}{dx} \cot(f(x)) = f'(x)\csc^2(f(x))$;
% hyperbolic trigo
$\frac{d}{dx} \cosh(f(x)) = f'(x)\sinh(f(x))$;
$\frac{d}{dx} \tanh(f(x)) = f'(x)\text{sech}^2(f(x))$;
$\frac{d}{dx} \text{csch}(f(x)) = -f'(x)\text{csch}(f(x))\coth(f(x))$;
$\frac{d}{dx} \text{sech}(f(x)) = -f'(x)\text{sech}(f(x))\tanh(f(x))$;
$\frac{d}{dx} \coth(f(x)) = -f'(x)\text{csch}^2(f(x))$;
% inverse trigo
$\frac{d}{dx} \sin^{-1}(f(x)) = \frac{f'(x)} {\sqrt{1-(f(x))^2}}$;
$\frac{d}{dx} \cos^{-1}(f(x)) = -\frac{f'(x)} {\sqrt{1-(f(x))^2}}$;
$\frac{d}{dx} \tan^{-1}(f(x)) = \frac{f'(x)} {1+(f(x))^2}$;
$\frac{d}{dx} (f(x))^{g(x)} = (f(x))^{g(x)} \frac{d}{dx} (g(x) \ln f(x))$;
Quotient Rule: $\frac{d}{dx} \frac{f(x)}{g(x)} = \frac{f'(x)g(x)-g'(x)f(x)}{(g(x))^2}$;

\subsubsection{Integrals}
FTC: $\int_{a}^{b} F'(x) dx = F(b) - F(a)$, $\frac{d}{dx}\int_a^x f(u)du = f(x)$;
$\int \frac{1} {a^2+(x+b)^2} dx = \frac{1}{a} \tan^{-1} \left( \frac{x+b}{a} \right)$;
$\int \frac{1} {\sqrt{a^2+(x+b)^2}} dx = \sin^{-1} \left( \frac{x+b}{a} \right)$;
$\int -\frac{1} {\sqrt{a^2+(x+b)^2}} dx = \cos^{-1} \left( \frac{x+b}{a} \right)$;
$\int \tan x dx = -\ln|\cos x|$; 
$\int \frac{1} {a^2-(x+b)^2} dx = \frac{1}{2a} \ln \left| \frac{x+b+a}{x+b-a} \right|$;
$\int \cot x dx = \ln|\cos x|$; 
$\int \frac{1} {(x+b)^2-a^2} dx = \frac{1}{2a} \ln \left| \frac{x+b-a}{x+b+a} \right|$; 
$\int \sec x dx = \ln|\sec x + \tan x|$; 
$\int \frac{1} {(x+b)^2-a^2} dx = \frac{1}{2a} \ln \left| \frac{x+b-a}{x+b+a} \right|$; 
$\int \csc x dx = \ln|\csc x + \cot x|$; 
\\
Substitution I: $u=g(x), \int f(g(x))g'(x) dx = \int f(u)du$;
Substitution II: $x=g(t), \int f(x)dx = \int f(g(t))g'(t)dt$;
\textbf{Integration by parts}: $\int fg' dx = fg - \int f'g dx$;
Arc length: $\int_a^b \sqrt{1+f'(x)^2}dx$;
Surface area: $\int_a^b 2\pi f(x) \sqrt{1+f'(x)^2}dx$;
Washer method: $\int_a^b \pi f(x)^2 dx$;
Cylindrical shell method ($y$-axis): $\int_a^b 2\pi xf(x) dx$;
Integrating powers of $\sin$ and $\cos$: $\cos x = \frac{e^{ix} + e^{-ix}}{2}$, 
$\sin x = \frac{e^{ix} - e^{-ix}}{2}$; 
Weierstrass substitution: $x = \tan(\theta/2)$, $\sin\theta = \frac{2x}{1+x^2}$,
	$\cos\theta = \frac{1-x^2}{1+x^2}$, $\tan\theta = \frac{2x}{1-x^2}$,
	$d\theta = \frac{2}{1+x^2}dx$,

\subsubsection{Matrices}
$(AB)_{ij} = \sum_{k=1}^R A_{ik} B_{kj}$; 
$\text{Tr}(A) = \text{Tr}(A^T)$;
$\text{Tr}(ABC) = \text{Tr}(BCA)$;
$\text{det}(A) = \sum_{j=1}^N (-1)^{i+j} A_{ij} M_{ij}$;
$\text{det}(A) = \text{det}(A^T)$;
$\text{det}(A^{-1}) = 1/\text{det}(A)$;
Cramer's Rule: $x_i = \frac{|\Delta_i|}{|A|}$;
Characteristic equation: $|A-\lambda I| = 0$;
$\sum_{i=1}^N \lambda_i = \text{Tr}(A)$;
$0 \not\in \Lambda \Rightarrow \exists A^{-1}$;

\subsection{Lecture 2: Infinite Series}
An infinite series is convergent if $S = \lim_{N\to\infty} S_N = \sum_{n=1}^\infty a_n$.
Arithmetic series: $S_N = \sum_{n=1}^N [a + (n-1)d] = \frac{N}{2} [2a+(N-1)d]$;
Geometric series: $S_N = \sum_{n=1}^N ar^{n-1} = \frac{a(1-r^N)}{1-r}$,
when $|r| < 1$, $S = \frac{a}{1-r}$;
Harmonic series is divergent: $\sum_{n=1}^\infty \frac{1}{n}$;
$\sum |a_n|$ converges $\Rightarrow$ absolute convergence $\Rightarrow$ terms
	can be reordered;
$\sum |a_n|$ diverges but $\sum a_n$ converges $\Rightarrow$ conditional convergence;

\subsubsection{Convegence tests}
\textbf{Preliminary test}: $\lim_{n\to\infty} \ne 0 \Rightarrow$ divergence;
\textbf{Comparison test}: $(u_i) \text{ converges } \land \forall n > N, 0 \le
	a_n \le u_n
\Rightarrow \sum_n a_n$ converges.
Conversely, $(u_i) \text{ diverges } \land \forall n > N, 0 \le u_n \le a_n
\Rightarrow \sum_n a_n$ diverges.
\textbf{Ratio test}: For $p = \lim_{n\to\infty} \left| \frac{a_{n+1}}{a_n} \right|, 
p < 1 \Rightarrow \text{convergence},\
p > 1 \Rightarrow \text{divergence},\
p = 1 \text{ is inconclusive};$
\textbf{Integral test}: If $0 < a_{n+1} < a_n$ for $n > N$, series is convergent if 
$\int^\infty f(x)dx$ is finite.

\subsubsection{Power series}
$f(x) = \sum_{n=0}^\infty a_n (x - x_0)^n$;
Interval of convergence: $|x-x_0| < \lim_{n\to\infty} \left| 
	\frac{a_{n+1}}{a_n} \right|$, convergence on endpoints have to be considered 
	seperately. 
\textbf{Operations with power series}:
1. $f(x)$ and $g(x)$ converges on $A \Rightarrow f(x)\pm g(x)$ and $f(x)g(x)$
	converges on $A$.
2. $f(x)$ and $g(x)$ converges $\forall x \Rightarrow f[g(x)]$ and $g[f(x)]$
	converges $\forall x$.
3. $f(x)$ converges $\Rightarrow f'(x)$ and $\int f(x)$ converges.
\textbf{Taylor's theorem}: $f(x) = \sum_{n=0}^N \frac{f^{(n)} (x_0)}{n!}
	(x-x_0)^n + R_N(x)$,
$R_N(x) = \frac{f^{(N+1)}(\eta)}{(N+1)!} (x-x_0)^{N+1}$ where $\eta \in [x_0,x]$; \\
\textbf{Taylor's series}: $f(x) = \sum_{n=0}^\infty \frac{f^{(n)} (x_0)}{n!}
	(x-x_0)^n$, $lim_{N\to\infty} |R_N(x)| = 0$;

\subsubsection{Maclaurin series}
$f(x) = \sum_{n=0}^\infty \frac{f^{(n)} (0)}{n!} x^n$;
$\sin x = x - \frac{x^3}{3!} + \frac{x^5}{5!} - \ldots = \sum_{n=0}^\infty 
	\frac{(-1)^n x^{2n+1}}{(2n+1)!}$;
$\cos x = 1 - \frac{x^2}{2!} + \frac{x^4}{4!} - \ldots = \sum_{n=0}^\infty 
	\frac{(-1)^n x^{2n}}{(2n)!}$;
$e^x = 1 + x + \frac{x^2}{2!} + \ldots = \sum_{n=0}^\infty 
	\frac{x^n}{n!}$;
$\ln(1+x) = x - \frac{x^2}{2} + \frac{x^3}{3} - \ldots = \sum_{n=0}^\infty 
	\frac{(-1)^n x^{n+1}}{n+1}$;

\subsubsection{Binomial series}: $(1+x)^N = 1 + Nx + \frac{N(N-1)}{2!} + \ldots
\sum_{n=0}^N \binom{N}{n} x^n$;

\subsection{Lecture 3: Complex Variables}
$z = x + iy = r(\cos\theta + i\sin\theta) = re^{i\theta}$;
$|z| = r = \sqrt{x^2 + y^2}$;
$\text{arg}(z) = \theta = \tan^{-1} (\frac{y}{x})$;
$z^* = x - iy = r(\cos(-\theta) + i\sin(-\theta))$;
$(z^*)^* = z$;
$(z_1 + z_2)^* = z_1^* + z_2^*$;
$(z_1z_2)^* = z_1^* z_2^*$;
$|z|^2 = zz^*$;
Division: $\frac{z_1}{z_2} = \frac{z_1z_2^*}{|z_2|^2}$;

\textbf{de Moivre's theorem}: $(e^{i\theta})^n = cos(n\theta) + i\sin(n\theta)$;
Complex logarithm: $\ln z = \ln r + i(\theta + 2n\pi)$;
Complex power: $z_1^{z_2} = e^{z_2\ln z_1}$;

\subsubsection{Complex trigonometric and hyperbolic functions}
$\sin z = \frac{e^{iz} - e^{-iz}}{2i}$; 
$\cos z = \frac{e^{iz} + e^{-iz}}{2}$; 
$\sinh z = \frac{e^{z} - e^{-z}}{2}$; 
$\cosh z = \frac{e^{z} + e^{-z}}{2}$; 
$\sin(iz) = i\sinh(z)$;
$\sinh(iz) = i\sin(z)$;
$\cos(iz) = \cosh(z)$;
$\cosh(iz) = \cos(z)$;
$\sin^{-1} z = -i\ln(iz + \sqrt{1-z^2}) = \csc^{-1} (\frac{1}{z})$;
$\cos^{-1} z = -i\ln(z + \sqrt{z^2-1}) = \sec^{-1} (\frac{1}{z})$;
$\sinh^{-1} z = \ln(z + \sqrt{z^2+1}) = \text{csch}^{-1} (\frac{1}{z})$;
$\cosh^{-1} z = \ln(z + \sqrt{z^2-1}) = \text{sech}^{-1} (\frac{1}{z})$;

\subsection{Lecture 4: Vector Algebra}
Magnitude: $A^2 = |\mathbf{A}|^2 = \sum_{i=1}^3 A_i^2$;
Direction cosines: $A_i = A\cos\theta_i \Rightarrow \sum_{i=1}^3 \cos^@\theta_i = 1$;
Dot (scalar) product: $\mathbf{A} \cdot \mathbf{B} = AB\cos\theta$;
$C^2 = \mathbf{C} \cdot \mathbf{C} = (\mathbf{A} + \mathbf{B}) \cdot
	(\mathbf{A} + \mathbf{B}) = A^2 + B^2 + 2AB\cos\theta$;
Cross (vector) product: $\mathbf{A} \times \mathbf{B} = AB\sin\theta
	\hat{\mathbf{e}}_n$ where $\hat{\mathbf{e}}_n$ is the unit normal obtained
	by the right-hand rule.
comp$_a b = \frac{a \cdot b}{|a|}$, proj$_a b = \frac{a \cdot b}{a \cdot a} a$;
$\mathbf{A} \times \mathbf{B} = - \mathbf{B} \times \mathbf{A}$;
\textbf{Levi-Civita symbol}: $\hat{\mathbf{e}}_i \times \hat{\mathbf{e}}_j
	= \sum_{k=1}^3 \epsilon_{ijk} \hat{\mathbf{e}}_k$;
$\sum_{k=1}^3 \epsilon_{ijk} \epsilon_{mnk} = \delta_{im} \delta_{jn} - 
	\delta_{in} \delta_{jm}$;
$\sum_{j=1}^3 \sum_{k=1}^3 \epsilon_{mjk} \epsilon_{njk} = 2\delta_{mn}$;
$\sum_{i=1}^3 \sum_{j=1}^3 \sum_{k=1}^3 \epsilon_{ijk} \epsilon_{ijk} = 6$;
$(\mathbf{A} \times \mathbf{B})_k = \sum_{i,j} \epsilon_{ijk} A_i B_i$;
Scalar triple product identity: 
$\mathbf{A} \cdot (\mathbf{B} \times \mathbf{C}) = \mathbf{B} \cdot (\mathbf{C}
	\times \mathbf{A}) = \mathbf{C} \cdot (\mathbf{A} \times \mathbf{B})$;
Vector triple product identity: 
$\mathbf{A} \times (\mathbf{B} \times \mathbf{C}) = \mathbf{B} (\mathbf{A}
	\cdot \mathbf{C}) - \mathbf{C} (\mathbf{A} \cdot \mathbf{B})$;

\subsection{Lecture 5: Partial Derivative}
$\frac{\partial f}{\partial x_i} = \lim_{\Delta x_i \to 0} 
	\frac{f(x_1, \ldots, x_i + \Delta x_i, \ldots, x_N) - 
	f(x_1, \ldots, x_N)}{\Delta x_i} = \left( \frac{\partial f}{\partial x_i}
	\right)_{(x_k), k \ne i}$;
Schwarz' theorem: $f_{x_jx_i} = f_{x_ix_j}$;

\subsubsection{Differentials}
\textbf{Total differential}: $df = \sum_{i=1}^N \frac{\partial f}{\partial x_i} dx_i$;
Reciprocity relation: $dy = 0 \Rightarrow \left( \frac{\partial
	z}{\partial x} \right)_y = \left( \frac{\partial x}{\partial z} \right)_y^{-1}$;
Cyclic relation: $dz = 0 \Rightarrow \left( \frac{\partial x}{\partial
	y} \right)_z \left( \frac{\partial y}{\partial z} \right)_x
	 \left( \frac{\partial z}{\partial x} \right)_y = -1$;
\textbf{Exact differential}: $\forall i, j, i \ne j, f_{x_ix_j} = f_{x_jx_i}$;
$\frac{\partial}{\partial x} \int^t f(x,t) dt = \int^t \frac{\partial
	f(x,t)}{\partial x} dt$;
Change of variables: For $x=x(u,v),\ y=y(u,v)$, $\frac{\partial}{\partial x} = 
	\frac{\partial u}{\partial x} \frac{\partial}{\partial u} + \frac{\partial v} 
	{\partial x} \frac{\partial}{\partial v}$, $\frac{\partial}{\partial y} = 
	\frac{\partial u}{\partial y} \frac{\partial}{\partial u} + \frac{\partial v} 
	{\partial y} \frac{\partial}{\partial v}$;
\\
Leibniz's rule: $\frac{d}{d x} \int_u^v f(x,t) dt = \int_u^v \frac{\partial
	f(x,t)}{\partial x} dt$;
$\frac{d}{d x} \int_{u(x)}^{v(x)} f(x,t) dt = 
	f[x,v(x)] \frac{dv(x)}{dx} - f[x,u(x)] \frac{du(x)}{dx}
	+ \int_{u(x)}^{v(x)} \frac{\partial f(x,t)}{\partial x} dt$;

\textbf{Taylor's theorem for multi-variable function}: 
	$f(x) = \sum_{n=0}^N \frac{1}{n!} \left[ \sum_{i=1}^M (x_i - x_{i_0}) 
	\frac{\partial}{\partial x_i} \right]^n f(x) \biggl|_{x=x_0} + R_N(x)$,
$R_N(x) = \frac{1}{(N+1)!} \left[ \sum_{i=1}^M (x_i - x_{i_0}) 
\frac{\partial}{\partial x_i} \right]^{N+1} f(x) \biggl|_{x=x_0}$;

\subsubsection{Stationary values of multi-variable functions}
Stationary points: $\frac{\partial f}{\partial x_i} = 0,\ \forall i$;
Second order behaviour: $\Delta f = f(\mathbf{x}) - f(\mathbf{x_0}) \approx
	\frac{1}{2!} \sum_{i=1}^N \sum_{j=1}^N \frac{\partial^2 f}{\partial x_i
	\partial x_j} \biggl|_{\mathbf{x} = \mathbf{x_0}} \Delta x_i \Delta x_j$;
Hessian matrix (real and symmetric): $H_ij = \frac{\partial^2 f}{\partial x_i
	\partial x_j}$;
For eigenvalues $\lambda_r$ and ortogonal eigenvectors $e_r$,
	$\mathbf{H} \mathbf{e}_r = \lambda_r \mathbf{e}_r$, where $\{\mathbf{e}_r\}$ is 
	the basis, $\Delta \mathbf{x} = \sum_{r=1}^N a_r \mathbf{e}_r$;
$\Delta f = \frac{1}{2} \Delta \mathbf{x}^T \mathbf{H} \Delta \mathbf{x}
	= \frac{1}{2} \sum_{r=1}^N a_r^2 \lambda_r$;
\textbf{Classification of stationary points}:
1. Minima: $\lambda_r > 0,\ \forall r$ or ($f_{xx}, f_{yy} > 0 \land f_{xy}^2 <
	f_{xx} f_{yy})$
2. Maxima: $\lambda_r > 0,\ \forall r$ or ($f_{xx}, f_{yy} < 0 \land f_{xy}^2 <
	f_{xx} f_{yy})$
3. Saddle point: $\lambda_r$ have mixed signs or ($f_{xx}, f_{yy} < 0 \land
	f_{xy}^2 \ge f_{xx} f_{yy})$
Using \textbf{Lagrange undetermined multiplier} to find the stationary points
	of $f(x,\ldots,x_N)$ subject to $K<N$ contraints, $g_j(x_1,\ldots,x_N) = 0$,
	$F(x_1,\ldots,s_N,\lambda_1,\ldots,\lambda_K) = f + \sum_{j=1}^K \lambda_j
	g_j$.
The stationary points are found by solving $dF = 0 \Rightarrow \frac{\partial f}
	{\partial x_i} + \sum_{j=1}^K \lambda_j \frac{\partial g_j}{\partial x_i} = 0$
	and $g_j = 0$.

\subsection{Lecture 6: Multiple Integrals}
$\int_a^b \int_c^d g(x)h(y) dydx = \int_a^b g(x) dx \int_c^d h(y) dy$;
\textbf{Jacobian}: $J = \frac{\partial(x,y)}{\partial(u,v)} = 
	\frac{\partial x}{\partial u} \frac{\partial y}{\partial v}
	- \frac{\partial x}{\partial v} \frac{\partial y}{\partial u}$;
$\iint_\mathcal{R} f(x,y) dxdy = \iint_{R'} g(u,v) \left|
	\frac{\partial(x,y)}{\partial(u,v)} \right| du dv$;
$\mathbf{J}_{xz} = \mathbf{J}_{xy}\mathbf{J}_{yz}$;
$|\mathbf{J}_{xy}| = |\mathbf{J}_{yx}|^{-1}$;
Polar coordinates: $dA = r\ dr d\theta$;
Spherical coordinates: $dV = r^2 \sin\theta \ dr d\theta$;
Elliptical coordinates: $dV = abcr^2 \sin\theta \ dr d\theta$;
Cylindrical coordinates: $dV = r\ dr d\theta$;

\subsection{Lecture 7: Grad, Div and Curl}
$\frac{d}{du} \phi \mathbf{a} = \phi \frac{d\mathbf{a}}{du} + \frac{d\phi}{du} 
	\mathbf{a}$;
$\frac{d}{du} \mathbf{a} \cdot \mathbf{b} = \mathbf{a} \cdot
	\frac{d\mathbf{b}}{du} + \frac{d\mathbf{a}}{du} \cdot \mathbf{b}$;
$\frac{d}{du} (\mathbf{a} \times \mathbf{b}) = \mathbf{a} \times
	\frac{d\mathbf{b}}{du} + \frac{d\mathbf{a}}{du} \times \mathbf{b}$;
Integration of vectors: $\int \mathbf{a} du = \mathbf{A}(u) + \mathbf{b}$, 
	$\int_{u_1}^{u_2} \mathbf{a}(u) = \mathbf{A}(u_2) - \mathbf{A}(u_1)$;
\textbf{Arc length}: $s = \int_{u_1}^{u_2} \sqrt{\frac{d\mathbf{r}}{du} \cdot
	\frac{d\mathbf{r}}{du}} du$;
\textbf{Tangent vector}: $\frac{d\mathbf{r}}{du} = \left( \frac{dx(u)}{du},
	\frac{dy(u)}{du}, \frac{dz(u)}{du} \right)$;
\textbf{Area of surface}: $A = \iint_\mathcal{R} \left| \frac{d\mathbf{r}}{du} \times 
	\frac{d\mathbf{r}}{dv} \right| dudv$;
\textbf{Gradient of scalar field}: $\Phi = \sum_{i=1}^3 \hat{\mathbf{e}}_i 
	\frac{\partial \Phi}{\partial x_i}$;
Infinitesimal change from $\mathbf{r}$ to $\mathbf{r} + d\mathbf{r}$: 
	$d\Phi = \nabla \Phi \cdot d\mathbf{r}$;
Total derivative along curve $\mathbf{r}(u)$: 
	$\frac{d\Phi}{du} = \nabla \Phi \cdot \frac{d\mathbf{r}}{du}$;
\textbf{Directional derivative} w.r.t. distance $s$ in direction $\mathbf{a}$: 
	$\frac{d\Phi}{ds} = \nabla \Phi \cdot \hat{\mathbf{a}}$;
Largest increase in $\Phi$ is in the direction of $\nabla \Phi$.
$\nabla\Phi$ is normal to the surface $\Phi(x,y,z)=c$ i.e. it is the normal
	derivative.
\textbf{Divergence} of vector field: $\nabla \cdot \mathbf{a} = \sum_{i=1}^3 
	\frac{\partial a_i}{\partial x_i}$;
$\nabla \cdot \mathbf{a} = 0 \Rightarrow \mathbf{a}$ is solenoidal.
\textbf{Curl} of vector field: $\nabla \times \mathbf{a} = \sum_{i,j,k}
	\epsilon \frac{\partial}{\partial x_i} a_j \hat{\mathbf{e}}_k$;
$\nabla \times \mathbf{a} = 0 \Rightarrow \mathbf{a}$ is irrotational / conservative.

\subsubsection{Identities of vector differential operators}
Grad, div and curl are distributive over addition and scalar multiplication.
$\nabla (\Phi\Psi) = \Phi (\nabla \Psi) + (\nabla \Phi) \Psi$;
$\nabla (\mathbf{a} \cdot \mathbf{b}) = \mathbf{a} \times (\nabla \times \mathbf{b}) 
	+ \mathbf{b} \times (\nabla \times \mathbf{a}) 
	+ (\mathbf{a} \cdot \nabla) \times \mathbf{b} 
	+ (\mathbf{b} \cdot \nabla) \times \mathbf{a}$;
$\nabla \cdot (\Phi\mathbf{a}) = \Phi (\nabla \cdot \mathbf{a}) + \mathbf{a} \cdot
	(\nabla \Phi)$;
$\nabla \cdot (\mathbf{a} \times \mathbf{b}) = 
	\mathbf{b} \cdot (\nabla \times \mathbf{a}) 
	- \mathbf{a} \cdot (\nabla \times \mathbf{b})$;
$\nabla \times (\Phi \mathbf{a}) = 
	\nabla \Phi \times \mathbf{a} + \Phi (\nabla \times \mathbf{a})$;
$\nabla \times (\mathbf{a} \times \mathbf{b}) = 
	\mathbf{a} (\nabla \cdot \mathbf{b}) 
	- \mathbf{b} (\nabla \cdot \mathbf{a}) 
	+ (\mathbf{b} \cdot \nabla) \mathbf{a} 
	- (\mathbf{a} \cdot \nabla) \mathbf{b}$;
$\nabla \times (\nabla \Phi) = \mathbf{0}$;
$\nabla \cdot (\nabla \times \mathbf{a}) = \mathbf{0}$;
$\nabla (\nabla \cdot \mathbf{a}) = \sum_{i=1}^3 \hat{\mathbf{e}}_i \left( 
	\sum_{j=1}^3 \frac{\partial^2 a_j}{\partial x_i \partial x_j} \right)$;
$\nabla \cdot (\nabla \Phi) = \nabla^2 \Phi = \sum_{i=1}^3 
	\frac{\partial^2 \Phi}{\partial x_i^2}$;
$\nabla \times (\nabla \times \mathbf{a}) = \nabla (\nabla \cdot \mathbf{a}) 
	- \nabla^2 \mathbf{a}$;
In cartesian coordinates,
	$\nabla^2 \mathbf{a} = \sum_{i,j} \frac{\partial^2 a_j}{\partial x_i^2}
	\hat{\mathbf{e}}_j
	= \sum_{i=1}^3 \nabla^2 a_i \hat{\mathbf{e}}_i$;

\subsubsection{Curvilinear Coordinates}
$\hat{\mathbf{e}}_i = \frac{1}{h_i} \frac{\partial \mathbf{r}}{\partial u_i}$,
$h_i = \left| \frac{\partial \mathbf{r}}{\partial u_i} \right|$;
$\nabla \Phi = \sum_{i=1}^3 \hat{\mathbf{e}}_i' \frac{1}{h_1} \frac{\partial
	\Phi}{\partial u_i}$;
$\nabla \mathbf{a} = \frac{1}{h_1h_2h_3} \sum_{i=1}^3 
	\frac{\partial} {\partial u_i} \left( \frac{a_1h_1h_2h_3}{h_i} \right)$;
$\nabla^2 \Phi = \frac{1}{h_1h_2h_3} \sum_{i=1}^3 
	\frac{\partial \Phi}{\partial u_i}
	\left( \frac{h_1h_2h_3}{h_i^2} \frac{\partial \Phi}{\partial u_i} \right)$;
\begin{flushleft}
	\nabla \time \mathbf{a} = \frac{1}{h_1h_2h_3} \left| \begin{array}{ccc}
		h_1 \hat{\mathbf{e}}_1' 
		& h_2 \hat{\mathbf{e}}_2' 
		& h_3 \hat{\mathbf{e}}_3' \\
		\frac{\partial}{\partial u_1}
		& \frac{\partial}{\partial u_2}
		& \frac{\partial}{\partial u_3} \\
		a_1h_1
		& a_2h_2
		& a_3h_3 \\
\end{array} \right|;
\end{flushleft}

Cylindrical coordinates: $(\rho, \phi, z), h_1 = 1, h_2 = \rho, h_3 = 1$;
Spherical coordinates: $(r, \theta, \phi), h_1 = 1, h_2 = r, h_3 = r\sin\theta$;

\subsection{Lecture 8: Line, Surface and Volume Integrals}

\subsubsection{Line integrals}
$\int_\mathcal{C} \Phi d\mathbf{r} = \sum_{k=1}^3 \hat{\mathbf{e}}_k
	\int_\mathcal{C} \Phi dx_k$;
$\int_\mathcal{C} \mathbf{a} \cdot d\mathbf{r} = \sum_{k=1}^3 \int_\mathcal{C}
	a_k dx_k = \int_\mathcal{C} \mathbf{a} \cdot \mathbf{r}'(u) du $;
$\int_\mathcal{C} \mathbf{a} \times d\mathbf{r} = \sum_{i,j,k} \epsilon_{ijk} 
	\hat{\mathbf{e}}_k \int_\mathcal{C} a_i dx_j$;
Line integrals w.r.t. arc lengths:
$\int_\mathcal{C} \Phi ds = \int_\mathcal{C} \Phi \sqrt{ \frac{d\mathbf{r}}{du} \cdot
	\frac{d\mathbf{r}}{du} }du$;
$\int_\mathcal{C} \mathbf{a} ds = \sum_{k=1}^3 \hat{\mathbf{e}}_k
	\int_\mathcal{C} a_k \sqrt{ \frac{d\mathbf{r}}{du} \cdot \frac{d\mathbf{r}}{du}
	}du$;

\subsubsection{Surface integrals}
$\iint_\mathcal{S} \Phi dS = \iint_\mathcal{S} \Phi \left| \frac{\partial
	\mathbf{r}}{\partial u} \times \frac{\partial \mathbf{r}}{\partial v} \right|
	dudv$;
$\iint_\mathcal{S} \Phi d\mathbf{S} = \pm \iint_\mathcal{S} \Phi \cdot
	\hat{\mathbf{n}} dS = \pm \iint_\mathcal{S} \Phi \left( \frac{\partial
	\mathbf{r}}{\partial u} \times \frac{\partial \mathbf{r}}{\partial v} \right)
	dudv$;
Projecting $S$ onto the $xy$-plane:
$\hat{\mathbf{n}} = \frac{\nabla F}{|\nabla F|},\ dxdy = \hat{\mathbf{e}}_z \cdot 
	dS \hat{\mathbf{n}}$, evaluating $\nabla F$ and $\partial F/\partial x$ in
	terms of $x$ and $y$,
$\iint_\mathcal{S} \Phi dS = \iint_\mathcal{S} \Phi \frac{|\nabla F| dxdy}
	{\partial F/\partial z}$,
$\iint_\mathcal{S} \Phi d\mathbf{S} = \pm \iint_\mathcal{S} \Phi \frac{\nabla F
	dxdy}{\partial F/\partial z}$;

\subsubsection{Volume integrals}
$\iiint_V \mathbf{a} dV = \sum_{k=1}^3 \hat{\mathbf{e}}_k \iiint_V a_k dxdydz$;

\subsubsection{Parametrizations}
Sphere: $\mathbf{r}(\theta, \phi) = (r\sin\theta\cos\phi, r\sin\theta\sin\phi, 
	r\cos\theta)$,
	$r_\theta \times r_\phi = (r^2 \sin^2\theta\cos\phi, r^2\sin^2\theta\sin\phi,
	r^2 \sin\theta\cos\theta)$;
	$|r_\theta \times r_\phi| = r^2 \sin\theta$;
Ellipse: $\mathbf{r}(\theta) = (a\cos\theta, b\sin\theta)$;
Ellipsoid: $\mathbf{r}(\theta, \phi) = (a\sin\theta\cos\phi, b\sin\theta\sin\phi, 
	c\cos\theta)$;

\subsection{Lecture 8: Line, Surface and Volume Integrals}
Fundamental theorem for gradient: $\int_a^b (\nabla \Phi) \cdot dr = \Phi(b) -
	\Phi(a)$;
Divergence / Gauss' theorem: $\iiint_{\mathcal{V}} \nabla \cdot \mathbf{a} dV
	= \oiint_{\mathcal{S}} \mathbf{a} \cdot d\mathbf{S}$;
Stoke's theorem: $\iint_{\mathcal{S}} (\nabla \times \mathbf{a}) \cdot d\mathbf{S}
	= \oint_{\mathcal{C}} \mathbf{a} \cdot d\mathbf{r}$;

\subsubsection{Green's theorem}
$\oint [f(x,y)dx + g(x,y) dy] = \iint (\frac{\partial g}{\partial x}
	- \frac{\partial f}{\partial y} dxdy$;
Two-dimensional divergence theorem: $\iint_{\mathcal{S}} \left[
	\frac{\partial a_x}{\partial x} + \frac{\partial a_y}{\partial y} \right] dxdy
	= \oint_{\mathcal{C}} [a_xdy - a_ydx]$;
Two-dimensional Stoke's theorem: $\iint_{\mathcal{S}} \left[
	\frac{\partial a_y}{\partial x} - \frac{\partial a_x}{\partial y} \right] dxdy
	= \oint_{\mathcal{C}} [a_xdx + a_ydy]$;

Green's first identity:
	$\iiint_{\mathcal{V}} (\nabla \Psi \cdot \nabla \Phi + \Psi \nabla^2 \Phi) dV
	= \oiint_{\mathcal{S}} \Psi \nabla \Phi \cdot d\mathbf{S}$;
Green's second identity:
	$\iiint_{\mathcal{V}} (\Psi \nabla^2 \Phi - \Phi \nabla^2 \Psi) dV
	= \oiint (\Psi \nabla \Phi - \Phi \nabla \Psi) \cdot d\mathbf{S}$;
Divergence-related theorems: 
	$\mathbf{a} = \Phi\mathbf{c} \Rightarrow \iiint_{\mathcal{V}} \nabla \Phi dV
	= \oiint_{\mathcal{S}} \Phi d\mathbf{S}$,
	$\mathbf{a} = \mathbf{b} \times \mathbf{c} \Rightarrow \iiint_{\mathcal{V}}
	\nabla \times \mathbf{b} dV
	= \oiint_{\mathcal{S}} d\mathbf{S} \times \mathbf{b}$;
Stoke's-related theorems: 
	$\mathbf{a} = \Phi\mathbf{c} \Rightarrow \iint_{\mathcal{V}} d\mathbf{S}
	\times \nabla \Phi
	= \oint_{\mathcal{C}} \Phi d\mathbf{r}$,
	$\mathbf{a} = \mathbf{b} \times \mathbf{c} \Rightarrow \iint_{\mathcal{S}}
	(d\mathbf{S} \times \nabla) \times \mathbf{b}
	= \oint_{\mathcal{S}} d\mathbf{r} \times \mathbf{b}$;

\subsubsection{Irrotational and solenoidal fields}

For irrotational $F$, $\nabla \times \mathbf{F} = 0$,
	$\oint_{\mathcal{C}} \mathbf{F} \cdot d\mathbf{r} =
	\iint_{\mathcal{S}} (\nabla \times \mathbf{F}) \cdot d\mathbf{S} = 0$;
$\int_{\mathcal{C}_1} \mathbf{F} \cdot d\mathbf{r} = \int_{\mathcal{C}_2}
	\mathbf{F} \cdot d\mathbf{r}$;
$\exists \text{ scalar potential } \Phi, C \text{ s.t. } \mathbf{F} = \pm
	\nabla (\Phi + C)$;

For solenoidal field, $G$, $\nabla \cdot \mathbf{G} = 0$,
	$\oiint_{\mathcal{S}} \mathbf{G} \cdot d\mathbf{S} =
	\iiint_{\mathcal{V}} \nabla \cdot \mathbf{G} dV = 0$;
$\int_{\mathcal{S}_1} \mathbf{G} \cdot d\mathbf{S} = \int_{\mathcal{S}_2}
	\mathbf{G} \cdot d\mathbf{S}$;
$\exists \text{ vector potential } \mathbf{A}, \nabla\Phi \text{ s.t. }
	\mathbf{G} = \nabla \times (\mathbf{A} + \nabla \Phi)$;

Helmholtz's theorem: $\forall \mathbf{A},\ \exists \Phi, \mathbf{A} \text{ s.t. }
	\mathbf{F} = -\nabla \Phi + \nabla \times \mathbf{A}$;
Assuming $\mathbf{F}$ goes to zero sufficient rapidly at $\infty$,
	$\Phi(\mathbf{r}) \frac{1}{4\pi} \iiint
	\frac{\nabla' \cdot \mathbf{F}(\mathbf{r}')}
	{|\mathbf{r} - \mathbf{r}'|} dx'dy'dz'$,
	$\mathbf{A}(\mathbf{r}) \frac{1}{4\pi} \iiint
	\frac{\nabla' \times \mathbf{F}(\mathbf{r}')}
	{|\mathbf{r} - \mathbf{r}'|} dx'dy'dz'$;

\subsection{Lecture 10: Fourier Series}
Dirichlet conditions: 
	(i) periodic, 
	(ii) single-valued and continuous except at finitely many points of finite
		discontinuities, 
	(iii) finite number of maxima and minima within period, 
	(iv) integral of $|f(x)|$ over period converges.
Dirichlet conditions satified $\Rightarrow$ Fourier series converges where
$f(x)$ is continuous.
At discontinuity $x = x_0$, $f_{\text{FS}}(x_0) = \lim_{\epsilon\to0}
	\left[ \frac{(x_0 + \epsilon) + (x_0 - \epsilon)}{2} \right]$;

Average of $f(x)$ over $(a,b)$: $\langle f(x) \rangle = \frac{1}{b-a} \int_a^b
	f(x) dx$;

\subsubsection{Useful trigonometric identities}
$\cos A \cos B = \frac{1}{2} [\cos(A-B) + \cos(A+B)]$;
$\sin A \sin B = \frac{1}{2} [\cos(A-B) - \cos(A+B)]$;
$\sin A \cos B = \frac{1}{2} [\sin(A-B) + \sin(A+B)]$;
$\langle \cos(nx) \rangle = \langle \sin(nx) \rangle = 0$;
$\langle \cos^2(nx) \rangle = \langle \sin^2(nx) \rangle = \frac{1}{2}$;

\textbf{$2\pi$-periodic function}: $f(x) = \frac{a_0}{2} + \sum_{n=1}^\infty a_n \cos(nx)
	+ \sum_{n=1}^\infty b_n \sin(nx)$,
$a_n = \frac{1}{\pi} \int_{-\pi}^\pi \cos(nx) f(x) dx,\ 
	b_n = \frac{1}{\pi} \int_{-\pi}^\pi \sin(nx) f(x) dx$;
\textbf{$2L$-periodic function}: $f(x) = \frac{a_0}{2} + \sum_{n=1}^\infty a_n
	\cos \left( \frac{n\pi x}{L} \right) + \sum_{n=1}^\infty b_n 
	\sin \left( \frac{n\pi x}{L} \right)$,
$a_n = \frac{1}{L} \int_{-\pi}^\pi \cos \left( \frac{n\pi x}{L} \right) f(x) dx,\ 
	b_n = \frac{1}{L} \int_{-\pi}^\pi \sin \left( \frac{n\pi x}{L} \right) f(x) dx$;
\textbf{Even function cosine series}:
$f(x) = \frac{a_0}{2} + \sum_{n=1}^\infty a_n
	\cos \left( \frac{n\pi x}{L} \right) + \sum_{n=1}^\infty b_n$,
$a_n = \frac{2}{L} \int_0^L \cos \left( \frac{n\pi x}{L} \right) f(x) dx$;
\textbf{Odd function sine series}:
$f(x) = \sum_{n=1}^\infty b_n \sin \left( \frac{n\pi x}{L} \right)$,
$b_n = \frac{2}{L} \int_0^L \sin \left( \frac{n\pi x}{L} \right) f(x) dx$;
Integral of $f_{\text{FS}}(x)$ is a Fourier representation of $F(x)$ iff 
	$\langle f(x) \rangle = 0$.
Derivative of $f_{\text{FS}}(x)$ is a Fourier representation of $f'(x)$ iff 
	$f'(x)$ satisfies Dirichlet conditions.

\textbf{Complex Fourier series}:
$f(x) = \sum_{n=-\infty}^\infty c_n \exp \left( i\frac{n\pi x}{L} \right)$;
$c_n = \frac{1}{2L} \int_{-L}^L \exp \left( -i\frac{n\pi x}{L} \right) f(x) dx$;
Parseval's theorem:
	$\left\langle [f(x)]^2 \right\rangle
	= \left( \frac{a_0}{2} \right)^2 + \frac{1}{2} \sum_{n=1}^\infty (a_n^2 + b_n^2)
	= \sum_{n=-\infty}^\infty |c_n|^2$;

\subsubsection{Common Fourier Series}
For $-\pi < x < \pi$,
$x = 2 \sum_{n=1}^\infty \frac{(-1)^{n+1}}{n} \sin (nx)$;
$|x| = \frac{\pi}{2} - \frac{4}{\pi} \sum_{n=1}^\infty \frac{1}{(2n-1)^2} \cos
	[(2n-1)x]$; 
$0, x = \frac{\pi}{4} - \frac{2}{\pi} \sum_{n=1}^\infty \frac{1}{(2n-1)^2} \cos
	[(2n-1)x] + \sum_{n=1}^\infty \frac{(-1)^{n+1}}{n} \sin (nx)$;
$-1, 1 = \frac{4}{\pi} \sum_{n=1}^\infty \frac{1}{2n-1} \sin [(2n-1)x]$;
$0, 1 = \frac{1}{2} + \frac{2}{\pi} \sum_{n=1}^\infty \frac{1}{2n-1} \sin [(2n-1)x]$;
$|\sin x| = \frac{2}{\pi} - \frac{4}{\pi} \sum_{n=1}^\infty \frac{1}{4n^2-1}
	\cos (2nx)$;
$|\cos x| = \frac{2}{\pi} - \frac{4}{\pi} \sum_{n=1}^\infty
	\frac{(-1)^n}{4n^2-1} \cos (2nx)$;
$0, \sin x = \frac{1}{\pi} + \frac{1}{2} \sin x- \frac{2}{\pi}
	\sum_{n=1}^\infty \frac{1}{4n^2-1} \cos (2nx)$;

\subsection{Lecture 11: Fourier Transform}
Wavelength: $\lambda_n = \frac{2L}{n}$, 
	wave number: $\frac{2\pi}{\lambda_n} = \frac{n\pi}{L}$,
	$\Delta k_n = k_{n+1} - k_n = \frac{\pi}{L}$;
Taking $L \to \infty, \Delta k_n \to 0$ yields
\textbf{Fourier's inversion theorem}: 
	$f(x) = \frac{1}{2\pi} \int_{-\infty}^\infty e^{ikx} \left[
	\int_{-\infty}^\infty e^{-iku} f(u) du \right] dk$
	$= \frac{1}{\sqrt{2\pi}} \int_{-\infty}^\infty e^{ikx} \tilde{f}(k) dk$;
	$\tilde{f}(k) = \frac{1}{\sqrt{2\pi}} \int_{-\infty}^\infty e^{-ikx} f(x) dx$;

\subsubsection{Properties of Fourier Transform}
$\mathcal{F}\{f(x)\} = \tilde{f}(k)$;
$\mathcal{F}$ is linear and invertible;
$\mathcal{F}\{ f(-x) \} = \tilde{f}(-k)$;
$\mathcal{F}\{ f^*(x) \} = \tilde{f}^*(k)$;
$\mathcal{F}\{ f^{(n)}(x) \} = (ik)^n \tilde{f}(k)$;
$\mathcal{F}\{ \int_{-\infty}^x f(u) du \} = \frac{1}{ik} \tilde{f}(k)$;
$\mathcal{F}\{ \tilde{f}(x) \} = f(-k)$;
$\mathcal{F}\{ f(x+a) \} = e^{ika} \tilde{f}(k)$;
$\mathcal{F}^{-1}\{ \tilde{f}(k+a) \} = e^{-ika} f(x)$;
$\mathcal{F}\{ f(ax) \} = \frac{1}{|a|} \tilde{f}\left( \frac{k}{a} \right)$;
$\mathcal{F}^{1}\{ \tilde{f}(ak) \} = \frac{1}{|a|} f\left( \frac{x}{a} \right)$;
$\tilde{f}(0) = \int_{\infty}^\infty f(x) dx$;
$\mathcal{F}\{ xf(x) \} = i\tilde{f}'(k)$;
$\mathcal{F}\{ x^nf(x) \} = (i)^n \tilde{f}^{(n)}(k)$;

Parseval's identity:
	$\int_{-\infty}^\infty f(x)g(x) dx = \int_{-\infty}^\infty
	\tilde{f}(k)\tilde{g}(k) dk$
Parseval's theorem:
	$\int_{-\infty}^\infty |f(x)|^2 dx = \int_{-\infty}^\infty
	|\tilde{f}(k)|^2 dk$
Convolution: $f(x) \ast g(x) = \int_{-\infty}^\infty f(y) g(x-y) dy$;
$f \ast g = g \ast f$;
$(f \ast g) \ast h = f \ast (g \ast f)$;
$f \ast (g + h) = f \ast g + f \ast h$;

Convolution theorem for Fourier transforms:
	$\mathcal{F}\{ f(x) \ast g(x) \} = \sqrt{2\pi} \tilde{f}(k) \tilde{g}(k)$;
	$\mathcal{F}\{ f(x) g(x) \} = \frac{1}{\sqrt{2\pi}} \tilde{f}(k) \ast
		\tilde{g}(k)$;
	$\mathcal{F}^{-1}\{ \tilde{f}(x) \ast \tilde{g}(x) \} = \sqrt{2\pi} f(k) g(k)$;
	$\mathcal{F}^{-1}\{ \tilde{f}(x) \tilde{g}(x) \} = \frac{1}{\sqrt{2\pi}}
		f(k) \ast g(k)$;
	
Three-dimensional Fourier transform:
$\tilde{f}(\mathbf{k}) = \frac{1}{(2\pi)^{3/2}} \iiint 
	e^{-i \mathbf{k} \cdot \mathbf{r}} f(\mathbf{r}) d^3\mathbf{r}$;
$f(\mathbf{r}) = \frac{1}{(2\pi)^{3/2}} \iiint 
	e^{-i \mathbf{k} \cdot \mathbf{r}} \tilde{f}(\mathbf{k}) d^3\mathbf{k}$;

Common Fourier Transfroms:
$\mathcal{F}\{ e^{-a|x|} \} = 2a (k^2 + a^2)^{-1}$;
$\mathcal{F}\{ (x^2 + a^2)^{-1} \} = \frac{1}{2a} e^{-a|k|}$;
$\mathcal{F}\{ \frac{\exp(-ax)}{x} \} = \frac{4\pi}{k^2 + \alpha^2}$;

\subsection{Lecture 12: Dirac Delta Distribution}
From Fourier's inversion theorem, $f(x) = \frac{1}{2\pi} \int_{-\infty}^\infty 
	e^{ikx} \left[ \int_{-\infty}^\infty e^{-ikx'} f(x') dx' \right] dk$
	$\Rightarrow f(x) = \int_{-\infty}^\infty f(x')
	\left[ \frac{1}{2\pi} \int_{-\infty}^\infty e^{ik(x-x')} dk \right] dx'$
$\delta(x-x') := \frac{1}{2\pi} \int_{-\infty}^\infty e^{ik(x-x')} dk$;

\subsubsection{Properties of $\delta$-distribution}
$\delta(x-x') = \delta(x'-x)$;
$\int_{-\infty}^\infty f(x) \delta^{(n)}(x-x') dx = (-1)^n f^{(n)} (x')$;
$g(a_k) = 0,\ g'(a_k) \ne 0,\ \delta[g(x)] = \sum_k \frac{\delta(x-a_k)}{|g'(a_k)|}$;

\subsubsection{Heaviside Step Function}
$H(x-x') = 1,\ x > x',\ = 0,\ x < x'$;
$|x-y| = (x-y)[ H(x-y) - H(y-x)]$;
$g(x) = g_1(x) H(x) H(k-x) + g_2(x) H(x-k)$;
$H'(x-x') = \delta(x-x')$;

\subsubsection{Fourier transforms involving $\delta$-distribution}
$\mathcal{F}\{e^{ik'x}\} = \sqrt{2\pi} \delta(k-k')$;
$\mathcal{F}\{ \cos x \} = \sqrt{\frac{\pi}{2}} [\delta(k-1) + \delta(k+1)]$;
$\mathcal{F}\{ \sin x \} = i\sqrt{\frac{\pi}{2}} [-\delta(k-1) + \delta(k+1)]$;
$\mathcal{F}\{ \delta(x-x') \} = \frac{1}{\sqrt{2\pi}} e^{ikx'}$;

\subsubsection{Three dimensional $\delta$-distribution}
$\iiint_{\varepsilon} \delta^3(\mathbf{r}-\mathbf{r}') d^3 \mathbf{r} = 1$;
$\delta^3(\mathbf{r}-\mathbf{r}') = \delta(x-x')\delta(y-y')\delta(z-z')$;
Curvilinear orthogonal coordinates:
$h_i = \sqrt{ 
	\left( \frac{\partial x}{\partial u_i} \right)^2
	+ \left( \frac{\partial y}{\partial u_i} \right)^2
	+ \left( \frac{\partial z}{\partial u_i} \right)^2
}$;
$\delta^3(\mathbf{r}-\mathbf{r}') = 
	\frac{\delta(u-u')}{h_1}
	\frac{\delta(v-v')}{h_2}
	\frac{\delta(z-z')}{h_3}$;
Cylindrical coordinates:
$\delta^3(\mathbf{r}-\mathbf{r}') = 
	\frac{1}{\rho}\delta(\rho-\rho')\delta(\phi-\phi')\delta(z-z')$;
Spherical coordinates:
$\delta^3(\mathbf{r}-\mathbf{r}') = 
	\frac{1}{r^2\sin\theta}\delta(r-r')\delta(\theta-\theta')\delta(\phi-\phi')$;

\subsubsection{Applications}
A collection of point charge $Q-K$ located at $\mathbf{r}_k$:
	$\rho_q(x,y,z) = \sum_k Q_k \delta(x-x_k)\delta(y-y_k)\delta(z-z_k)$; \\
A collection of point charge $Q-K$ located at $\mathbf{r}_k$ on a sphere of radius $a$:
	$\rho_q(r,\theta,\phi) = \frac{\delta(r-a)}{a^2} \sum_k
	\frac{Q_k}{\sin\theta_k} \delta(\theta-\theta_k)\delta(\phi-\phi_k)$; \\
A uniformly charged shell with total charge $Q$ and radius $a$:
	$\rho_q(r,\theta,\phi) = \frac{Q}{4\pi a^2} \delta(r-a)$; \\
An origin-centered uniformly charged ring with total charge $Q$ and radius $a$
	on the $xy$-plane :
	$\rho_q(\rho,\phi,z) = \frac{Q}{2\pi a} \delta(\rho-a) \delta(z)$;

\subsection{Lecture 13 \& 14: Second-Order Linear ODEs}
$\mathcal{L}y(x) := \frac{d^2y}{dx^2} + P\frac{dy}{dx} + Qy = F$;
General solution: $y(x) = Y_c + Y_p = c_1y_1 + c_2y_2 + Y_p$, where
	complementary function: $\mathcal{L}Y_c = 0$, particular integral:
	$\mathcal{L}Y_p = F$,
	known as a particular solution once $c_1, c_2$ are fixed.
At most two linearly independent solutions, Wronskian $W(x) = y_1y_2' - y_2y_1'
	= C\exp\left[-\int^x P(u)du \right] \ne 0$;

\subsubsection{Ordinary and singular points}
Ordinary points: $P, Q$ finite as $x \to x_0$;
Regular singular points: $P \lor Q$ diverges as $x \to x_0$ but $(x-x_0)P$ and
$(x-x_0)^2Q$ remains finite;
Irregular singular points: $(x-x_0)P \lor (x-x_0)^2Q$ diverges;
Checking for $|x|=\infty$: Let $u=x^{-1}$, $\frac{dy}{dx} = -u^2\frac{dy}{du}$,
	$\frac{d^2y}{dx^2} = u^3 \left( 2\frac{dy}{du} + u \frac{d^2y}{du^2} \right)$;

\subsubsection{Series solutions}
Frobenius series: $y(x) = (x-x_0)^\sigma \sum_{n=0}^\infty a_n (x-x_0)^n,\ a_0 \ne 0$,
	practically expanded about $x_0=x$ with substitution $u=x-x_0$;
Fuch's theorem: $\exists$ series solution if $x=x_0$ is ordinary or regular singular;
Radius of convergence is generally equal to neary singularity.
$\frac{d^2y}{dx^2} + \frac{s}{x}\frac{dy}{dx} + \frac{t}{x^2}y = 0$,
	$s = xP, t = x^2Q$;
After substitution, $\sum_{n=0}^\infty [(n+\sigma)(n+\sigma-1) + s(n+\sigma) 
	+ t]a_nx^n = 0$;
\textbf{Indicial equation}:
	$\sigma(\sigma-1) + s(0)\sigma + t(0) = 0$;
Substitute $\sigma_1, \sigma_2$ and obtain a recurrence to yield a series solution. 
If $x=0$ is ordinary, $\sigma_1 = 1, \sigma_2 = 0$;
$\sigma_1 = \sigma_2 \Rightarrow$ one solution;
$|\sigma_1 - \sigma_2| \not\in \mathbb{Z} \Rightarrow$ two solutions;
$|\sigma_1 - \sigma_2| \in \mathbb{Z} \Rightarrow$ max$\{\sigma_1, \sigma_2\}$ is guaranteed to yield a solution.

\subsubsection{Obtaining a second solution}
First-order linear ODE: $y' + Py = Q$, integrating factor: $\mu = \exp \left[ 
	\int^x P(\xi)d\xi \right]$, $y = \frac{1}{\mu} \int^x \mu(\xi) Q(\xi) d\xi$;

Second solution found by solving $y_2' - \frac{1}{y_1} y_1' y_2 = \frac{W}{y_1}$, \\
\textbf{second solution}: $y_2 = y_1 \int^x \frac{1}{[y_1(\xi)]^2}
	\exp \left[ - \int^\xi P(u) du \right] d\xi$;
Special case: $y'' = Qy \Rightarrow y_2 = y_1 \int^x \frac{d\xi}{[y_1(\xi)]^2}$;

Series forms of $P, Q$, $P = \sum_{i=-1}^\infty p_ix^i,\ Q =
	\sum_{j=-2}^\infty q_jx^j$,
	$p_{-1} = q_{-2} = q_{-1} = 0$ if $x=0$ is ordinary; \\
Indicial equation yields: $\sigma^2 + (p_{-1} -1)\sigma + q_{-2} = 0$; \\
For $\sigma_1 = \alpha,\ \sigma_2 = \alpha - N,\ N \in \mathbb{Z}^+\cup\{0\}$,
	$p_{-1} = N-2\alpha + 1, q_{-2} = \alpha(\alpha - N)$;
$\int^\xi P(u) du = \xi^{-p_{-1}} \sum_{m=0}^\infty p_m' \xi^m$,
$\frac{1}{[y_1(\xi)]^2} = \xi^{-2\alpha} \sum_{n=0}^\infty a_n'\xi^n$,
$\frac{1}{[y_1(\xi)]^2} \int^\xi P(u) du = \xi^{-N-1} \sum_{k=0}^\infty c_k \xi^k$;
\textbf{Series form of second solution}, $y_2 = c y_1 \ln|x| + x^{\sigma_2}
	\sum_{n=0}^\infty b_n x^n$;

\text{Particular integral}: $Y_p = -\int^x \frac{f(u)[y_2(u)y_1(x) - y_1(u)y_2(x)]}
	{y_1(u)y_2'(u) - y_2(u)y_1'(u)} du$;

\textbf{Constant coefficients}: $\lambda = \frac{-P_0 \pm \sqrt{P_0^2 - 4Q_0}}{2}$;
$\lambda_+ \ne \lambda_- \Rightarrow 
	Y_c = c_1 \exp (\lambda_+ x) + c_2 \exp(\lambda_- x)$,
	$Y_p = \int^x \frac{\exp[\lambda_+(x-u) - \exp[\lambda_-(x-u)]}
		{\lambda_+ - \lambda_-} f(u) du$;
$\lambda_+ = \lambda_- = \lambda \Rightarrow 
	Y_c = c_1 \exp (\lambda x) + c_2 x\exp(\lambda x)$,
	$Y_p = \int^x (x-u) \exp [\lambda(x-u)] f(u) du$;

\subsection{Lecture 15: Laplace Transform}
$\mathcal{L}\{f(t)\} = \bar{f}(s) = \int_0^\infty e^{-st} f(t) d t$,
	$\bar{f}(x) \to 0$ as $s \to \infty$,
	$s$ can be chosen sufficiently large s.t. $e^{-st}f(t)$ converges even if
	$f(t) \not\to 0$ as $t \to \infty$; \\
$\mathcal{L}$ is linear;

\subsubsection{Common Laplace Transforms}
$\mathcal{L} \{ c \} = \frac{c}{s}$;
$\mathcal{L} \{ ct^n \} = \frac{cn!}{s^{n+1}}$;
$\mathcal{L} \{ \sin (bt) \} = \frac{b}{s^2 + b^2}$;
$\mathcal{L} \{ \cos (bt) \} = \frac{s}{s^2 + b^2}$;
$\mathcal{L} \{ \sinh (at) \} = \frac{a}{s^2 - a^2}$;
$\mathcal{L} \{ \cosh (at) \} = \frac{s}{s^2 - a^2}$;
$\mathcal{L} \{ e^{at} \} = \frac{1}{s - a}$;
$\mathcal{L} \{ t^n e^{at} \} = \frac{n!}{(s - a)^{n+1}}$;

\subsubsection{Properties of Laplace Transform}
$\mathcal{L} \left\{ \frac{d^n f(t)}{d t^n} \right\} 
	= s^n \bar{f}(x) - \sum_{r=0}^{n-1} s^{n-1-r} f^{(r)} (0)$,
	e.g.  $\mathcal{L} \left\{ \frac{d^2 f(t)}{d t^n} \right\} 
	= s^2 \bar{f}(s) - sf(0) - f'(0)$;
$\mathcal{L} \left\{ \int_0^t f(u) du \right\} = \frac{1}{s} \bar{f}(s)$;

$\mathcal{L} \{ f(at) \} = \frac{1}{a} \bar{f} \left( \frac{s}{a} \right)$;
$\mathcal{L} \{ e^{a} f(t) \} = \bar{f} (s-a)$;
$\mathcal{L} \{ t f(t) \} = -\bar{f}' (s)$;
$\mathcal{L} \{ t^n f(t) \} = (-1)^n \bar{f}^{(n)} (s)$;
$\mathcal{L} \{ H(t-a) f(t-a) \} = e^{-sa}\bar{f} (s)$;

$\frac{d^n}{d t^n} \mathcal{L} \left\{ f(t) \right\} = \mathcal{L} \{ (-t)^n f(t) \}$;
$\int_s^\infty \mathcal{L} \left\{ f(t) \right\} ds' 
	= \mathcal{L} \left\{ \frac{f(t)}{t} \right\}$;

\textbf{Laplace convolution theorem}: $\mathcal{L} \{ f(t) \ast g(t) \} 
	= \bar{f}(s) \bar{g}(s)$;
$\mathcal{L}^{-1} \{ \bar{f}(s) \bar{g}(s) \} = f(t) \ast g(t)$;

\textbf{Evaluating improper integrals}: Direct substitution i.e. $\int_0^\infty
	e^{-at} f(t) dt = \left[ \int_0^\infty e^{-st} f(t) dt \right]_{s=a}
	= [ \mathcal{L} \{ f(t) \} ]_{s=a}$,
	Integral of transform i.e.
	$\int_0^\infty \frac{f(t)}{t} dt = \left[ \int_0^\infty e^{-st} 
	\frac{f(t)}{t} dt \right]_{s=0} = \left[ \mathcal{L} \left\{ 
	\frac{f(t)}{t} \right\} \right]_{s=0} = \left[ \int_s^\infty 
	\mathcal{L} \{ f(t) \} ds' \right]_{s=0}$;
	Find Laplace transform and take inverse i.e. $I = \mathcal{L}^{-1} 
	(\mathcal{L} I )$;
Solving linear ODEs: take Laplace transform of ODE, solve algebraic equation
	for $\bar{f}(s)$ and take $\mathcal{L}^{-1}$;
% You can even have references
%\rule{0.3\linewidth}{0.25pt}
%\scriptsize
%\bibliographystyle{abstract}
%\bibliography{refFile}

\end{multicols}
\end{document}
